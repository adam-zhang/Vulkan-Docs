% Copyright (c) 2013-2016 The Khronos Group Inc.
%
% Permission is hereby granted, free of charge, to any person obtaining a
% copy of this software and/or associated documentation files (the
% "Materials"), to deal in the Materials without restriction, including
% without limitation the rights to use, copy, modify, merge, publish,
% distribute, sublicense, and/or sell copies of the Materials, and to
% permit persons to whom the Materials are furnished to do so, subject to
% the following conditions:
%
% The above copyright notice and this permission notice shall be included
% in all copies or substantial portions of the Materials.
%
% THE MATERIALS ARE PROVIDED "AS IS", WITHOUT WARRANTY OF ANY KIND,
% EXPRESS OR IMPLIED, INCLUDING BUT NOT LIMITED TO THE WARRANTIES OF
% MERCHANTABILITY, FITNESS FOR A PARTICULAR PURPOSE AND NONINFRINGEMENT.
% IN NO EVENT SHALL THE AUTHORS OR COPYRIGHT HOLDERS BE LIABLE FOR ANY
% CLAIM, DAMAGES OR OTHER LIABILITY, WHETHER IN AN ACTION OF CONTRACT,
% TORT OR OTHERWISE, ARISING FROM, OUT OF OR IN CONNECTION WITH THE
% MATERIALS OR THE USE OR OTHER DEALINGS IN THE MATERIALS.

\documentclass{article}

% Various packages of possible use. {index} creates problems for some reason.
% \usepackage{index}
\def\specpdftitle{The Khronos Vulkan API Registry}
\usepackage[pdftex,bookmarksnumbered=true,linktocpage,plainpages=false,pdftitle={\specpdftitle},colorlinks=true]{hyperref}
\usepackage{amsmath,enumerate,epsfig,framed,mdframed,multicol,longtable,times,url}
\usepackage[normalem]{ulem}
% underscore allows use of _ instead of \_ in text. [strings] allows _ in file names
\usepackage[strings]{underscore}
\usepackage[breakwords,fit]{truncate}

% Index everything down through \paragraph
\setcounter{tocdepth}{5}

\makeatletter
\makeatother
\makeindex

% Some commonly used abbreviations

\def\code#1{{\tt #1}}
\def\tag#1{{\tt <#1>\index{<#1>}}}
% Because I keep mistyping \tag
\def\arg#1\tag{#1}
\def\attr#1{{\tt #1\index{#1}}}

\begin{document}

\title{The Khronos Vulkan API Registry for Vulkan}
\author{Jon Leech}
\date{Last updated 2016/02/07}
\maketitle

\begin{abstract}

This document describes the Khronos Vulkan API Registry schema, and provides
some additional information about using the registry and scripts to generate
a variety of outputs including C header files and several types of asciidoc
include files used in the Vulkan API specification and reference pages. The
underlying XML files and scripts are located on the Khronos Gitlab server,
currently in the {\em master} branch at URL

\begin{center}
{\bf \href{https://gitlab.khronos.org/vulkan/vulkan/tree/master/src/spec/}
          {https://gitlab.khronos.org/vulkan/vulkan/tree/master/src/spec/}}
\end{center}

\end{abstract}

\tableofcontents

\section{Introduction}

The registry uses an XML representation of the Vulkan API and a set of
Python scripts to manipulate the XML, based on the lxml Python bindings. It
includes an XML schema and validator.

The registry schema is based on, but not identical to the schema used for
the previously published OpenGL, OpenGL ES and EGL API registries. The
schema was extended to represent additional types and concepts not needed
for those APIS, such as structure and enumerant types.

The processed C header file corresponding to the registry is checked in
under src/vulkan/vulkan.h .


\section{Getting Started}
\label{starting}

You will need some tools installed.
For Linux systems, you will need the following packages installed (Debian
package names are noted, Ubuntu package names are probably the same).

\begin{itemize}
\item Python 3.x (\code{python3})
\item The lxml Python package (\code{python3-lxml})
\item Libxml (\code{libxml2})
\item A make tool, such as GNU make
\end{itemize}

The toolchain has also been run successfully on Microsoft Windows in the
Cygwin environment, and the individual Python-based components of the
toolchain can be run directly from the Windows command line. More details on
this are under src/spec/README in Git.

Once you have the right tools installed, perform the following steps:

\begin{itemize}
\item Check out the ``vulkan'' repository linked above from Khronos Gitlab
      (there are instructions at the link)
\item Go to \code{vulkan/src/spec} in your checked-out repo
\item Invoke \code{make clobber ; make install}
\end{itemize}

This should regenerate \code{vulkan.h} and install it in
\code{../vulkan/vulkan.h}. The result should be identical to the
version you just pulled from Gitlab (use \code{git diff
../vulkan/vulkan.h} to check).

Other Makefile targets include:

\begin{itemize}
\item \code{vulkan-docs} - regenerate the API specification and reference
      page asciidoc include files in \code{../doc/specs/vulkan/} under the
      subdirectories \code{enums}, \code{flags}, \code{protos}, and
      \code{structs}. These files are pulled in by the ref pages and API
      spec to put function prototypes and struct and enum declarations in
      documentation.

      You should probably {\bf not} update these files if you are proposing
      an API change - just the \code{vulkan.h} header. When a git merge is
      accepted including your change, the docs will be updated at that time.
      Including the dozens of include files in your merge makes it hard to
      read and adds no information beyond the header updates. Sometimes it
      is useful to include just one or two of the include files for context
      of a global change, however.

      However, you are strongly encouraged to updated the {\bf
      non-}autogenerated docs under \code{../doc/specs/vulkan/man/*.txt} to
      reflect API changes.
\item \code{full_install} - equivalent to \code{install} followed by
      \code{vulkan-docs}.
\item \code{validate} - validate \code{vk.xml} against the XML schema.
      Recommended if you're making nontrivial changes.
%% \item \code{vk.json} - generate JSON version of \code{vk.xml}. Currently the
%%       converter is generating syntactically but not semantically legal JSON,
%%       and it's difficult to map the XML schemaanso don't use this target yet.
\item \code{readme.pdf} - regenerate this document from the LaTeX source.
      Most people will never need to do this. If you do, you must have
      pdflatex installed, preferably from the TeTeX distribution.
\end{itemize}

If you just want to modify the API, changing \code{vk.xml} and running
``make'' should be all that's needed. See appendix~\ref{examples} for some
examples of modifying the XML.

If you want to repurpose the registry for reasons other than header file
and ref page include generation, or to generate headers for languages
other than C, start with the Makefile rules and inspect the files
\code{vk.xml}, \code{genvk.py}, \code{reg.py}, and \code{generator.py}.

If you're using other platforms, merge requests with additional
documentation on using the tools on those platforms would be very helpful.

\subsection{Header Generation Script - \code{genvk.py}}

When generating header files using the \code{genvk.py} script, an API name
and profile name are required, as shown in the Makefile examples.
Additionally, specific API versions and extensions can be required or
excluded. Based on this information, the generator script extracts the
relevant interfaces and creates a C-language header file for them.
\code{genvk.py} contains predefined generator options for the current
development version of Vulkan 1.0.

The generator script is intended to be generalizable to other languages by
writing new generator classes. Such generators would have to rewrite the C
types and definitions in the XML to something appropriate to their language.

\subsection{Registry Processing Script - \code{reg.py}}

XML processing is done in \code{reg.py}, which contains several objects and
methods for loading registries and extracting interfaces and extensions for
use in header generation. There is some internal documentation in the form
of comments, although nothing more extensive exists yet.

\subsection{Output Generator Script - \code{generator.py}}

Once the registry is loaded, the \code{COutputGenerator} class defined in
\code{generator.py} is used to create a header file. The
\code{DocOutputGenerator} class is used to create the asciidoc include
files. Output generators for other purposes can be added as needed.


\section{Vulkan Registry Schema}
\label{schema}

The format of the Vulkan registry is a top level \tag{registry} tag
containing \tag{types}, \tag{enums}, \tag{commands}, \tag{feature}, and
\tag{extension} tags describing the different elements of an API, as
explained below. This description corresponds to a formal Relax NG schema
file, \code{registry.rnc}, against which the XML registry files can be
validated.

At present the only registry in this schema is the core Vulkan API registry,
\code{vk.xml}.


\subsection{Profiles}
\label{schema:profile}

Types and enumerants can have different definitions depending on the API
profile requested. This capability is not used in the current Vulkan API but
may be in the future. Features and extensions can include some elements
conditionally depending on the API profile requested.


\subsection{API Names}
\label{schema:apiname}

The schema supports, but does not currently use an \attr{api} attribute on
several tags. This is an arbitrary string, specified at header generation
time, for labelling properties of a specific API or API profile. The string
can be, but is not necessarily, an actual API name. Names starting with ``vk''
are suggested if and when we start defining profiles of Vulkan.


\section{Registry Root (\tag{registry} tag)}
\label{schema:root}

A \tag{registry} contains the entire definition of one or more related
APIs.

\subsection{Attributes of \tag{registry} tags}

None.

\subsection{Contents of \tag{registry} tags}

Zero or more of each of the following tags, normally in this order
(although order shouldn't be important):

\begin{itemize}
\item \tag{comment} - Contains arbitrary text, such as a copyright
      statement.
\item \tag{vendorids} (see section~\ref{tag:vendorids}) - defines Khronos
      vendor IDs, described in detail in the ``Layers and Extensions''
      appendix of the Vulkan Specification.
\item \tag{tags} (see section~\ref{tag:tags}) - defines author prefixes used
      for extensions and layers. Prefixes are described in detail in the
      ``Layers and Extensions'' appendix of the Vulkan Specification.
\item \tag{types} (see section~\ref{tag:types}) - defines API types.
      Usually only one tag is used.
\item \tag{enums} (see section~\ref{tag:enums}) - defines API token names
      and values. Usually multiple tags are used. Related groups may be
      tagged as an enumerated type corresponding to a \tag{type} tag, and
      resulting in a C \code{enum} declaration. This ability is heavily used
      in the Vulkan API.
\item \tag{commands} (see section~\ref{tag:commands}) - defines API
      commands (functions). Usually only one tag is used.
\item \tag{feature} (see section~\ref{tag:feature}) - defines API
      feature interfaces (API versions, more or less). One tag per
      feature set.
\item \tag{extensions} (see section~\ref{tag:extensions}) - defines API
      extension interfaces. Usually only one tag is used, wrapping many
      extensions.
\end{itemize}


\section{Vendor IDs (\tag{vendorids} tag)}
\label{tag:vendorids}

The \tag{vendorids} tag contains individual \tag{vendorid} tags defining
vendor IDs for physical devices which do not have PCI vendor IDs.

Each \tag{vendorid} tag contains information defining a single vendor ID.

\subsection{Attributes of \tag{vendorid} tags}

\begin{itemize}
\item \attr{name} - required. The author prefix, as registered with Khronos.
      This must match an author prefix in the \attr{name} field of a
      \tag{tag} tag.
\item \attr{id} - required. The reserved vendor ID, as a hexadecimal number.
\item \attr{comment} - optional. Arbitrary string (unused).
\end{itemize}


\section{Author Prefixes (\tag{tags} tag)}
\label{tag:tags}

The \tag{tags} tag contains individual \tag{tag} tags defining each of the
reserved author prefixes used by extension and layer authors.

Each \tag{tag} tag contains information defining a single author prefix.

\subsection{Attributes of \tag{tag} tags}

\begin{itemize}
\item \attr{name} - required. The author prefix, as registered with Khronos.
      A short, upper-case string, usually an abbreviation of an author,
      project or company name.
\item \attr{author} - required. The author name, such as a full company
      or project name.
\item \attr{contact} - required. The contact who registered or is currently
      responsible for extensions and layers using the prefix, including
      sufficient contact information to reach the contact such as individual
      name together with email address, Github username, or other contact
      information.
\end{itemize}


\section{API types (\tag{types} tag)}
\label{tag:types}

The \tag{types} tag contains individual \tag{type} tags describing each
of the derived types used in the API.

Each \tag{type} tag contains information which can be used to generate C
code corresponding to the type. In many cases, this is simply legal C code
with attributes or embedded tags denoting the type name and other types used
in defining this type. In some cases, additional attribute and embedded type
information is used to generate more complicated C types.


\subsection{Attributes of \tag{type} tags}

\begin{itemize}
\item \attr{requires} - optional. Another type name this type requires to
      complete its definition.
\item \attr{name} - optional. Name of this type (if not defined in the tag
      body).
\item \attr{api} - optional. An API name (see \tag{feature} below) which
      specializes this definition of the named type, so that the same API
      types may have different definitions for e.g. GL ES and GL. This is
      unlikely to be used in Vulkan, where a single API supports desktop and
      mobile devices, but the functionality is retained.
\item \attr{category} - optional. A string which indicates that this type
      contains a more complex structured definition. At present the only
      accepted categories are \code{basetype} \code{bitmask}, \code{define},
      \code{enum}, \code{funcpointer}, \code{group}, \code{handle},
      \code{include}, \code{struct}, and \code{union}, as described below.
\item \attr{comment} - optional. Arbitrary string (unused).
\item \attr{parent} only applicable if category is \code{handle}. Notes another type with
      the \code{handle} category that acts as a parent object for this type.
\item \attr{returnedonly} only applicable if category is \code{struct} or \code{union}.
      Notes that this struct/union is going to be filled in by the API, rather than an
      application filling it out and passing it to the API.
\end{itemize}

\subsection{Contents of \tag{type} tags}

The valid contents depend on the \attr{category} attribute.

\subsubsection{Enumerated types - \attr{category} "enum"}

If the \attr{category} tag has the value \code{enum}, the type is a C
enumeration. The body of the tag is ignored in this case. The value of the
\attr{name} attribute must be provided and must match the \attr{name}
attribute of a \tag{enums} tag (see section~\ref{tag:enums}). The enumerant
values defined within the \tag{enums} tag are used to generate a C
\code{enum} type declaration.

\subsubsection{Structure types - \attr{category} "struct" or "union"}

If the \attr{category} tag has the values \code{struct} or \code{union}, the
type is a C structure or union, respectively. In this case, the \attr{name}
attribute must be provided, and the contents of the \tag{type} tag are a
series of \tag{member} tags defining the members of the aggregate type, in
order, followed by an optional \tag{validity} tag including asciidoc
validation language for the structure contents.


\paragraph{Structure member (\tag{member}) tags}
\par\vskip 11pt

The \tag{member} tag defines the type and name of a structure or union
member.

\paragraph{Attributes of \tag{member} tags}

\begin{itemize}
\item \attr{len} - if the member is an array, len may be one or more of the following
           things, separated by commas (one for each array indirection): another
           member of that struct, ``null-terminated'' for a string, ``1'' to indicate it's
           just a pointer (used for nested pointers), or a latex equation (prefixed with
           ``latex:'')
\item \attr{externsync} - denotes that the member should be externally synchronized
           when accessed by Vulkan
\item \attr{optional} - whether this value can be omitted by providing \code{NULL} (for
          pointers), \code{VK_NULL_HANDLE} (for handles) or 0 (for bitmasks/values)
\item \attr{noautovalidity} - prevents automatic validity language being generated
          for the tagged item. Only suppresses item-specific validity - parenting issues
          etc. are still captured.
\end{itemize}

\paragraph{Contents of \tag{member} tags}
\par\vskip 11pt

The text elements of a \tag{member} tag, with all other tags removed,
is a legal C declaration of a struct or union member. In addition it may
contain two semantic tags:

\begin{itemize}
\item The \tag{type} tag is optional. It contains text which is a valid type
      name found in another \tag{type} tag, and indicates that this type
      must be previously defined for the definition of the command to
      succeed. Builtin C types should not be wrapper in \tag{type} tags.
\item The \tag{name} tag is required, and contains the struct/union member
      name being described.
\end{itemize}

\paragraph{Validation (\tag{validity}) tags}
\par\vskip 11pt

The \tag{validity} tag, if present defines valid use cases and values for
structure members.

\paragraph{Contents of \tag{validity} tags}
\par\vskip 11pt

Each \tag{validity} tag contains zero or more \tag{usage} tags. Each
\tag{usage} tag is intended to represent a specific validation requirement
for the structure and include arbitrary asciidoc text describing that
requirement.


\subsubsection{All other types}

If the \attr{category} attribute is one of \code{basetype}, \code{bitfield},
\code{define}, \code{funcpointer}, \code{group}, \code{handle} or
\code{include}, or is not specified, \tag{type} contains text which is legal
C code for a type declaration. It may also contain embedded tags:

\begin{itemize}
\item \tag{type} - nested type tags contain other type names which are
      required by the definition of this type.
\item \tag{apientry/} - insert a platform calling convention macro here
      during header generation, used mostly for function pointer types.
\item \tag{name} - contains the name of this type (if not defined in the
      tag attributes).
\end{itemize}

There is no restriction on which sorts of definitions may be made in a given
category, although the contents of tags with \attr{category} \code{enum},
\code{struct} or \code{union} are interpreted specially as described above.

However, when generating the header, types within each category are grouped
together, and categories are generated in the order given by the following
list. Therefore, types in a category should correspond to the intended
purpose given for that category. If this recommendation is not followed, it
is possible that the resulting header file will not compile due to
out-of-order type dependencies.

\begin{itemize}
  \item \code{include} ({\tt \#include} directives)
  \item \code{define} (macro {\tt \#define} directives)
  \item \code{basetype} (scalar typedefs, such as the definition of
        {\tt VkFlags})
  \item \code{handle} (invocations of macros defining scalar types such as
        {\tt VkInstance})
  \item \code{enum} (enumeration types and {\tt \#define} for constant values)
  \item \code{group} (currently unused)
  \item \code{bitmask} (enumeration types whose members are bitmasks)
  \item \code{funcpointer} (function pointer typedefs)
  \item \code{struct} and \code{union} together (struct and union types)
\end{itemize}


\subsection{Example of a \tag{types} tag}
\label{tag:types:example}

\begin{verbatim}
<types>
    <type name="stddef">#include &lt;stddef.h&gt;</type>
    <type requires="stddef">typedef ptrdiff_t <name>VKlongint</name>;</type>
    <type name="VkEnum" category="enum"/>
    <type category="struct" name="VkStruct">
        <member><type>VkEnum</type> <name>srcEnum</name></member>
        <member><type>VkEnum</type> <name>dstEnum</name></member>
    </type>
</types>

<enums name="VkEnum" type="enum">
    <enum value="0" name="VK_ENUM_ZERO"/>
    <enum value="42" name="VK_ENUM_FORTY_TWO"/>
</enums>
\end{verbatim}

The {\tt VkStruct} type is defined to require the types {\tt VkEnum} and
{\tt VKlongint} as well. If {\tt VkStruct} is in turn required by a command
or another type during header generation, it will result in the following
declarations:

\begin{verbatim}
#include <stddef.h>
typedef ptrdiff_t VKlongint.

typedef enum {
    VK_ENUM_ZERO = 0,
    VK_ENUM_FORTY_TWO = 42
} VkEnum;

typedef struct {
    VkEnum    dstEnum;
    VkLongint dstVal;
} VkStruct;
\end{verbatim}

Note that the angle brackets around \code{stddef.h} are represented as XML
entities in the registry. This could also be done using a CDATA block but
unless there are many characters requiring special representation in XML,
using entities is preferred.


\section{Enumerant Blocks (\tag{enums} tag)}
\label{tag:enums}

The \tag{enums} tags contain individual \tag{enum} tags describing each of
the token names used in the API. In some cases these correspond to a C
\code{enum}, and in some cases they are simply compile-time constants (e.g.
\code{\#define}).

\subsection{Attributes of \tag{enums} tags}

\begin{itemize}
\item \attr{name} - optional. String naming the C \code{enum} type whose
      members are defined by this enum group. If present, this attribute
      should match the \attr{name} attribute of a corresponding \tag{type}
      tag.
\item \attr{type} - optional. String describing the data type of the values
      of this group of enums. At present the only accepted categories are
      \code{enum} and \code{bitmask}, as described below.
\item \attr{start}, \attr{end} - optional. Integers defining the start and
      end of a reserved range of enumerants for a particular vendor or
      purpose. \attr{start} must be $\leq$ \attr{end}. These fields define
      formal enumerant allocations, and are made by the Khronos Registrar on
      request from implementers following the enum allocation policy.
\item \attr{vendor} - optional. String describing the vendor or purpose to
      whom a reserved range of enumerants is allocated.
\item \attr{comment} - optional. Arbitrary string (unused).
\end{itemize}

\subsection{Contents of \tag{enums} tags}

Each \tag{enums} block contains zero or more \tag{enum} and \tag{unused}
tags, in arbitrary order (although they are typically ordered by sorting
on enumerant values, to improve human readability).

\subsection{Example of \tag{enums} tags}

An example showing a tag with attribute \attr{type}="\code{enum}" is given
above in section~\ref{tag:types:example}. The following example is for
non-enumerated tokens.

\begin{verbatim}
<enums>
    <enum value="256" name="VK_MAX_EXTENSION_NAME"/>
    <enum value="MAX_FLOAT"  name="VK_LOD_CLAMP_NONE"/>
</enums>
\end{verbatim}

When processed into a C header, and assuming all these tokens were
required, this results in

\begin{verbatim}
#define VK_MAX_EXTENSION_NAME   256
#define VK_LOD_CLAMP_NONE       MAX_FLOAT
\end{verbatim}


\section{Enumerants (\tag{enum} tag)}
\label{tag:enum}

Each \tag{enum} tag defines a single Vulkan (or other API) token.

\subsection{Attributes of \tag{enum} tags}

\begin{itemize}
\item \attr{value} or \attr{bitpos} - exactly one of these is allowed and
      required. \attr{value} is an enumerant value in the form of a legal C
      constant (usually a literal decimal or hexadecimal integer, though
      arbitrary strings are allowed). \attr{bitpos} is a literal integer bit
      position in a bitmask.
\item \attr{name} - required. Enumerant name, a legal C preprocessor token
      name.
\item \attr{api} - optional. An API name which specializes this definition of
      the named enum, so that different APIs may have different values for
      the same token. May be used to address a subtle incompatibilities.
\item \attr{type} - optional. Used only when \attr{value} is specified. C
      suffix for the value to force it to a specific type. Currently only
      \code{u} and \code{ull} are used, for \code{unsigned} 32- and 64-bit
      integer values, respectively. Separated from \attr{value} since this
      eases parsing and sorting of values, and rarely used.
\item \attr{alias} - optional. Name of another enumerant this is an alias
      of, used where token names have been changed as a result of profile
      changes or for consistency purposes. An enumerant alias is simply a
      different \attr{name} for the exact same \attr{value} or
      \attr{bitpos}.
\end{itemize}

\subsection{Contents of \tag{enum} tags}

\tag{enum} tags have no allowed contents. All information is contained
in the attributes.


\section{Unused Enumerants (\tag{unused} tag)}
\label{tag:unused}

Each \tag{unused} tag defines a range of enumerants which is allocated, but
not yet assigned to specific enums. This just tracks the unused values for
the Registrar's use, and is not used for header generation.

\subsection{Attributes of \tag{unused} tags}

\begin{itemize}
\item \attr{start} - required, \attr{end} - optional. Integers defining the
      start and end of an unused range of enumerants. \attr{start} must be
      $\leq$ \attr{end}. If \attr{end} is not present, then \attr{start}
      defines a single unused enumerant. This range should not exceed the
      range reserved by the surrounding \tag{enums} tag.
\item \attr{vendor} - optional. String describing the vendor or purposes to
      whom a reserved range of enumerants is allocated. Usually identical to
      the \attr{vendor} attribute of the surrounding \attr{enums} block.
\item \attr{comment} - optional. Arbitrary string (unused).
\end{itemize}

\subsection{Contents of \tag{unused} tags}

None.


\section{Command Blocks (\tag{commands} tag)}
\label{tag:commands}

The \tag{commands} tag contains definitions of each of the functions
(commands) used in the API.

\subsection{Attributes of \tag{commands} tags}

None.

\subsection{Contents of \tag{commands} tags}

Each \tag{commands} block contains zero or more \tag{command} tags, in
arbitrary order (although they are typically ordered by sorting on the
command name, to improve human readability).


\section{Commands (\tag{command} tag)}
\label{tag:command}

The \tag{command} tag contains a structured definition of a single API
command (function).

\subsection{Attributes of \tag{command} tags}

\begin{itemize}
\item \attr{queues} - optional. A string identifying the command queues
      this command can be placed on. The format of the string is one or
      more of the terms \code{"compute"}, \code{"dma"}, and
      \code{"graphics"}, with multiple terms separated by commas
      (\code{","}).
\item \attr{renderpass} - optional. A string identifying whether the
      command can be issued only inside a render pass (\code{"inside"}),
      only outside a render pass (\code{"outside"}), or both
      (\code{"both"}).
\item \attr{cmdbufferlevel} - optional. A string identifying the command buffer
      levels that this command can be called by. The format of the string is
      one or more of the terms \code{"primary"} and \code{"secondary"}, with
      multiple terms separated by commas (\code{","}).
\item \attr{comment} - optional. Arbitrary string (unused).
\end{itemize}

\subsection{Contents of \tag{command} tags}

\begin{itemize}
\item \tag{proto} is required and must be the first element. It is a tag
      defining the C function prototype of a command as described below, up
      to the function name and return type but not including function
      parameters.
\item \tag{param} elements for each command parameter follow, defining
      its name and type, as described below. If a command takes no
      arguments, it has no \tag{param} tags.
\item An optional \tag{validity} tag including asciidoc validation language
      for the command parameters.
\end{itemize}

Following these elements, the remaining elements in a \tag{command}
tag are optional and may be in any order:

\begin{itemize}
\item \tag{alias} - optional. Has no attributes and contains a string which
      is the name of another command this command is an alias of, used when
      promoting a function from vendor to Khronos extension or Khronos
      extension to core API status. A command alias describes the case where
      there are two function names which resolve to the {\bf same} entry
      point in the underlying layer stack.
\item \tag{description} - optional. Unused text.
\item \tag{implicitexternsyncparams} - optional. Contains a list of
      \tag{param} tags, each containing Asciidoc source text describing an
      object which is not a parameter of the command, but is related to one,
      and which also requires external synchronization as described in
      section~\ref{tag:command:param:attr}. The text is intended to be
      incorporated into the API specification.
\end{itemize}


\subsection{Command prototype (\tag{proto} tags)}
\label{tag:command:proto}

The \tag{proto} tag defines the return type and name of a command.

\subsubsection{Attributes of \tag{proto} tags}

None.

% \begin{itemize}
% \item \attr{group} - group name, an arbitrary string.
% \end{itemize}
%
% If the group name is defined, it may be interpreted as described in
% section~\ref{tag:group:meaning}.

\subsubsection{Contents of \tag{proto} tags}

The text elements of a \tag{proto} tag, with all other tags removed, is
legal C code describing the return type and name of a command. In addition
to text, it may contain two semantic tags:

\begin{itemize}
\item The \tag{type} tag is optional, and contains text which is a valid
      type name found in a \tag{type} tag. It indicates that this type must
      be previously defined for the definition of the command to succeed.
      Builtin C types, and any derived types which are expected to be found
      in other header files, should not be wrapped in \tag{type} tags.
\item The \tag{name} tag is required, and contains the command name
      being described.
\end{itemize}

\subsection{Command parameter (\tag{param} tags)}
\label{tag:command:param}

The \tag{param} tag defines the type and name of a parameter. Its contents
are very similar to the \tag{member} tag used to define struct and union
members.

\subsubsection{Attributes of \tag{param} tags}
\label{tag:command:param:attr}

\begin{itemize}
\item \attr{len} - if the param is an array, len may be one or more of the following
           things, separated by commas (one for each array indirection): another
           param of that command, ``null-terminated'' for a string, ``1'' to indicate it's
           just a pointer (used for nested pointers), or a latex equation (prefixed with
           ``latex:'')
\item \attr{optional} - whether this value can be omitted by providing \code{NULL} (for
          pointers), \code{VK_NULL_HANDLE} (for handles) or 0 (for bitmasks/values)
\item \attr{noautovalidity} -  prevents automatic validity language being generated
          for the tagged item. Only suppresses item-specific validity - parenting issues
          etc. are still captured.
\item \attr{externsync} - optional. A boolean string, which must have the
      value \code{"true"} if present, indicating that this parameter (e.g.
      the object a handle refers to, or the contents of an array a pointer
      refers to) is modified by the command, and is not protected against
      modification in multiple app threads. Parameters which do not have
      this attribute are assumed to not require external synchronization.
\end{itemize}

\subsubsection{Contents of \tag{param} tags}

The text elements of a \tag{param} tag, with all other tags removed, is
legal C code describing the type and name of a function parameter. In
addition it may contain two semantic tags:

\begin{itemize}
\item The \tag{type} tag is optional, and contains text which is a
      valid type name found in \tag{type} tag, and indicates that this
      type must be previously defined for the definition of the command
      to succeed. Builtin C types, and any derived types which are
      expected to be found in other header files, should not be wrapped
      in \tag{type} tags.
\item The \tag{name} tag is required, and contains the parameter name being
      described.
\end{itemize}

\subsection{Example of a \tag{commands} tag}

\begin{verbatim}
<commands>
    <command>
        <proto><type>VkResult</type> <name>vkCreateInstance</name></proto>
        <param>const <type>VkInstanceCreateInfo</type>* <name>pCreateInfo</name></param>
        <param><type>VkInstance</type>* <name>pInstance</name></param>
    </command>
</commands>
\end{verbatim}

When processed into a C header, this results in

\begin{verbatim}
VkResult vkCreateInstance(
    const VkInstanceCreateInfo* pCreateInfo,
    VkInstance* pInstance);
\end{verbatim}

\subsection{Parameter validation (\tag{validity}) tags}
\label{tag:command:validity}

The \tag{validity} tag, if present defines valid use cases and values for
command parameters

\paragraph{Contents of \tag{validity} tags}
\par\vskip 11pt

Each \tag{validity} tag contains zero or more \tag{usage} tags. Each
\tag{usage} tag is intended to represent a specific validation requirement
for the command, and contains arbitrary asciidoc text describing that
requirement.


\section{API Features / Versions (\tag{feature} tag)}
\label{tag:feature}

API features are described in individual \tag{feature} tags. A feature is
the set of interfaces (enumerants and commands) defined by a particular API
and version, such as Vulkan 1.0, and includes all profiles of that API and
version.

\subsection{Attributes of \tag{feature} tags}

\begin{itemize}
\item \attr{api} - required. API name this feature is for (see
      section~\ref{schema:apiname}), such as \code{vk}.
\item \attr{name} - required. Version name, used as the C preprocessor token
      under which the version's interfaces are protected against multiple
      inclusion. Example: \code{VK\_VERSION\_1\_0}.
\item \attr{number} - required. Feature version number, usually a string
      interpreted as $majorNumber.minorNumber$. Example: \code{4.2}.
\item \attr{protect} - optional. An additional preprocessor token used to
      protect a feature definition. Usually another feature or extension
      \attr{name}. Rarely used, for odd circumstances where the definition
      of a feature or extension requires another to be defined first.
\item \attr{comment} - optional. Arbitrary string (unused).
\end{itemize}

\subsection{Contents of \tag{feature} tags}

Zero or more \tag{require} and \tag{remove} tags (see
section~\ref{tag:required}), in arbitrary order. Each tag describes a
set of interfaces that is respectively required for, or removed from,
this feature, as described below.

\subsection{Example of a \tag{feature} tag}

\begin{verbatim}
<feature api="vulkan" name="VK_VERSION_1_0" number="1.0">
    <require comment="Header boilerplate">
        <type name="vk_platform"/>
    </require>
    <require comment="API constants">
        <enum name="VK_MAX_PHYSICAL_DEVICE_NAME"/>
        <enum name="VK_LOD_CLAMP_NONE"/>
    </require>
    <require comment="Device initialization">
        <command name="vkCreateInstance"/>
    </require>
</feature>
\end{verbatim}

When processed into a C header for Vulkan, this results in:

\begin{verbatim}
#ifndef VK_VERSION_1_0
#define VK_VERSION_1_0 1
#define VK_MAX_EXTENSION_NAME   256
#define VK_LOD_CLAMP_NONE       MAX_FLOAT
typedef VkResult (VKAPI_PTR *PFN_vkCreateInstance)(const VkInstanceCreateInfo* pCreateInfo, VkInstance* pInstance);
#ifdef VK_PROTOTYPES
VKAPI_ATTR VkResult VKAPI_CALL vkCreateInstance(
    const VkInstanceCreateInfo*                 pCreateInfo,
    VkInstance*                                 pInstance);
#endif
#endif /* VK_VERSION_1_0 */
\end{verbatim}


\section{Extension Blocks (\tag{extensions} tag)}
\label{tag:extensions}

The \tag{extensions} tag contains definitions of each of the extenions
which are defined for the API.

\subsection{Attributes of \tag{extensions} tags}

None.

\subsection{Contents of \tag{extensions} tags}

Each \tag{extensions} block contains zero or more \tag{extension} tags,
each describing an API extension, in arbitrary order (although they are
typically ordered by sorting on the extension name, to improve human
readability).


\section{API Extensions (\tag{extension} tag)}
\label{tag:extension}

API extensions are described in individual \tag{extension} tags. An
extension is the set of interfaces defined by a particular API extension
specification, such as \code{ARB\_multitexture}. \tag{extension} is
similar to \tag{feature}, but instead of having \attr{version} and
\attr{profile} attributes, instead has a \attr{supported} attribute,
which describes the set of API names which the extension can potentially
be implemented against.

\subsection{Attributes of \tag{extension} tags}

\begin{itemize}
\item \attr{name} - required. Extension name, following the conventions
      in the Vulkan Specification.
      Example: \code{name="VK\_VERSION\_1\_0"}.
\item \attr{number} - required. A decimal number which is the registered,
      unique extension number for \attr{name}.
\item \attr{supported} - required. A regular expression, with an implicit
      \code{\^{}} and \code{\$} bracketing it, which should match the
      \attr{api} tag of a set of \tag{feature} tags.
\item \attr{protect} - optional. An additional preprocessor token used to
      protect an extension definition. Usually another feature or extension
      \attr{name}. Rarely used, for odd circumstances where the definition
      of an extension requires another extension or a header file to be
      defined first.
\item \attr{author} - optional. The author name, such as a full company
      name. If not present, this can be taken from the corresponding
      \tag{tag} attribute. However, \code{EXT} and other multi-vendor
      extensions may not have a well-defined author or contact in the tag.
\item \attr{contact} - optional. The contact who registered or is currently
      responsible for extensions and layers using the tag, including
      sufficient contact information to reach the contact such as individual
      name together with email address, Github username, or other contact
      information. If not present, this can be taken from the corresponding
      \tag{tag} attribute just like \attr{author}.
\item \attr{comment} - optional. Arbitrary string (unused).
\end{itemize}

\subsection{Contents of \tag{extension} tags}

Zero or more \tag{require} and \tag{remove} tags (see
section~\ref{tag:required}), in arbitrary order. Each tag describes a
set of interfaces that is respectively required for, or removed from,
this extension, as described below.

\subsection{Example of an \tag{extensions} tag}

\begin{verbatim}
<extension name="VK_KHR_display_swapchain" number="4" supported="vulkan">
    <require>
        <enum value="9" name="VK_KHR_DISPLAY_SWAPCHAIN_SPEC_VERSION"/>
        <enum value="4" name="VK_KHR_DISPLAY_SWAPCHAIN_EXTENSION_NUMBER"/>
        <enum value="&quot;VK_KHR_display_swapchain&quot;"
              name="VK_KHR_DISPLAY_SWAPCHAIN_EXTENSION_NAME"/>
        <type name="VkDisplayPresentInfoKHR"/>
        <command name="vkCreateSharedSwapchainsKHR"/>
    </require>
</extension>
\end{verbatim}

The \attr{supported} attribute says that the extension is defined for the
default profile (\code{vulkan}). When processed into a C header for the
\code{vulkan} profile, this results in header contents something like
(assuming corresponding definitions of the specified \tag{type} and
\tag{command} elsewhere in the XML):

\begin{verbatim}
#define VK_KHR_display_swapchain 1
#define VK_KHR_DISPLAY_SWAPCHAIN_SPEC_VERSION 9
#define VK_KHR_DISPLAY_SWAPCHAIN_EXTENSION_NUMBER 4
#define VK_KHR_DISPLAY_SWAPCHAIN_EXTENSION_NAME "VK_KHR_display_swapchain"

typedef struct VkDisplayPresentInfoKHR {
    VkStructureType                             sType;
    const void*                                 pNext;
    VkRect2D                                    srcRect;
    VkRect2D                                    dstRect;
    VkBool32                                    persistent;
} VkDisplayPresentInfoKHR;

typedef VkResult (VKAPI_PTR *PFN_vkCreateSharedSwapchainsKHR)(
    VkDevice device, uint32_t swapchainCount,
    const VkSwapchainCreateInfoKHR* pCreateInfos,
    const VkAllocationCallbacks* pAllocator,
    VkSwapchainKHR* pSwapchains);

#ifdef VK_PROTOTYPES
VKAPI_ATTR VkResult VKAPI_CALL vkCreateSharedSwapchainsKHR(
    VkDevice                                    device,
    uint32_t                                    swapchainCount,
    const VkSwapchainCreateInfoKHR*             pCreateInfos,
    const VkAllocationCallbacks*                pAllocator,
    VkSwapchainKHR*                             pSwapchains);
#endif
\end{verbatim}


\section{Required and Removed Interfaces (\tag{require} and \tag{remove} tags)}
\label{tag:required}

A \tag{require} block defines a set of interfaces (types, enumerants and
commands) {\em required} by a \tag{feature} or \tag{extension}. A
\tag{remove} block defines a set of interfaces {\em removed} by a
\tag{feature} (this is primarily useful for future profiles of an API which
may choose to deprecated and/or remove some interfaces - extensions should
never remove interfaces, although this usage is allowed by the schema).
Except for the tag name and behavior, the contents of \tag{require} and
\tag{remove} tags are identical.

\subsection{Attributes of \tag{require} and \tag{remove} tags}

\begin{itemize}
\item \attr{profile} - optional. String name of an API profile. Interfaces
      in the tag are only required (or removed) if the specified profile is
      being generated. If not specified, interfaces are required (or
      removed) for all API profiles.
\item \attr{comment} - optional. Arbitrary string (unused).
\item \attr{api} - optional. An API name (see section~\ref{schema:apiname}).
      Interfaces in the tag are only required (or removed) if the specified
      API is being generated. If not specified, interfaces are required (or
      removed) for all APIs.

      {\bf The \attr{api} attribute is only supported inside
      \tag{extension} tags,} since \tag{feature} tags already define a
      specific API.
\end{itemize}

\subsection{Contents of \tag{require} and \tag{remove} tags}

Zero or more of the following tags, in any order:

\begin{itemize}
\item \tag{command} specifies an required (or removed) command defined
      in a \tag{commands} block. The tag has no content, but contains
      attributes:

      \begin{itemize}
      \item \attr{name} - required. Name of the command.
      \item \attr{comment} - optional. Arbitrary string (unused).
      \end{itemize}
\item \tag{enum} specifies an required (or removed) enumerant defined in a
      \tag{enums} block. All forms of this tag support the following
      attributes:

      \begin{itemize}
      \item \attr{name} - required. Name of the enumerant.
      \item \attr{comment} - optional. Arbitrary string (unused).
      \end{itemize}

      There are two forms of \tag{enum} tags:

      {\bf Reference enums} simply pull in the definition of an enumerant
      given in a separate \tag{enums} block. Reference enums are the most
      common usage, and no attributes other than \attr{name} and
      \attr{comment} are supported for them.

      {\bf Extension enums} define the value of an enumerant inline in an
      \tag{extensions} block. There are several variants, depending on which
      additional tags are defined:

      \begin{itemize}
      \item Attributes \attr{value} and (optionally) \attr{type} define a
            constant value in the same fashion as an \tag{enum} tag in an
            \tag{enums} block (see section~\ref{tag:enum}).
      \item Attribute \attr{bitpos} defines a constant bitmask value in the
            same fashion as an \tag{enum} tag in an \tag{enums} block (see
            section~\ref{tag:enum}).
      \item Attribute \attr{extends} allows an extension enumerant to be
            added to a separately defined enumerated type whose name is
            specified by the contents of the \attr{extends} attribute (e.g.
            a \tag{type} tag with \attr{category} "enum", pulling in an
            \tag{enums} block).

            There are two ways to extend an enumerated type:

            \begin{itemize}
            \item If \attr{bitpos} is also specified, the tag defines a
                  bitmask value, but adds its definition to the enumeranted
                  type specified by \attr{extends} instead of as a
                  compile-time constant.
            \item If \attr{offset} is also specified, the tag adds a new
                  enumerant value to the enumerated type specified by
                  \attr{extends}. The actual value defined depends on the
                  extension number (the \attr{number} attribute of the
                  \tag{extensions} tag) and the offset, as defined in the
                  ``Layers and Extensions'' appendix of the Vulkan
                  Specification. The \attr{dir} attribute may also be
                  specified as {\tt dir="-"} if the calculated value should
                  be negative instead of positive. Negative enumerant values
                  are normally used only for Vulkan error codes.
            \end{itemize}
      \end{itemize}

      Examples of various types of extension enumerants are given below in
      section~\ref{tag:required:examples}.
\item \tag{type} specifies a required (or removed) type defined in a
      \tag{types} block. Most types are picked up implicitly by using the
      \tag{type} tags of commands, but in a few cases, additional types need
      to be specified explicitly. It is unlikely that a type would ever be
      removed, although this usage is allowed by the schema. The tag has no
      content, but contains elements:

      \begin{itemize}
      \item \attr{name} - required. Name of the type.
      \item \attr{comment} - optional. Arbitrary string (unused).
      \end{itemize}
\end{itemize}


\subsection{Examples of Extension Enumerants}
\label{tag:required:examples}

Examples of each of the supported extension enumerant \tag{enum} tags are
given below. Note that extension enumerants are supported only inside
\tag{extension} blocks - not in \tag{feature} blocks\footnote{
    However, we will have to define additional XML tags and/or syntax for
    future core versions of Vulkan, to properly tag and group core
    enumerants for Vulkan 1.0, 1.1, etc.}.

\begin{verbatim}
<extensions>
    <extension name="VK_KHR_test_extension" number="1" supported="vulkan">
        <require>
            <enum value="42" name="VK_KHR_theanswer"/>
            <enum bitpos="29" name="VK_KHR_bitmask"/>
            <enum offset="0" dir="-" extends="VkResult"
                  name="VK_ERROR_SURFACE_LOST_KHR"/>
            <enum offset="1" extends="VkResult"
                  name="VK_SUBOPTIMAL_KHR"/>
            <enum bitpos="31" extends="VkResult"
                  name="VK_KHR_EXTENSION_BITFIELD"/>
        </require>
    </extension>
</extensions>
\end{verbatim}

The corresponding header file will include definitions like this:

\begin{verbatim}
typedef enum VkResult {
    <previously defined VkResult enumerant values},
    VK_ERROR_SURFACE_LOST_KHR = -1000000000,
    VK_SUBOPTIMAL_KHR = 1000000001,
    VK_KHR_EXTENSION_BITFIELD = 0x80000000,
};

#define VK_KHR_test_extension 1
#define VK_KHR_theanswer 42
#define VK_KHR_bitmask 0x20000000
\end{verbatim}

\clearpage
\appendix

\section{Examples / FAQ / How Do I?}
\label{examples}

For people new to the Registry, it won't be immediately obvious how to make
changes. This section includes some tips and examples that will help you
make changes to the Vulkan headers by changing the Registry XML description.

First, follow the steps in section~\ref{starting} to get the Vulkan Gitlab
repository containing the registry and assemble the tools necessary to work
with the XML registry. Once you're able to regenerate \code{vulkan.h} from
\code{vk.xml}, you can start making changes.


\subsection{General Strategy}

If you are {\bf adding} to the API, perform the following steps to {\bf
create} the description of that API element:

\begin{itemize}
\item For each type, enum group, compile-time constant, and command being
      added, create appropriate new \tag{type}, \tag{enums}, \tag{enum}, or
      \tag{command} tags defining the interface in question.
\item Make sure that all added types and commands appropriately tag their
      dependencies on other types by adding nested \tag{type} tags.
\item Make sure that each new tag defines the name of the corresponding
      type, enum group, constant, or command, and that structure/union types
      and commands tag the types and names of all their members and
      parameters. This is essential for the automatic dependency process to
      work.
\end{itemize}

If you are {\bf modifying} existing APIs, just make appropriate changes
in the existing tags.

Once the definition is added, proceed to the next section to create
dependencies on the changed feature.


\subsection{API Feature Dependencies}

When you add new API elements, they will not result in corresponding changes
in the generated header unless they are {\bf required} by the interface
being generated. This makes it possible to include different API versions
and extensions in a single registry and pull them out as needed. So you must
introduce a dependency on new features in the corresponding \tag{feature}
tag.

Initially, the only API feature is Vulkan 1.0, so there is only one
\tag{feature} tag in \code{vk.xml}. You can find it by searching
for the following block of \code{vk.xml}:

\begin{verbatim}
<!-- SECTION: Vulkan API interface definitions -->
<feature api="vulkan" name="VK_VERSION_1_0" number="1.0">
\end{verbatim}

Inside the \tag{feature} tag are nested multiple \tag{require} tags. These
are just being used as a logical grouping mechanism for related parts of
Vulkan 1.0 at present, though they may have more meaningful roles in the
future if different API profiles are defined.


\subsubsection{API Feature Walkthrough}

This section walks through the first few required API features in the
\code{vk.xml} \tag{feature} tag, showing how each requirement pulls in type,
token, and command definitions and turns those into definitions in the C
header file \code{vulkan.h} .

Consider the first few lines of the \tag{feature}:

\begin{verbatim}
<require comment="Header boilerplate">
    <type name="vk_platform"/>
</require>
<require comment="API constants">
    <enum name="VK_MAX_PHYSICAL_DEVICE_NAME"/>
    <enum name="VK_MAX_EXTENSION_NAME"/>
    ...
</require>
<require comment="Device initialization">
    <command name="vkCreateInstance"/>
    ...
\end{verbatim}

The first \tag{require} block says to require a type named
\code{vk_platform}. If you look at the beginning of the \tag{types} section,
there's a corresponding definition:


\begin{verbatim}
<type name="vk_platform">#include "vk_platform.h"
#define VK_MAKE_VERSION(major, minor, patch) \
    ((major &lt;&lt; 22) | (minor &lt;&lt; 12) | patch)
    ...
\end{verbatim}

\noindent section which is invoked by the requirement and emits a bunch of
boilerplate C code. The explicit dependency isn't strictly required since
\code{vk_platform} will be required by many other types, but placing it
first causes this to appear first in the output file.

Note that \code{vk_platform} does not correspond to an actual C type, but
instead to a collection of freeform preprocessor includes and macros and
comments. Most other \tag{type} tags do define a specific type and are much
simpler, but this approach can be used to inject arbitrary C into
\code{vulkan.h} {\bf when there's no other way}. In general inserting
arbitrary C is strongly discouraged outside of specific special cases like
this.

The next \tag{require} block pulls in some compile-time constants. These
correspond to the definitions found in the first \tag{enums} section of
\code{vk.xml}:

\begin{verbatim}
<!-- SECTION: Vulkan enumerant (token) definitions. -->

<enums comment="Misc. hardcoded constants - not an enumerated type">
        <!-- This is part of the header boilerplate -->
    <enum value="256"        name="VK_MAX_PHYSICAL_DEVICE_NAME"/>
    <enum value="256"        name="VK_MAX_EXTENSION_NAME"/>
    ...
\end{verbatim}

The third \tag{require} block starts pulling in some Vulkan commands. The
first command corresponds to the following definition found in the
\tag{commands} section of \code{vk.xml}:

\begin{verbatim}
<commands>
    <command>
        <proto><type>VkResult</type> <name>vkCreateInstance</name></proto>
        <param>const <type>VkInstanceCreateInfo</type>* <name>pCreateInfo</name></param>
        <param><type>VkInstance</type>* <name>pInstance</name></param>
    </command>
    ...
\end{verbatim}

In turn, the \tag{command} tag requires the \tag{type}s \code{VkResult},
\code{VkInstanceCreateInfo}, and \code{VkInstance} as part of its
definition. The definitions of these types are determined as follows:

For \code{VkResult}, the corresponding required \tag{type} is:

\begin{verbatim}
<type name="VkResult" category="enum"/>
\end{verbatim}

\noindent Since this is an enumeration type, it simply links to an
\tag{enums} tag with the same name:

\begin{verbatim}
<enums name="VkResult" type="enum" expand="VK_RESULT" comment="Error and return codes">
    <!-- Return codes for successful operation execution (positive values) -->
    <enum value="0"     name="VK_SUCCESS"/>
    <enum value="1"     name="VK_UNSUPPORTED"/>
    <enum value="2"     name="VK_NOT_READY"/>
    ...
\end{verbatim}

For \tag{VkInstanceCreateInfo}, the required \tag{type} is:

\begin{verbatim}
<type category="struct" name="VkInstanceCreateInfo">
    <member><type>VkStructureType</type>        <name>sType</name></member>                          <!-- Should be VK_STRUCTURE_TYPE_INSTANCE_CREATE_INFO -->
    <member>const void*                         <name>pNext</name></member>                          <!-- Pointer to next structure -->
    <member>const <type>VkApplicationInfo</type>* <name>pAppInfo</name></member>
    <member>const <type>VkAllocCallbacks</type>* <name>pAllocCb</name></member>
    <member><type>uint32_t</type>               <name>extensionCount</name></member>
    <member>const <type>char</type>*const*      <name>ppEnabledExtensionNames</name></member>        <!-- layer or extension name to be enabled -->
</type>
\end{verbatim}

\noindent This is a structure type, defining a C \code{struct} with all the
members defined in each \tag{member} tag in order. In addition, it requires
some other types, whose definitions are located by name in exactly the same
fashion.

For the final direct dependency of the command, \code{VkInstance}, the
required \tag{type} is:

\begin{verbatim}
<!-- Types which can be void pointers or class pointers, selected at compile time -->
<type>VK_DEFINE_BASE_HANDLE(<name>VkObject</name>)</type>
<type>VK_DEFINE_DISP_SUBCLASS_HANDLE(<name>VkInstance</name>, <type>VkObject</type>)</type>
\end{verbatim}

\noindent In this case, the type \code{VkInstance} is defined by a special
compile-time macro which defines it as a derived class of \code{VkObject}
(for C++) or a less typesafe definition for (for C). This macro isn't part
of the type dependency analysis, just the boilerplate used in the header.

If these are the only \tag{feature} dependencies in \code{vk.xml}, the
resulting \code{vulkan.h} header will look like this:

\begin{verbatim}

#ifndef __vulkan_h_
#define __vulkan_h_ 1

#ifdef __cplusplus
extern "C" {
#endif

/*
** Copyright (c) 2015-2016 The Khronos Group Inc.
    ...
*/

/*
** This header is generated from the Khronos Vulkan XML API Registry.
**
** Generated on date 20160208
*/


#define VK_VERSION_1_0 1
#include "vk_platform.h"
#define VK_MAKE_VERSION(major, minor, patch) \
    ((major << 22) | (minor << 12) | patch)

// Vulkan API version supported by this file
#define VK_API_VERSION VK_MAKE_VERSION(0, 104, 0)

#if defined (__cplusplus) && (VK_UINTPTRLEAST64_MAX == UINTPTR_MAX)
    #define VK_TYPE_SAFE_COMPATIBLE_HANDLES 1
#endif

#if defined(VK_TYPE_SAFE_COMPATIBLE_HANDLES) && !defined(VK_DISABLE_TYPE_SAFE_HANDLES)
    #define VK_DEFINE_PTR_HANDLE(_obj) struct _obj##_T { char _dummy; }; typedef _obj##_T* _obj;
    #define VK_DEFINE_PTR_SUBCLASS_HANDLE(_obj, _base) struct _obj##_T : public _base##_T {}; typedef _obj##_T* _obj;

    #define VK_DEFINE_BASE_HANDLE(_obj) VK_DEFINE_PTR_HANDLE(_obj)
    #define VK_DEFINE_DISP_SUBCLASS_HANDLE(_obj, _base) VK_DEFINE_PTR_SUBCLASS_HANDLE(_obj, _base)
    #define VK_DEFINE_NONDISP_SUBCLASS_HANDLE(_obj, _base) VK_DEFINE_PTR_SUBCLASS_HANDLE(_obj, _base)
#else
    #define VK_DEFINE_BASE_HANDLE(_obj) typedef VkUintPtrLeast64 _obj;
    #define VK_DEFINE_DISP_SUBCLASS_HANDLE(_obj, _base) typedef uintptr_t _obj;
    #define VK_DEFINE_NONDISP_SUBCLASS_HANDLE(_obj, _base) typedef VkUintPtrLeast64 _obj;
#endif

typedef enum {
    VK_SUCCESS = 0,
    VK_UNSUPPORTED = 1,
    VK_NOT_READY = 2,
    ...
} VkResult;
typedef enum {
    VK_STRUCTURE_TYPE_APPLICATION_INFO = 0,
    ...
} VKStructureType;
typedef struct {
    VkStructureType                             sType;
    const void*                                 pNext;
    const char*                                 pAppName;
    uint32_t                                    appVersion;
    const char*                                 pEngineName;
    uint32_t                                    engineVersion;
    uint32_t                                    apiVersion;
} VkApplicationInfo;
typedef enum {
    VK_SYSTEM_ALLOC_TYPE_API_OBJECT = 0,
    ...
} VkSystemAllocType;
typedef void* (VKAPI_PTR *PFN_vkAllocFunction)(
    void*                           pUserData,
    size_t                          size,
    size_t                          alignment,
    VkSystemAllocType               allocType);
typedef void (VKAPI_PTR *PFN_vkFreeFunction)(
    void*                           pUserData,
    void*                           pMem);
typedef struct {
    void*                                       pUserData;
    PFN_vkAllocFunction                         pfnAlloc;
    PFN_vkFreeFunction                          pfnFree;
} VkAllocCallbacks;
typedef struct {
    VkStructureType                             sType;
    const void*                                 pNext;
    const VkApplicationInfo*                    pAppInfo;
    const VkAllocCallbacks*                     pAllocCb;
    uint32_t                                    extensionCount;
    const char*const*                           ppEnabledExtensionNames;
} VkInstanceCreateInfo;
VK_DEFINE_BASE_HANDLE(VkObject)
VK_DEFINE_DISP_SUBCLASS_HANDLE(VkInstance, VkObject)
#define VK_MAX_PHYSICAL_DEVICE_NAME       256
#define VK_MAX_EXTENSION_NAME             256
typedef VkResult (VKAPI_PTR *PFN_vkCreateInstance)(const VkInstanceCreateInfo* pCreateInfo, VkInstance* pInstance);
#ifdef VK_PROTOTYPES
VKAPI_ATTR VkResult VKAPI_CALL vkCreateInstance(
    const VkInstanceCreateInfo*                 pCreateInfo,
    VkInstance*                                 pInstance);
#endif

#ifdef __cplusplus
}
#endif

#endif
\end{verbatim}

Note that several additional types are pulled in by the type dependency
analysis, but only those types, commands, and tokens required by the
specified features are generated.


\subsection{How To Add A Compile-Time Constant}

Go to the \tag{feature} tag and search for the nested block labelled

\begin{verbatim}
<require comment="API constants">
\end{verbatim}

In this block, add an (appropriately indented) tag like

\begin{verbatim}
    <enum name="VK_THE_ANSWER"/>
\end{verbatim}

Then go to the \tag{enums} block labelled

\begin{verbatim}
<enums comment="Misc. hardcoded constants - not an enumerated type">
\end{verbatim}

In this block, add a tag whose \attr{name} attribute matches the \attr{name}
you defined above and whose \attr{value} attribute is the value to give the
constant:

\begin{verbatim}
    <enum value="42"    name="VK_THE_ANSWER"/>
\end{verbatim}


\subsection{How To Add A Struct or Union Type}

For this example, assume we want to define a type
corresponding to a C \code{struct} defined as follows:

\begin{verbatim}
typedef struct {
    VkStructureType          sType;
    const void*              pNext;
    const VkApplicationInfo* pAppInfo;
    const VkAllocCallbacks*  pAllocCb;
    uint32_t                 extensionCount;
    const char*const*        ppEnabledExtensionNames;
} VkInstanceCreateInfo;
\end{verbatim}

If \code{VkInstanceCreateInfo} is the type of a parameter of a command in
the API, make sure that command's definition (see below for how to add a
command) puts \code{VkInstanceCreateInfo} in nested \tag{type} tags where
it's used.

Otherwise, if the struct type is not used directly by a command in the API,
nor required by a chain of type dependencies for other commands, an explicit
\tag{type} dependency should be added to the \tag{feature} tag. Go to the
\tag{types} tag and search for the nested block labelled

\begin{verbatim}
<require comment="API types not used by commands">
    ...
\end{verbatim}

In this block, add a tag whose \attr{name} attribute matches the \attr{name}
of the struct type being defined:

\begin{verbatim}
<require comment="API types not used by commands">
    <type name="VkInstanceCreateInfo"/>
    ...
\end{verbatim}

Then go to the \tag{types} tag and add a new \tag{type} tag defining
the struct names and members, somewhere below
the corresponding comment, like this:

\begin{verbatim}
<types>
    ...
    <!-- Struct types -->
        <type category="struct" name="VkInstanceCreateInfo">
            <member><type>VkStructureType</type>
                    <name>sType</name></member>
            <member>const void*
                    <name>pNext</name></member>
            <member>const <type>VkApplicationInfo</type>*
                    <name>pAppInfo</name></member>
            <member>const <type>VkAllocCallbacks</type>*
                    <name>pAllocCb</name></member>
            <member><type>uint32_t</type>
                    <name>extensionCount</name></member>
            <member>const <type>char</type>*const*
                    <name>ppEnabledExtensionNames</name></member>
        </type>
        ...
\end{verbatim}

If any of the member types are types also defined in the header, make sure
to enclose those type names in nested \tag{type} tags, as shown above. Basic
C types should not be tagged.

If the type is a C \code{union}, rather than a \code{struct}, then set the
value of the \attr{category} attribute to \code{"union"} instead of
\code{"struct"}.


\subsection{How To Add An Enumerated Type}

For this example, assume we want to define a type corresponding to a C
\code{enum} defined as follows:

\begin{verbatim}
typedef enum {
    VK_DEVICE_CREATE_VALIDATION_BIT = 0x00000001,
    VK_DEVICE_CREATE_MULTI_DEVICE_IQ_MATCH_BIT = 0x00000002;
} VkDeviceCreateFlagBits.
\end{verbatim}

If \code{VkDeviceCreateFlagBits} is the type of a parameter to a command in
the API, or of a member in a structure or union, make sure that command
parameter or struct member's definition puts \code{VkDeviceCreateFlagBits}
in nested \tag{type} tags where it's used.

Otherwise, if the enumerated type is not used directly by a command in the
API, nor required by a chain of type dependencies for commands and structs,
an explicit \tag{type} dependency should be added to the \tag{feature} tag
in exactly the same fashion as described above for \code{struct} types.

Next, go to the line labelled

\begin{verbatim}
<!-- SECTION: Vulkan enumerant (token) definitions. -->
\end{verbatim}

Below this line, add an \tag{enums} tag whose \attr{name} attribute matches
the \tag{type} name \code{VkDeviceCreateFlagBits}, and whose contents
correspond to the individual fields of the enumerated type:

\begin{verbatim}
<enums name="VkDeviceCreateFlagBits" type="bitmask">
    <enum bitpos="0" name="VK_DEVICE_CREATE_VALIDATION_BIT"/>
    <enum bitpos="1" name="VK_DEVICE_CREATE_MULTI_DEVICE_IQ_MATCH_BIT"/>
</enums>
\end{verbatim}

Several other attributes of the \tag{enums} tag can be set. In this case,
the \attr{type} attribute is set to \code{"bitmask"}, indicating that the
individual enumerants represent elements of a bitmask.

The individual \tag{enum} tags define the enumerants, just like the
definition for compile-time constants described above. In this case, because
the enumerants are bitfields, their values are specified using the
\attr{bitpos} attribute. The value of this attribute must be an integer in
the range $[0,31]$ specifying a single bit number, and the resulting value
is printed as a hexadecimal constant corresponding to that bit.

It is also possible to specify enumerant values using the \attr{value}
attribute, in which case the specified numeric value is passed through to
the C header unchanged.


\subsection{How to Add A Command}

For this example, assume we want to define the command:

\begin{verbatim}
VKAPI_ATTR VkResult VKAPI_CALL vkCreateInstance(
    const VkInstanceCreateInfo*                 pCreateInfo,
    VkInstance*                                 pInstance);
\end{verbatim}

Commands must always be explicitly required in the \tag{feature} tag. In
that tag, you can use an existing \tag{require} block including API features
which the new command should be grouped with, or define a new block. For
this example, add a new block, and require the command by using the
\tag{command} tag inside that block:

\begin{verbatim}
<!-- SECTION: Vulkan API interface definitions -->
<feature api="vulkan" name="VK_VERSION_1_0" number="1.0">
    ...
    <require comment="Device initialization">
        <command name="vkCreateInstance"/>
    </require>
    ...
</feature>
\end{verbatim}

The \tag{require} block may include a \attr{comment} attribute whose value
is a descriptive comment of the contents required within that block. The
comment is not currently used in header generation, but might be in the
future, so use comments which are polite and meaningful to users of
\code{vulkan.h}.

Then go to the \tag{commands} tag and add a new \tag{command} tag defining
the command, preferably sorted into alphabetic order with other commands for
ease of reading, as follows:

\begin{verbatim}
<!-- SECTION: Vulkan command definitions -->
<commands>
    ...
    <command>
        <proto><type>VkResult</type>
               <name>vkCreateInstance</name></proto>
        <param>const <type>VkInstanceCreateInfo</type>*
               <name>pCreateInfo</name></param>
        <param><type>VkInstance</type>*
               <name>pInstance</name></param>
    </command>
    ...
</commands>
\end{verbatim}

The \tag{proto} tag defines the return type and function name of the
command. The \tag{param} tags define the command's parameters in the order
in which they're passed, including the parameter type and name. The contents
are laid out in the same way as the structure \tag{member} tags described
previously.


\subsection{More Complicated API Representations}

The registry schema can represent a good deal of additional information, for
example by creating multiple \tag{feature} tags defining different API
versions and extensions. This capability is not yet relevant to Vulkan.
Those capabilities will be documented as they are needed.


\subsection{More Complicated Output Formats And Other Languages}

The registry schema is oriented towards C-language APIs. Types and commands
are defined using syntax which is a subset of C, especially for structure
members and command parameters. It would be possible to use a
language-independent syntax for representing such information, but since we
are writing a C API, any such representation would have to be converted into
C anyway at some stage.

The \code{vulkan.h} header is written using an {\bf output generator} object
in the Python scripts. This output generator is specialized for C, but the
design of the scripts is intended to support writing output generators for
other languages as well as purposes such as documentation (e.g. generating
Asciidoc fragments corresponding to types and commands for use in the API
specification and reference pages). When targeting other languages, the
amount of parsing required to convert type declarations into other languages
is small. However, it will probably be necessary to modify some of the
boilerplate C text, or specialize the tags by language, to support such
generators.


\subsection{Additional Semantic Tagging}

The schema is being extended to support semantic tags describing
various properties of API features, such as:

\begin{itemize}
\item constraints on allowed scalar values to function parameters (non-NULL,
      normalized floating-point, etc.)
\item length of arrays corresponding to function pointer parameters
\item miscellaneous properties of commands such as whether the application
      or system is responsible for threadsafe use; which queues they may be
      issued on; whether they are aliases or otherwise related to other
      commands; etc.
\end{itemize}

These tags will be used by other tools for purposes such as helping create
validation layers, generating serialization code, and so on. We'd like to
eventually represent everything about the API that is amenable to automatic
processing within the registry schema. Please make suggestions on the Gitlab
issue tracker.


\subsection{Stability of the XML Database and Schema}
\label{general:stability}

The Vulkan XML schema is evolving in response to corresponding
changes in the Vulkan API and ecosystem. Most such change will probably be
confined to adding attributes to existing tags and properly expressing the
relationships to them, and making API changes corresponding to accepted
feature requests. Changes to the schema should be described in the change
log of this document (see section~\ref{changelog}). Changes to the
\code{.xml} files and Python scripts are logged in Gitlab history.


\section{Change Log}
\label{changelog}

\begin{itemize}
\item 2016/02/07 - Add \attr{vendorids} tags for Khronos vendor IDs.
\item 2015/12/10 - Add \attr{author} and \attr{contact} attributes for
      \tag{extension} tags.
\item 2015/12/07 - Move \code{vulkan/vulkan.h} to a subdirectory.
\item 2015/12/01 - Add \tag{tags} tags for author tags.
\item 2015/11/18 - Bring documentation and schema up to date for extension
      enumerants.
\item 2015/11/02 - Bring documentation and schema up to date with several
      recent merges, including \tag{validity} tags. Still out of date WRT
      extension enumerants, but that will change soon.
\item 2015/09/08 - Rename {\tt threadsafe} attribute to \attr{externsync},
      and {\tt implicitunsafeparams} tag to \attr{implicitexternsync}.
\item 2015/09/07 - Update \tag{command} tag description to remove the
      \attr{threadsafe} attribute and replace it with a combination of
      \attr{threadunsafe} attributes on individual parameters, and
      \tag{implicitunsafeparams} tags describing additional unsafe objects
      for the command.
\item 2015/08/04 - Add \code{basetype} and \code{funcpointer}
      \attr{category} values for type tags, and explain the intended use and
      order in which types in each category are emitted.
\item 2015/07/02 - Update description of Makefile targets. Add
      descriptions of \attr{threadsafe}, \attr{queues}, and
      \attr{renderpass} attributes of \tag{command} tags in
      section~\ref{tag:command}, and of \attr{modified} attributes of
      \tag{param} tags in section~\ref{tag:command:param}
\item 2015/06/17 - add descriptions of allowed \attr{category} attribute
      values of \tag{type} tags, used to group and sort related categories
      of declarations together in the generated header.
\item 2015/06/04 - Add examples of making changes and additions to the
      registry in appendix~\ref{examples}.
\item 2015/06/03 - Move location to new ``vulkan'' Git repository. Add
      definition of \tag{type} tags for C struct/unions. Start adding
      examples of making changes in appendix~\ref{examples}.
\item 2015/06/02 - Branch from OpenGL specfile documentation and bring up to
      date with current Vulkan schema.
\end{itemize}

% % Copyright (c) 2013-2016 The Khronos Group Inc.
%
% Permission is hereby granted, free of charge, to any person obtaining a
% copy of this software and/or associated documentation files (the
% "Materials"), to deal in the Materials without restriction, including
% without limitation the rights to use, copy, modify, merge, publish,
% distribute, sublicense, and/or sell copies of the Materials, and to
% permit persons to whom the Materials are furnished to do so, subject to
% the following conditions:
%
% The above copyright notice and this permission notice shall be included
% in all copies or substantial portions of the Materials.
%
% THE MATERIALS ARE PROVIDED "AS IS", WITHOUT WARRANTY OF ANY KIND,
% EXPRESS OR IMPLIED, INCLUDING BUT NOT LIMITED TO THE WARRANTIES OF
% MERCHANTABILITY, FITNESS FOR A PARTICULAR PURPOSE AND NONINFRINGEMENT.
% IN NO EVENT SHALL THE AUTHORS OR COPYRIGHT HOLDERS BE LIABLE FOR ANY
% CLAIM, DAMAGES OR OTHER LIABILITY, WHETHER IN AN ACTION OF CONTRACT,
% TORT OR OTHERWISE, ARISING FROM, OUT OF OR IN CONNECTION WITH THE
% MATERIALS OR THE USE OR OTHER DEALINGS IN THE MATERIALS.

\documentclass{article}

% Various packages of possible use. {index} creates problems for some reason.
% \usepackage{index}
\def\specpdftitle{The Khronos Vulkan API Registry}
\usepackage[pdftex,bookmarksnumbered=true,linktocpage,plainpages=false,pdftitle={\specpdftitle},colorlinks=true]{hyperref}
\usepackage{amsmath,enumerate,epsfig,framed,mdframed,multicol,longtable,times,url}
\usepackage[normalem]{ulem}
% underscore allows use of _ instead of \_ in text. [strings] allows _ in file names
\usepackage[strings]{underscore}
\usepackage[breakwords,fit]{truncate}

% Index everything down through \paragraph
\setcounter{tocdepth}{5}

\makeatletter
\makeatother
\makeindex

% Some commonly used abbreviations

\def\code#1{{\tt #1}}
\def\tag#1{{\tt <#1>\index{<#1>}}}
% Because I keep mistyping \tag
\def\arg#1\tag{#1}
\def\attr#1{{\tt #1\index{#1}}}

\begin{document}

\title{The Khronos Vulkan API Registry for Vulkan}
\author{Jon Leech}
\date{Last updated 2016/02/07}
\maketitle

\begin{abstract}

This document describes the Khronos Vulkan API Registry schema, and provides
some additional information about using the registry and scripts to generate
a variety of outputs including C header files and several types of asciidoc
include files used in the Vulkan API specification and reference pages. The
underlying XML files and scripts are located on the Khronos Gitlab server,
currently in the {\em master} branch at URL

\begin{center}
{\bf \href{https://gitlab.khronos.org/vulkan/vulkan/tree/master/src/spec/}
          {https://gitlab.khronos.org/vulkan/vulkan/tree/master/src/spec/}}
\end{center}

\end{abstract}

\tableofcontents

\section{Introduction}

The registry uses an XML representation of the Vulkan API and a set of
Python scripts to manipulate the XML, based on the lxml Python bindings. It
includes an XML schema and validator.

The registry schema is based on, but not identical to the schema used for
the previously published OpenGL, OpenGL ES and EGL API registries. The
schema was extended to represent additional types and concepts not needed
for those APIS, such as structure and enumerant types.

The processed C header file corresponding to the registry is checked in
under src/vulkan/vulkan.h .


\section{Getting Started}
\label{starting}

You will need some tools installed.
For Linux systems, you will need the following packages installed (Debian
package names are noted, Ubuntu package names are probably the same).

\begin{itemize}
\item Python 3.x (\code{python3})
\item The lxml Python package (\code{python3-lxml})
\item Libxml (\code{libxml2})
\item A make tool, such as GNU make
\end{itemize}

The toolchain has also been run successfully on Microsoft Windows in the
Cygwin environment, and the individual Python-based components of the
toolchain can be run directly from the Windows command line. More details on
this are under src/spec/README in Git.

Once you have the right tools installed, perform the following steps:

\begin{itemize}
\item Check out the ``vulkan'' repository linked above from Khronos Gitlab
      (there are instructions at the link)
\item Go to \code{vulkan/src/spec} in your checked-out repo
\item Invoke \code{make clobber ; make install}
\end{itemize}

This should regenerate \code{vulkan.h} and install it in
\code{../vulkan/vulkan.h}. The result should be identical to the
version you just pulled from Gitlab (use \code{git diff
../vulkan/vulkan.h} to check).

Other Makefile targets include:

\begin{itemize}
\item \code{vulkan-docs} - regenerate the API specification and reference
      page asciidoc include files in \code{../doc/specs/vulkan/} under the
      subdirectories \code{enums}, \code{flags}, \code{protos}, and
      \code{structs}. These files are pulled in by the ref pages and API
      spec to put function prototypes and struct and enum declarations in
      documentation.

      You should probably {\bf not} update these files if you are proposing
      an API change - just the \code{vulkan.h} header. When a git merge is
      accepted including your change, the docs will be updated at that time.
      Including the dozens of include files in your merge makes it hard to
      read and adds no information beyond the header updates. Sometimes it
      is useful to include just one or two of the include files for context
      of a global change, however.

      However, you are strongly encouraged to updated the {\bf
      non-}autogenerated docs under \code{../doc/specs/vulkan/man/*.txt} to
      reflect API changes.
\item \code{full_install} - equivalent to \code{install} followed by
      \code{vulkan-docs}.
\item \code{validate} - validate \code{vk.xml} against the XML schema.
      Recommended if you're making nontrivial changes.
%% \item \code{vk.json} - generate JSON version of \code{vk.xml}. Currently the
%%       converter is generating syntactically but not semantically legal JSON,
%%       and it's difficult to map the XML schemaanso don't use this target yet.
\item \code{readme.pdf} - regenerate this document from the LaTeX source.
      Most people will never need to do this. If you do, you must have
      pdflatex installed, preferably from the TeTeX distribution.
\end{itemize}

If you just want to modify the API, changing \code{vk.xml} and running
``make'' should be all that's needed. See appendix~\ref{examples} for some
examples of modifying the XML.

If you want to repurpose the registry for reasons other than header file
and ref page include generation, or to generate headers for languages
other than C, start with the Makefile rules and inspect the files
\code{vk.xml}, \code{genvk.py}, \code{reg.py}, and \code{generator.py}.

If you're using other platforms, merge requests with additional
documentation on using the tools on those platforms would be very helpful.

\subsection{Header Generation Script - \code{genvk.py}}

When generating header files using the \code{genvk.py} script, an API name
and profile name are required, as shown in the Makefile examples.
Additionally, specific API versions and extensions can be required or
excluded. Based on this information, the generator script extracts the
relevant interfaces and creates a C-language header file for them.
\code{genvk.py} contains predefined generator options for the current
development version of Vulkan 1.0.

The generator script is intended to be generalizable to other languages by
writing new generator classes. Such generators would have to rewrite the C
types and definitions in the XML to something appropriate to their language.

\subsection{Registry Processing Script - \code{reg.py}}

XML processing is done in \code{reg.py}, which contains several objects and
methods for loading registries and extracting interfaces and extensions for
use in header generation. There is some internal documentation in the form
of comments, although nothing more extensive exists yet.

\subsection{Output Generator Script - \code{generator.py}}

Once the registry is loaded, the \code{COutputGenerator} class defined in
\code{generator.py} is used to create a header file. The
\code{DocOutputGenerator} class is used to create the asciidoc include
files. Output generators for other purposes can be added as needed.


\section{Vulkan Registry Schema}
\label{schema}

The format of the Vulkan registry is a top level \tag{registry} tag
containing \tag{types}, \tag{enums}, \tag{commands}, \tag{feature}, and
\tag{extension} tags describing the different elements of an API, as
explained below. This description corresponds to a formal Relax NG schema
file, \code{registry.rnc}, against which the XML registry files can be
validated.

At present the only registry in this schema is the core Vulkan API registry,
\code{vk.xml}.


\subsection{Profiles}
\label{schema:profile}

Types and enumerants can have different definitions depending on the API
profile requested. This capability is not used in the current Vulkan API but
may be in the future. Features and extensions can include some elements
conditionally depending on the API profile requested.


\subsection{API Names}
\label{schema:apiname}

The schema supports, but does not currently use an \attr{api} attribute on
several tags. This is an arbitrary string, specified at header generation
time, for labelling properties of a specific API or API profile. The string
can be, but is not necessarily, an actual API name. Names starting with ``vk''
are suggested if and when we start defining profiles of Vulkan.


\section{Registry Root (\tag{registry} tag)}
\label{schema:root}

A \tag{registry} contains the entire definition of one or more related
APIs.

\subsection{Attributes of \tag{registry} tags}

None.

\subsection{Contents of \tag{registry} tags}

Zero or more of each of the following tags, normally in this order
(although order shouldn't be important):

\begin{itemize}
\item \tag{comment} - Contains arbitrary text, such as a copyright
      statement.
\item \tag{vendorids} (see section~\ref{tag:vendorids}) - defines Khronos
      vendor IDs, described in detail in the ``Layers and Extensions''
      appendix of the Vulkan Specification.
\item \tag{tags} (see section~\ref{tag:tags}) - defines author prefixes used
      for extensions and layers. Prefixes are described in detail in the
      ``Layers and Extensions'' appendix of the Vulkan Specification.
\item \tag{types} (see section~\ref{tag:types}) - defines API types.
      Usually only one tag is used.
\item \tag{enums} (see section~\ref{tag:enums}) - defines API token names
      and values. Usually multiple tags are used. Related groups may be
      tagged as an enumerated type corresponding to a \tag{type} tag, and
      resulting in a C \code{enum} declaration. This ability is heavily used
      in the Vulkan API.
\item \tag{commands} (see section~\ref{tag:commands}) - defines API
      commands (functions). Usually only one tag is used.
\item \tag{feature} (see section~\ref{tag:feature}) - defines API
      feature interfaces (API versions, more or less). One tag per
      feature set.
\item \tag{extensions} (see section~\ref{tag:extensions}) - defines API
      extension interfaces. Usually only one tag is used, wrapping many
      extensions.
\end{itemize}


\section{Vendor IDs (\tag{vendorids} tag)}
\label{tag:vendorids}

The \tag{vendorids} tag contains individual \tag{vendorid} tags defining
vendor IDs for physical devices which do not have PCI vendor IDs.

Each \tag{vendorid} tag contains information defining a single vendor ID.

\subsection{Attributes of \tag{vendorid} tags}

\begin{itemize}
\item \attr{name} - required. The author prefix, as registered with Khronos.
      This must match an author prefix in the \attr{name} field of a
      \tag{tag} tag.
\item \attr{id} - required. The reserved vendor ID, as a hexadecimal number.
\item \attr{comment} - optional. Arbitrary string (unused).
\end{itemize}


\section{Author Prefixes (\tag{tags} tag)}
\label{tag:tags}

The \tag{tags} tag contains individual \tag{tag} tags defining each of the
reserved author prefixes used by extension and layer authors.

Each \tag{tag} tag contains information defining a single author prefix.

\subsection{Attributes of \tag{tag} tags}

\begin{itemize}
\item \attr{name} - required. The author prefix, as registered with Khronos.
      A short, upper-case string, usually an abbreviation of an author,
      project or company name.
\item \attr{author} - required. The author name, such as a full company
      or project name.
\item \attr{contact} - required. The contact who registered or is currently
      responsible for extensions and layers using the prefix, including
      sufficient contact information to reach the contact such as individual
      name together with email address, Github username, or other contact
      information.
\end{itemize}


\section{API types (\tag{types} tag)}
\label{tag:types}

The \tag{types} tag contains individual \tag{type} tags describing each
of the derived types used in the API.

Each \tag{type} tag contains information which can be used to generate C
code corresponding to the type. In many cases, this is simply legal C code
with attributes or embedded tags denoting the type name and other types used
in defining this type. In some cases, additional attribute and embedded type
information is used to generate more complicated C types.


\subsection{Attributes of \tag{type} tags}

\begin{itemize}
\item \attr{requires} - optional. Another type name this type requires to
      complete its definition.
\item \attr{name} - optional. Name of this type (if not defined in the tag
      body).
\item \attr{api} - optional. An API name (see \tag{feature} below) which
      specializes this definition of the named type, so that the same API
      types may have different definitions for e.g. GL ES and GL. This is
      unlikely to be used in Vulkan, where a single API supports desktop and
      mobile devices, but the functionality is retained.
\item \attr{category} - optional. A string which indicates that this type
      contains a more complex structured definition. At present the only
      accepted categories are \code{basetype} \code{bitmask}, \code{define},
      \code{enum}, \code{funcpointer}, \code{group}, \code{handle},
      \code{include}, \code{struct}, and \code{union}, as described below.
\item \attr{comment} - optional. Arbitrary string (unused).
\item \attr{parent} only applicable if category is \code{handle}. Notes another type with
      the \code{handle} category that acts as a parent object for this type.
\item \attr{returnedonly} only applicable if category is \code{struct} or \code{union}.
      Notes that this struct/union is going to be filled in by the API, rather than an
      application filling it out and passing it to the API.
\end{itemize}

\subsection{Contents of \tag{type} tags}

The valid contents depend on the \attr{category} attribute.

\subsubsection{Enumerated types - \attr{category} "enum"}

If the \attr{category} tag has the value \code{enum}, the type is a C
enumeration. The body of the tag is ignored in this case. The value of the
\attr{name} attribute must be provided and must match the \attr{name}
attribute of a \tag{enums} tag (see section~\ref{tag:enums}). The enumerant
values defined within the \tag{enums} tag are used to generate a C
\code{enum} type declaration.

\subsubsection{Structure types - \attr{category} "struct" or "union"}

If the \attr{category} tag has the values \code{struct} or \code{union}, the
type is a C structure or union, respectively. In this case, the \attr{name}
attribute must be provided, and the contents of the \tag{type} tag are a
series of \tag{member} tags defining the members of the aggregate type, in
order, followed by an optional \tag{validity} tag including asciidoc
validation language for the structure contents.


\paragraph{Structure member (\tag{member}) tags}
\par\vskip 11pt

The \tag{member} tag defines the type and name of a structure or union
member.

\paragraph{Attributes of \tag{member} tags}

\begin{itemize}
\item \attr{len} - if the member is an array, len may be one or more of the following
           things, separated by commas (one for each array indirection): another
           member of that struct, ``null-terminated'' for a string, ``1'' to indicate it's
           just a pointer (used for nested pointers), or a latex equation (prefixed with
           ``latex:'')
\item \attr{externsync} - denotes that the member should be externally synchronized
           when accessed by Vulkan
\item \attr{optional} - whether this value can be omitted by providing \code{NULL} (for
          pointers), \code{VK_NULL_HANDLE} (for handles) or 0 (for bitmasks/values)
\item \attr{noautovalidity} - prevents automatic validity language being generated
          for the tagged item. Only suppresses item-specific validity - parenting issues
          etc. are still captured.
\end{itemize}

\paragraph{Contents of \tag{member} tags}
\par\vskip 11pt

The text elements of a \tag{member} tag, with all other tags removed,
is a legal C declaration of a struct or union member. In addition it may
contain two semantic tags:

\begin{itemize}
\item The \tag{type} tag is optional. It contains text which is a valid type
      name found in another \tag{type} tag, and indicates that this type
      must be previously defined for the definition of the command to
      succeed. Builtin C types should not be wrapper in \tag{type} tags.
\item The \tag{name} tag is required, and contains the struct/union member
      name being described.
\end{itemize}

\paragraph{Validation (\tag{validity}) tags}
\par\vskip 11pt

The \tag{validity} tag, if present defines valid use cases and values for
structure members.

\paragraph{Contents of \tag{validity} tags}
\par\vskip 11pt

Each \tag{validity} tag contains zero or more \tag{usage} tags. Each
\tag{usage} tag is intended to represent a specific validation requirement
for the structure and include arbitrary asciidoc text describing that
requirement.


\subsubsection{All other types}

If the \attr{category} attribute is one of \code{basetype}, \code{bitfield},
\code{define}, \code{funcpointer}, \code{group}, \code{handle} or
\code{include}, or is not specified, \tag{type} contains text which is legal
C code for a type declaration. It may also contain embedded tags:

\begin{itemize}
\item \tag{type} - nested type tags contain other type names which are
      required by the definition of this type.
\item \tag{apientry/} - insert a platform calling convention macro here
      during header generation, used mostly for function pointer types.
\item \tag{name} - contains the name of this type (if not defined in the
      tag attributes).
\end{itemize}

There is no restriction on which sorts of definitions may be made in a given
category, although the contents of tags with \attr{category} \code{enum},
\code{struct} or \code{union} are interpreted specially as described above.

However, when generating the header, types within each category are grouped
together, and categories are generated in the order given by the following
list. Therefore, types in a category should correspond to the intended
purpose given for that category. If this recommendation is not followed, it
is possible that the resulting header file will not compile due to
out-of-order type dependencies.

\begin{itemize}
  \item \code{include} ({\tt \#include} directives)
  \item \code{define} (macro {\tt \#define} directives)
  \item \code{basetype} (scalar typedefs, such as the definition of
        {\tt VkFlags})
  \item \code{handle} (invocations of macros defining scalar types such as
        {\tt VkInstance})
  \item \code{enum} (enumeration types and {\tt \#define} for constant values)
  \item \code{group} (currently unused)
  \item \code{bitmask} (enumeration types whose members are bitmasks)
  \item \code{funcpointer} (function pointer typedefs)
  \item \code{struct} and \code{union} together (struct and union types)
\end{itemize}


\subsection{Example of a \tag{types} tag}
\label{tag:types:example}

\begin{verbatim}
<types>
    <type name="stddef">#include &lt;stddef.h&gt;</type>
    <type requires="stddef">typedef ptrdiff_t <name>VKlongint</name>;</type>
    <type name="VkEnum" category="enum"/>
    <type category="struct" name="VkStruct">
        <member><type>VkEnum</type> <name>srcEnum</name></member>
        <member><type>VkEnum</type> <name>dstEnum</name></member>
    </type>
</types>

<enums name="VkEnum" type="enum">
    <enum value="0" name="VK_ENUM_ZERO"/>
    <enum value="42" name="VK_ENUM_FORTY_TWO"/>
</enums>
\end{verbatim}

The {\tt VkStruct} type is defined to require the types {\tt VkEnum} and
{\tt VKlongint} as well. If {\tt VkStruct} is in turn required by a command
or another type during header generation, it will result in the following
declarations:

\begin{verbatim}
#include <stddef.h>
typedef ptrdiff_t VKlongint.

typedef enum {
    VK_ENUM_ZERO = 0,
    VK_ENUM_FORTY_TWO = 42
} VkEnum;

typedef struct {
    VkEnum    dstEnum;
    VkLongint dstVal;
} VkStruct;
\end{verbatim}

Note that the angle brackets around \code{stddef.h} are represented as XML
entities in the registry. This could also be done using a CDATA block but
unless there are many characters requiring special representation in XML,
using entities is preferred.


\section{Enumerant Blocks (\tag{enums} tag)}
\label{tag:enums}

The \tag{enums} tags contain individual \tag{enum} tags describing each of
the token names used in the API. In some cases these correspond to a C
\code{enum}, and in some cases they are simply compile-time constants (e.g.
\code{\#define}).

\subsection{Attributes of \tag{enums} tags}

\begin{itemize}
\item \attr{name} - optional. String naming the C \code{enum} type whose
      members are defined by this enum group. If present, this attribute
      should match the \attr{name} attribute of a corresponding \tag{type}
      tag.
\item \attr{type} - optional. String describing the data type of the values
      of this group of enums. At present the only accepted categories are
      \code{enum} and \code{bitmask}, as described below.
\item \attr{start}, \attr{end} - optional. Integers defining the start and
      end of a reserved range of enumerants for a particular vendor or
      purpose. \attr{start} must be $\leq$ \attr{end}. These fields define
      formal enumerant allocations, and are made by the Khronos Registrar on
      request from implementers following the enum allocation policy.
\item \attr{vendor} - optional. String describing the vendor or purpose to
      whom a reserved range of enumerants is allocated.
\item \attr{comment} - optional. Arbitrary string (unused).
\end{itemize}

\subsection{Contents of \tag{enums} tags}

Each \tag{enums} block contains zero or more \tag{enum} and \tag{unused}
tags, in arbitrary order (although they are typically ordered by sorting
on enumerant values, to improve human readability).

\subsection{Example of \tag{enums} tags}

An example showing a tag with attribute \attr{type}="\code{enum}" is given
above in section~\ref{tag:types:example}. The following example is for
non-enumerated tokens.

\begin{verbatim}
<enums>
    <enum value="256" name="VK_MAX_EXTENSION_NAME"/>
    <enum value="MAX_FLOAT"  name="VK_LOD_CLAMP_NONE"/>
</enums>
\end{verbatim}

When processed into a C header, and assuming all these tokens were
required, this results in

\begin{verbatim}
#define VK_MAX_EXTENSION_NAME   256
#define VK_LOD_CLAMP_NONE       MAX_FLOAT
\end{verbatim}


\section{Enumerants (\tag{enum} tag)}
\label{tag:enum}

Each \tag{enum} tag defines a single Vulkan (or other API) token.

\subsection{Attributes of \tag{enum} tags}

\begin{itemize}
\item \attr{value} or \attr{bitpos} - exactly one of these is allowed and
      required. \attr{value} is an enumerant value in the form of a legal C
      constant (usually a literal decimal or hexadecimal integer, though
      arbitrary strings are allowed). \attr{bitpos} is a literal integer bit
      position in a bitmask.
\item \attr{name} - required. Enumerant name, a legal C preprocessor token
      name.
\item \attr{api} - optional. An API name which specializes this definition of
      the named enum, so that different APIs may have different values for
      the same token. May be used to address a subtle incompatibilities.
\item \attr{type} - optional. Used only when \attr{value} is specified. C
      suffix for the value to force it to a specific type. Currently only
      \code{u} and \code{ull} are used, for \code{unsigned} 32- and 64-bit
      integer values, respectively. Separated from \attr{value} since this
      eases parsing and sorting of values, and rarely used.
\item \attr{alias} - optional. Name of another enumerant this is an alias
      of, used where token names have been changed as a result of profile
      changes or for consistency purposes. An enumerant alias is simply a
      different \attr{name} for the exact same \attr{value} or
      \attr{bitpos}.
\end{itemize}

\subsection{Contents of \tag{enum} tags}

\tag{enum} tags have no allowed contents. All information is contained
in the attributes.


\section{Unused Enumerants (\tag{unused} tag)}
\label{tag:unused}

Each \tag{unused} tag defines a range of enumerants which is allocated, but
not yet assigned to specific enums. This just tracks the unused values for
the Registrar's use, and is not used for header generation.

\subsection{Attributes of \tag{unused} tags}

\begin{itemize}
\item \attr{start} - required, \attr{end} - optional. Integers defining the
      start and end of an unused range of enumerants. \attr{start} must be
      $\leq$ \attr{end}. If \attr{end} is not present, then \attr{start}
      defines a single unused enumerant. This range should not exceed the
      range reserved by the surrounding \tag{enums} tag.
\item \attr{vendor} - optional. String describing the vendor or purposes to
      whom a reserved range of enumerants is allocated. Usually identical to
      the \attr{vendor} attribute of the surrounding \attr{enums} block.
\item \attr{comment} - optional. Arbitrary string (unused).
\end{itemize}

\subsection{Contents of \tag{unused} tags}

None.


\section{Command Blocks (\tag{commands} tag)}
\label{tag:commands}

The \tag{commands} tag contains definitions of each of the functions
(commands) used in the API.

\subsection{Attributes of \tag{commands} tags}

None.

\subsection{Contents of \tag{commands} tags}

Each \tag{commands} block contains zero or more \tag{command} tags, in
arbitrary order (although they are typically ordered by sorting on the
command name, to improve human readability).


\section{Commands (\tag{command} tag)}
\label{tag:command}

The \tag{command} tag contains a structured definition of a single API
command (function).

\subsection{Attributes of \tag{command} tags}

\begin{itemize}
\item \attr{queues} - optional. A string identifying the command queues
      this command can be placed on. The format of the string is one or
      more of the terms \code{"compute"}, \code{"dma"}, and
      \code{"graphics"}, with multiple terms separated by commas
      (\code{","}).
\item \attr{renderpass} - optional. A string identifying whether the
      command can be issued only inside a render pass (\code{"inside"}),
      only outside a render pass (\code{"outside"}), or both
      (\code{"both"}).
\item \attr{cmdbufferlevel} - optional. A string identifying the command buffer
      levels that this command can be called by. The format of the string is
      one or more of the terms \code{"primary"} and \code{"secondary"}, with
      multiple terms separated by commas (\code{","}).
\item \attr{comment} - optional. Arbitrary string (unused).
\end{itemize}

\subsection{Contents of \tag{command} tags}

\begin{itemize}
\item \tag{proto} is required and must be the first element. It is a tag
      defining the C function prototype of a command as described below, up
      to the function name and return type but not including function
      parameters.
\item \tag{param} elements for each command parameter follow, defining
      its name and type, as described below. If a command takes no
      arguments, it has no \tag{param} tags.
\item An optional \tag{validity} tag including asciidoc validation language
      for the command parameters.
\end{itemize}

Following these elements, the remaining elements in a \tag{command}
tag are optional and may be in any order:

\begin{itemize}
\item \tag{alias} - optional. Has no attributes and contains a string which
      is the name of another command this command is an alias of, used when
      promoting a function from vendor to Khronos extension or Khronos
      extension to core API status. A command alias describes the case where
      there are two function names which resolve to the {\bf same} entry
      point in the underlying layer stack.
\item \tag{description} - optional. Unused text.
\item \tag{implicitexternsyncparams} - optional. Contains a list of
      \tag{param} tags, each containing Asciidoc source text describing an
      object which is not a parameter of the command, but is related to one,
      and which also requires external synchronization as described in
      section~\ref{tag:command:param:attr}. The text is intended to be
      incorporated into the API specification.
\end{itemize}


\subsection{Command prototype (\tag{proto} tags)}
\label{tag:command:proto}

The \tag{proto} tag defines the return type and name of a command.

\subsubsection{Attributes of \tag{proto} tags}

None.

% \begin{itemize}
% \item \attr{group} - group name, an arbitrary string.
% \end{itemize}
%
% If the group name is defined, it may be interpreted as described in
% section~\ref{tag:group:meaning}.

\subsubsection{Contents of \tag{proto} tags}

The text elements of a \tag{proto} tag, with all other tags removed, is
legal C code describing the return type and name of a command. In addition
to text, it may contain two semantic tags:

\begin{itemize}
\item The \tag{type} tag is optional, and contains text which is a valid
      type name found in a \tag{type} tag. It indicates that this type must
      be previously defined for the definition of the command to succeed.
      Builtin C types, and any derived types which are expected to be found
      in other header files, should not be wrapped in \tag{type} tags.
\item The \tag{name} tag is required, and contains the command name
      being described.
\end{itemize}

\subsection{Command parameter (\tag{param} tags)}
\label{tag:command:param}

The \tag{param} tag defines the type and name of a parameter. Its contents
are very similar to the \tag{member} tag used to define struct and union
members.

\subsubsection{Attributes of \tag{param} tags}
\label{tag:command:param:attr}

\begin{itemize}
\item \attr{len} - if the param is an array, len may be one or more of the following
           things, separated by commas (one for each array indirection): another
           param of that command, ``null-terminated'' for a string, ``1'' to indicate it's
           just a pointer (used for nested pointers), or a latex equation (prefixed with
           ``latex:'')
\item \attr{optional} - whether this value can be omitted by providing \code{NULL} (for
          pointers), \code{VK_NULL_HANDLE} (for handles) or 0 (for bitmasks/values)
\item \attr{noautovalidity} -  prevents automatic validity language being generated
          for the tagged item. Only suppresses item-specific validity - parenting issues
          etc. are still captured.
\item \attr{externsync} - optional. A boolean string, which must have the
      value \code{"true"} if present, indicating that this parameter (e.g.
      the object a handle refers to, or the contents of an array a pointer
      refers to) is modified by the command, and is not protected against
      modification in multiple app threads. Parameters which do not have
      this attribute are assumed to not require external synchronization.
\end{itemize}

\subsubsection{Contents of \tag{param} tags}

The text elements of a \tag{param} tag, with all other tags removed, is
legal C code describing the type and name of a function parameter. In
addition it may contain two semantic tags:

\begin{itemize}
\item The \tag{type} tag is optional, and contains text which is a
      valid type name found in \tag{type} tag, and indicates that this
      type must be previously defined for the definition of the command
      to succeed. Builtin C types, and any derived types which are
      expected to be found in other header files, should not be wrapped
      in \tag{type} tags.
\item The \tag{name} tag is required, and contains the parameter name being
      described.
\end{itemize}

\subsection{Example of a \tag{commands} tag}

\begin{verbatim}
<commands>
    <command>
        <proto><type>VkResult</type> <name>vkCreateInstance</name></proto>
        <param>const <type>VkInstanceCreateInfo</type>* <name>pCreateInfo</name></param>
        <param><type>VkInstance</type>* <name>pInstance</name></param>
    </command>
</commands>
\end{verbatim}

When processed into a C header, this results in

\begin{verbatim}
VkResult vkCreateInstance(
    const VkInstanceCreateInfo* pCreateInfo,
    VkInstance* pInstance);
\end{verbatim}

\subsection{Parameter validation (\tag{validity}) tags}
\label{tag:command:validity}

The \tag{validity} tag, if present defines valid use cases and values for
command parameters

\paragraph{Contents of \tag{validity} tags}
\par\vskip 11pt

Each \tag{validity} tag contains zero or more \tag{usage} tags. Each
\tag{usage} tag is intended to represent a specific validation requirement
for the command, and contains arbitrary asciidoc text describing that
requirement.


\section{API Features / Versions (\tag{feature} tag)}
\label{tag:feature}

API features are described in individual \tag{feature} tags. A feature is
the set of interfaces (enumerants and commands) defined by a particular API
and version, such as Vulkan 1.0, and includes all profiles of that API and
version.

\subsection{Attributes of \tag{feature} tags}

\begin{itemize}
\item \attr{api} - required. API name this feature is for (see
      section~\ref{schema:apiname}), such as \code{vk}.
\item \attr{name} - required. Version name, used as the C preprocessor token
      under which the version's interfaces are protected against multiple
      inclusion. Example: \code{VK\_VERSION\_1\_0}.
\item \attr{number} - required. Feature version number, usually a string
      interpreted as $majorNumber.minorNumber$. Example: \code{4.2}.
\item \attr{protect} - optional. An additional preprocessor token used to
      protect a feature definition. Usually another feature or extension
      \attr{name}. Rarely used, for odd circumstances where the definition
      of a feature or extension requires another to be defined first.
\item \attr{comment} - optional. Arbitrary string (unused).
\end{itemize}

\subsection{Contents of \tag{feature} tags}

Zero or more \tag{require} and \tag{remove} tags (see
section~\ref{tag:required}), in arbitrary order. Each tag describes a
set of interfaces that is respectively required for, or removed from,
this feature, as described below.

\subsection{Example of a \tag{feature} tag}

\begin{verbatim}
<feature api="vulkan" name="VK_VERSION_1_0" number="1.0">
    <require comment="Header boilerplate">
        <type name="vk_platform"/>
    </require>
    <require comment="API constants">
        <enum name="VK_MAX_PHYSICAL_DEVICE_NAME"/>
        <enum name="VK_LOD_CLAMP_NONE"/>
    </require>
    <require comment="Device initialization">
        <command name="vkCreateInstance"/>
    </require>
</feature>
\end{verbatim}

When processed into a C header for Vulkan, this results in:

\begin{verbatim}
#ifndef VK_VERSION_1_0
#define VK_VERSION_1_0 1
#define VK_MAX_EXTENSION_NAME   256
#define VK_LOD_CLAMP_NONE       MAX_FLOAT
typedef VkResult (VKAPI_PTR *PFN_vkCreateInstance)(const VkInstanceCreateInfo* pCreateInfo, VkInstance* pInstance);
#ifdef VK_PROTOTYPES
VKAPI_ATTR VkResult VKAPI_CALL vkCreateInstance(
    const VkInstanceCreateInfo*                 pCreateInfo,
    VkInstance*                                 pInstance);
#endif
#endif /* VK_VERSION_1_0 */
\end{verbatim}


\section{Extension Blocks (\tag{extensions} tag)}
\label{tag:extensions}

The \tag{extensions} tag contains definitions of each of the extenions
which are defined for the API.

\subsection{Attributes of \tag{extensions} tags}

None.

\subsection{Contents of \tag{extensions} tags}

Each \tag{extensions} block contains zero or more \tag{extension} tags,
each describing an API extension, in arbitrary order (although they are
typically ordered by sorting on the extension name, to improve human
readability).


\section{API Extensions (\tag{extension} tag)}
\label{tag:extension}

API extensions are described in individual \tag{extension} tags. An
extension is the set of interfaces defined by a particular API extension
specification, such as \code{ARB\_multitexture}. \tag{extension} is
similar to \tag{feature}, but instead of having \attr{version} and
\attr{profile} attributes, instead has a \attr{supported} attribute,
which describes the set of API names which the extension can potentially
be implemented against.

\subsection{Attributes of \tag{extension} tags}

\begin{itemize}
\item \attr{name} - required. Extension name, following the conventions
      in the Vulkan Specification.
      Example: \code{name="VK\_VERSION\_1\_0"}.
\item \attr{number} - required. A decimal number which is the registered,
      unique extension number for \attr{name}.
\item \attr{supported} - required. A regular expression, with an implicit
      \code{\^{}} and \code{\$} bracketing it, which should match the
      \attr{api} tag of a set of \tag{feature} tags.
\item \attr{protect} - optional. An additional preprocessor token used to
      protect an extension definition. Usually another feature or extension
      \attr{name}. Rarely used, for odd circumstances where the definition
      of an extension requires another extension or a header file to be
      defined first.
\item \attr{author} - optional. The author name, such as a full company
      name. If not present, this can be taken from the corresponding
      \tag{tag} attribute. However, \code{EXT} and other multi-vendor
      extensions may not have a well-defined author or contact in the tag.
\item \attr{contact} - optional. The contact who registered or is currently
      responsible for extensions and layers using the tag, including
      sufficient contact information to reach the contact such as individual
      name together with email address, Github username, or other contact
      information. If not present, this can be taken from the corresponding
      \tag{tag} attribute just like \attr{author}.
\item \attr{comment} - optional. Arbitrary string (unused).
\end{itemize}

\subsection{Contents of \tag{extension} tags}

Zero or more \tag{require} and \tag{remove} tags (see
section~\ref{tag:required}), in arbitrary order. Each tag describes a
set of interfaces that is respectively required for, or removed from,
this extension, as described below.

\subsection{Example of an \tag{extensions} tag}

\begin{verbatim}
<extension name="VK_KHR_display_swapchain" number="4" supported="vulkan">
    <require>
        <enum value="9" name="VK_KHR_DISPLAY_SWAPCHAIN_SPEC_VERSION"/>
        <enum value="4" name="VK_KHR_DISPLAY_SWAPCHAIN_EXTENSION_NUMBER"/>
        <enum value="&quot;VK_KHR_display_swapchain&quot;"
              name="VK_KHR_DISPLAY_SWAPCHAIN_EXTENSION_NAME"/>
        <type name="VkDisplayPresentInfoKHR"/>
        <command name="vkCreateSharedSwapchainsKHR"/>
    </require>
</extension>
\end{verbatim}

The \attr{supported} attribute says that the extension is defined for the
default profile (\code{vulkan}). When processed into a C header for the
\code{vulkan} profile, this results in header contents something like
(assuming corresponding definitions of the specified \tag{type} and
\tag{command} elsewhere in the XML):

\begin{verbatim}
#define VK_KHR_display_swapchain 1
#define VK_KHR_DISPLAY_SWAPCHAIN_SPEC_VERSION 9
#define VK_KHR_DISPLAY_SWAPCHAIN_EXTENSION_NUMBER 4
#define VK_KHR_DISPLAY_SWAPCHAIN_EXTENSION_NAME "VK_KHR_display_swapchain"

typedef struct VkDisplayPresentInfoKHR {
    VkStructureType                             sType;
    const void*                                 pNext;
    VkRect2D                                    srcRect;
    VkRect2D                                    dstRect;
    VkBool32                                    persistent;
} VkDisplayPresentInfoKHR;

typedef VkResult (VKAPI_PTR *PFN_vkCreateSharedSwapchainsKHR)(
    VkDevice device, uint32_t swapchainCount,
    const VkSwapchainCreateInfoKHR* pCreateInfos,
    const VkAllocationCallbacks* pAllocator,
    VkSwapchainKHR* pSwapchains);

#ifdef VK_PROTOTYPES
VKAPI_ATTR VkResult VKAPI_CALL vkCreateSharedSwapchainsKHR(
    VkDevice                                    device,
    uint32_t                                    swapchainCount,
    const VkSwapchainCreateInfoKHR*             pCreateInfos,
    const VkAllocationCallbacks*                pAllocator,
    VkSwapchainKHR*                             pSwapchains);
#endif
\end{verbatim}


\section{Required and Removed Interfaces (\tag{require} and \tag{remove} tags)}
\label{tag:required}

A \tag{require} block defines a set of interfaces (types, enumerants and
commands) {\em required} by a \tag{feature} or \tag{extension}. A
\tag{remove} block defines a set of interfaces {\em removed} by a
\tag{feature} (this is primarily useful for future profiles of an API which
may choose to deprecated and/or remove some interfaces - extensions should
never remove interfaces, although this usage is allowed by the schema).
Except for the tag name and behavior, the contents of \tag{require} and
\tag{remove} tags are identical.

\subsection{Attributes of \tag{require} and \tag{remove} tags}

\begin{itemize}
\item \attr{profile} - optional. String name of an API profile. Interfaces
      in the tag are only required (or removed) if the specified profile is
      being generated. If not specified, interfaces are required (or
      removed) for all API profiles.
\item \attr{comment} - optional. Arbitrary string (unused).
\item \attr{api} - optional. An API name (see section~\ref{schema:apiname}).
      Interfaces in the tag are only required (or removed) if the specified
      API is being generated. If not specified, interfaces are required (or
      removed) for all APIs.

      {\bf The \attr{api} attribute is only supported inside
      \tag{extension} tags,} since \tag{feature} tags already define a
      specific API.
\end{itemize}

\subsection{Contents of \tag{require} and \tag{remove} tags}

Zero or more of the following tags, in any order:

\begin{itemize}
\item \tag{command} specifies an required (or removed) command defined
      in a \tag{commands} block. The tag has no content, but contains
      attributes:

      \begin{itemize}
      \item \attr{name} - required. Name of the command.
      \item \attr{comment} - optional. Arbitrary string (unused).
      \end{itemize}
\item \tag{enum} specifies an required (or removed) enumerant defined in a
      \tag{enums} block. All forms of this tag support the following
      attributes:

      \begin{itemize}
      \item \attr{name} - required. Name of the enumerant.
      \item \attr{comment} - optional. Arbitrary string (unused).
      \end{itemize}

      There are two forms of \tag{enum} tags:

      {\bf Reference enums} simply pull in the definition of an enumerant
      given in a separate \tag{enums} block. Reference enums are the most
      common usage, and no attributes other than \attr{name} and
      \attr{comment} are supported for them.

      {\bf Extension enums} define the value of an enumerant inline in an
      \tag{extensions} block. There are several variants, depending on which
      additional tags are defined:

      \begin{itemize}
      \item Attributes \attr{value} and (optionally) \attr{type} define a
            constant value in the same fashion as an \tag{enum} tag in an
            \tag{enums} block (see section~\ref{tag:enum}).
      \item Attribute \attr{bitpos} defines a constant bitmask value in the
            same fashion as an \tag{enum} tag in an \tag{enums} block (see
            section~\ref{tag:enum}).
      \item Attribute \attr{extends} allows an extension enumerant to be
            added to a separately defined enumerated type whose name is
            specified by the contents of the \attr{extends} attribute (e.g.
            a \tag{type} tag with \attr{category} "enum", pulling in an
            \tag{enums} block).

            There are two ways to extend an enumerated type:

            \begin{itemize}
            \item If \attr{bitpos} is also specified, the tag defines a
                  bitmask value, but adds its definition to the enumeranted
                  type specified by \attr{extends} instead of as a
                  compile-time constant.
            \item If \attr{offset} is also specified, the tag adds a new
                  enumerant value to the enumerated type specified by
                  \attr{extends}. The actual value defined depends on the
                  extension number (the \attr{number} attribute of the
                  \tag{extensions} tag) and the offset, as defined in the
                  ``Layers and Extensions'' appendix of the Vulkan
                  Specification. The \attr{dir} attribute may also be
                  specified as {\tt dir="-"} if the calculated value should
                  be negative instead of positive. Negative enumerant values
                  are normally used only for Vulkan error codes.
            \end{itemize}
      \end{itemize}

      Examples of various types of extension enumerants are given below in
      section~\ref{tag:required:examples}.
\item \tag{type} specifies a required (or removed) type defined in a
      \tag{types} block. Most types are picked up implicitly by using the
      \tag{type} tags of commands, but in a few cases, additional types need
      to be specified explicitly. It is unlikely that a type would ever be
      removed, although this usage is allowed by the schema. The tag has no
      content, but contains elements:

      \begin{itemize}
      \item \attr{name} - required. Name of the type.
      \item \attr{comment} - optional. Arbitrary string (unused).
      \end{itemize}
\end{itemize}


\subsection{Examples of Extension Enumerants}
\label{tag:required:examples}

Examples of each of the supported extension enumerant \tag{enum} tags are
given below. Note that extension enumerants are supported only inside
\tag{extension} blocks - not in \tag{feature} blocks\footnote{
    However, we will have to define additional XML tags and/or syntax for
    future core versions of Vulkan, to properly tag and group core
    enumerants for Vulkan 1.0, 1.1, etc.}.

\begin{verbatim}
<extensions>
    <extension name="VK_KHR_test_extension" number="1" supported="vulkan">
        <require>
            <enum value="42" name="VK_KHR_theanswer"/>
            <enum bitpos="29" name="VK_KHR_bitmask"/>
            <enum offset="0" dir="-" extends="VkResult"
                  name="VK_ERROR_SURFACE_LOST_KHR"/>
            <enum offset="1" extends="VkResult"
                  name="VK_SUBOPTIMAL_KHR"/>
            <enum bitpos="31" extends="VkResult"
                  name="VK_KHR_EXTENSION_BITFIELD"/>
        </require>
    </extension>
</extensions>
\end{verbatim}

The corresponding header file will include definitions like this:

\begin{verbatim}
typedef enum VkResult {
    <previously defined VkResult enumerant values},
    VK_ERROR_SURFACE_LOST_KHR = -1000000000,
    VK_SUBOPTIMAL_KHR = 1000000001,
    VK_KHR_EXTENSION_BITFIELD = 0x80000000,
};

#define VK_KHR_test_extension 1
#define VK_KHR_theanswer 42
#define VK_KHR_bitmask 0x20000000
\end{verbatim}

\clearpage
\appendix

\section{Examples / FAQ / How Do I?}
\label{examples}

For people new to the Registry, it won't be immediately obvious how to make
changes. This section includes some tips and examples that will help you
make changes to the Vulkan headers by changing the Registry XML description.

First, follow the steps in section~\ref{starting} to get the Vulkan Gitlab
repository containing the registry and assemble the tools necessary to work
with the XML registry. Once you're able to regenerate \code{vulkan.h} from
\code{vk.xml}, you can start making changes.


\subsection{General Strategy}

If you are {\bf adding} to the API, perform the following steps to {\bf
create} the description of that API element:

\begin{itemize}
\item For each type, enum group, compile-time constant, and command being
      added, create appropriate new \tag{type}, \tag{enums}, \tag{enum}, or
      \tag{command} tags defining the interface in question.
\item Make sure that all added types and commands appropriately tag their
      dependencies on other types by adding nested \tag{type} tags.
\item Make sure that each new tag defines the name of the corresponding
      type, enum group, constant, or command, and that structure/union types
      and commands tag the types and names of all their members and
      parameters. This is essential for the automatic dependency process to
      work.
\end{itemize}

If you are {\bf modifying} existing APIs, just make appropriate changes
in the existing tags.

Once the definition is added, proceed to the next section to create
dependencies on the changed feature.


\subsection{API Feature Dependencies}

When you add new API elements, they will not result in corresponding changes
in the generated header unless they are {\bf required} by the interface
being generated. This makes it possible to include different API versions
and extensions in a single registry and pull them out as needed. So you must
introduce a dependency on new features in the corresponding \tag{feature}
tag.

Initially, the only API feature is Vulkan 1.0, so there is only one
\tag{feature} tag in \code{vk.xml}. You can find it by searching
for the following block of \code{vk.xml}:

\begin{verbatim}
<!-- SECTION: Vulkan API interface definitions -->
<feature api="vulkan" name="VK_VERSION_1_0" number="1.0">
\end{verbatim}

Inside the \tag{feature} tag are nested multiple \tag{require} tags. These
are just being used as a logical grouping mechanism for related parts of
Vulkan 1.0 at present, though they may have more meaningful roles in the
future if different API profiles are defined.


\subsubsection{API Feature Walkthrough}

This section walks through the first few required API features in the
\code{vk.xml} \tag{feature} tag, showing how each requirement pulls in type,
token, and command definitions and turns those into definitions in the C
header file \code{vulkan.h} .

Consider the first few lines of the \tag{feature}:

\begin{verbatim}
<require comment="Header boilerplate">
    <type name="vk_platform"/>
</require>
<require comment="API constants">
    <enum name="VK_MAX_PHYSICAL_DEVICE_NAME"/>
    <enum name="VK_MAX_EXTENSION_NAME"/>
    ...
</require>
<require comment="Device initialization">
    <command name="vkCreateInstance"/>
    ...
\end{verbatim}

The first \tag{require} block says to require a type named
\code{vk_platform}. If you look at the beginning of the \tag{types} section,
there's a corresponding definition:


\begin{verbatim}
<type name="vk_platform">#include "vk_platform.h"
#define VK_MAKE_VERSION(major, minor, patch) \
    ((major &lt;&lt; 22) | (minor &lt;&lt; 12) | patch)
    ...
\end{verbatim}

\noindent section which is invoked by the requirement and emits a bunch of
boilerplate C code. The explicit dependency isn't strictly required since
\code{vk_platform} will be required by many other types, but placing it
first causes this to appear first in the output file.

Note that \code{vk_platform} does not correspond to an actual C type, but
instead to a collection of freeform preprocessor includes and macros and
comments. Most other \tag{type} tags do define a specific type and are much
simpler, but this approach can be used to inject arbitrary C into
\code{vulkan.h} {\bf when there's no other way}. In general inserting
arbitrary C is strongly discouraged outside of specific special cases like
this.

The next \tag{require} block pulls in some compile-time constants. These
correspond to the definitions found in the first \tag{enums} section of
\code{vk.xml}:

\begin{verbatim}
<!-- SECTION: Vulkan enumerant (token) definitions. -->

<enums comment="Misc. hardcoded constants - not an enumerated type">
        <!-- This is part of the header boilerplate -->
    <enum value="256"        name="VK_MAX_PHYSICAL_DEVICE_NAME"/>
    <enum value="256"        name="VK_MAX_EXTENSION_NAME"/>
    ...
\end{verbatim}

The third \tag{require} block starts pulling in some Vulkan commands. The
first command corresponds to the following definition found in the
\tag{commands} section of \code{vk.xml}:

\begin{verbatim}
<commands>
    <command>
        <proto><type>VkResult</type> <name>vkCreateInstance</name></proto>
        <param>const <type>VkInstanceCreateInfo</type>* <name>pCreateInfo</name></param>
        <param><type>VkInstance</type>* <name>pInstance</name></param>
    </command>
    ...
\end{verbatim}

In turn, the \tag{command} tag requires the \tag{type}s \code{VkResult},
\code{VkInstanceCreateInfo}, and \code{VkInstance} as part of its
definition. The definitions of these types are determined as follows:

For \code{VkResult}, the corresponding required \tag{type} is:

\begin{verbatim}
<type name="VkResult" category="enum"/>
\end{verbatim}

\noindent Since this is an enumeration type, it simply links to an
\tag{enums} tag with the same name:

\begin{verbatim}
<enums name="VkResult" type="enum" expand="VK_RESULT" comment="Error and return codes">
    <!-- Return codes for successful operation execution (positive values) -->
    <enum value="0"     name="VK_SUCCESS"/>
    <enum value="1"     name="VK_UNSUPPORTED"/>
    <enum value="2"     name="VK_NOT_READY"/>
    ...
\end{verbatim}

For \tag{VkInstanceCreateInfo}, the required \tag{type} is:

\begin{verbatim}
<type category="struct" name="VkInstanceCreateInfo">
    <member><type>VkStructureType</type>        <name>sType</name></member>                          <!-- Should be VK_STRUCTURE_TYPE_INSTANCE_CREATE_INFO -->
    <member>const void*                         <name>pNext</name></member>                          <!-- Pointer to next structure -->
    <member>const <type>VkApplicationInfo</type>* <name>pAppInfo</name></member>
    <member>const <type>VkAllocCallbacks</type>* <name>pAllocCb</name></member>
    <member><type>uint32_t</type>               <name>extensionCount</name></member>
    <member>const <type>char</type>*const*      <name>ppEnabledExtensionNames</name></member>        <!-- layer or extension name to be enabled -->
</type>
\end{verbatim}

\noindent This is a structure type, defining a C \code{struct} with all the
members defined in each \tag{member} tag in order. In addition, it requires
some other types, whose definitions are located by name in exactly the same
fashion.

For the final direct dependency of the command, \code{VkInstance}, the
required \tag{type} is:

\begin{verbatim}
<!-- Types which can be void pointers or class pointers, selected at compile time -->
<type>VK_DEFINE_BASE_HANDLE(<name>VkObject</name>)</type>
<type>VK_DEFINE_DISP_SUBCLASS_HANDLE(<name>VkInstance</name>, <type>VkObject</type>)</type>
\end{verbatim}

\noindent In this case, the type \code{VkInstance} is defined by a special
compile-time macro which defines it as a derived class of \code{VkObject}
(for C++) or a less typesafe definition for (for C). This macro isn't part
of the type dependency analysis, just the boilerplate used in the header.

If these are the only \tag{feature} dependencies in \code{vk.xml}, the
resulting \code{vulkan.h} header will look like this:

\begin{verbatim}

#ifndef __vulkan_h_
#define __vulkan_h_ 1

#ifdef __cplusplus
extern "C" {
#endif

/*
** Copyright (c) 2015-2016 The Khronos Group Inc.
    ...
*/

/*
** This header is generated from the Khronos Vulkan XML API Registry.
**
** Generated on date 20160208
*/


#define VK_VERSION_1_0 1
#include "vk_platform.h"
#define VK_MAKE_VERSION(major, minor, patch) \
    ((major << 22) | (minor << 12) | patch)

// Vulkan API version supported by this file
#define VK_API_VERSION VK_MAKE_VERSION(0, 104, 0)

#if defined (__cplusplus) && (VK_UINTPTRLEAST64_MAX == UINTPTR_MAX)
    #define VK_TYPE_SAFE_COMPATIBLE_HANDLES 1
#endif

#if defined(VK_TYPE_SAFE_COMPATIBLE_HANDLES) && !defined(VK_DISABLE_TYPE_SAFE_HANDLES)
    #define VK_DEFINE_PTR_HANDLE(_obj) struct _obj##_T { char _dummy; }; typedef _obj##_T* _obj;
    #define VK_DEFINE_PTR_SUBCLASS_HANDLE(_obj, _base) struct _obj##_T : public _base##_T {}; typedef _obj##_T* _obj;

    #define VK_DEFINE_BASE_HANDLE(_obj) VK_DEFINE_PTR_HANDLE(_obj)
    #define VK_DEFINE_DISP_SUBCLASS_HANDLE(_obj, _base) VK_DEFINE_PTR_SUBCLASS_HANDLE(_obj, _base)
    #define VK_DEFINE_NONDISP_SUBCLASS_HANDLE(_obj, _base) VK_DEFINE_PTR_SUBCLASS_HANDLE(_obj, _base)
#else
    #define VK_DEFINE_BASE_HANDLE(_obj) typedef VkUintPtrLeast64 _obj;
    #define VK_DEFINE_DISP_SUBCLASS_HANDLE(_obj, _base) typedef uintptr_t _obj;
    #define VK_DEFINE_NONDISP_SUBCLASS_HANDLE(_obj, _base) typedef VkUintPtrLeast64 _obj;
#endif

typedef enum {
    VK_SUCCESS = 0,
    VK_UNSUPPORTED = 1,
    VK_NOT_READY = 2,
    ...
} VkResult;
typedef enum {
    VK_STRUCTURE_TYPE_APPLICATION_INFO = 0,
    ...
} VKStructureType;
typedef struct {
    VkStructureType                             sType;
    const void*                                 pNext;
    const char*                                 pAppName;
    uint32_t                                    appVersion;
    const char*                                 pEngineName;
    uint32_t                                    engineVersion;
    uint32_t                                    apiVersion;
} VkApplicationInfo;
typedef enum {
    VK_SYSTEM_ALLOC_TYPE_API_OBJECT = 0,
    ...
} VkSystemAllocType;
typedef void* (VKAPI_PTR *PFN_vkAllocFunction)(
    void*                           pUserData,
    size_t                          size,
    size_t                          alignment,
    VkSystemAllocType               allocType);
typedef void (VKAPI_PTR *PFN_vkFreeFunction)(
    void*                           pUserData,
    void*                           pMem);
typedef struct {
    void*                                       pUserData;
    PFN_vkAllocFunction                         pfnAlloc;
    PFN_vkFreeFunction                          pfnFree;
} VkAllocCallbacks;
typedef struct {
    VkStructureType                             sType;
    const void*                                 pNext;
    const VkApplicationInfo*                    pAppInfo;
    const VkAllocCallbacks*                     pAllocCb;
    uint32_t                                    extensionCount;
    const char*const*                           ppEnabledExtensionNames;
} VkInstanceCreateInfo;
VK_DEFINE_BASE_HANDLE(VkObject)
VK_DEFINE_DISP_SUBCLASS_HANDLE(VkInstance, VkObject)
#define VK_MAX_PHYSICAL_DEVICE_NAME       256
#define VK_MAX_EXTENSION_NAME             256
typedef VkResult (VKAPI_PTR *PFN_vkCreateInstance)(const VkInstanceCreateInfo* pCreateInfo, VkInstance* pInstance);
#ifdef VK_PROTOTYPES
VKAPI_ATTR VkResult VKAPI_CALL vkCreateInstance(
    const VkInstanceCreateInfo*                 pCreateInfo,
    VkInstance*                                 pInstance);
#endif

#ifdef __cplusplus
}
#endif

#endif
\end{verbatim}

Note that several additional types are pulled in by the type dependency
analysis, but only those types, commands, and tokens required by the
specified features are generated.


\subsection{How To Add A Compile-Time Constant}

Go to the \tag{feature} tag and search for the nested block labelled

\begin{verbatim}
<require comment="API constants">
\end{verbatim}

In this block, add an (appropriately indented) tag like

\begin{verbatim}
    <enum name="VK_THE_ANSWER"/>
\end{verbatim}

Then go to the \tag{enums} block labelled

\begin{verbatim}
<enums comment="Misc. hardcoded constants - not an enumerated type">
\end{verbatim}

In this block, add a tag whose \attr{name} attribute matches the \attr{name}
you defined above and whose \attr{value} attribute is the value to give the
constant:

\begin{verbatim}
    <enum value="42"    name="VK_THE_ANSWER"/>
\end{verbatim}


\subsection{How To Add A Struct or Union Type}

For this example, assume we want to define a type
corresponding to a C \code{struct} defined as follows:

\begin{verbatim}
typedef struct {
    VkStructureType          sType;
    const void*              pNext;
    const VkApplicationInfo* pAppInfo;
    const VkAllocCallbacks*  pAllocCb;
    uint32_t                 extensionCount;
    const char*const*        ppEnabledExtensionNames;
} VkInstanceCreateInfo;
\end{verbatim}

If \code{VkInstanceCreateInfo} is the type of a parameter of a command in
the API, make sure that command's definition (see below for how to add a
command) puts \code{VkInstanceCreateInfo} in nested \tag{type} tags where
it's used.

Otherwise, if the struct type is not used directly by a command in the API,
nor required by a chain of type dependencies for other commands, an explicit
\tag{type} dependency should be added to the \tag{feature} tag. Go to the
\tag{types} tag and search for the nested block labelled

\begin{verbatim}
<require comment="API types not used by commands">
    ...
\end{verbatim}

In this block, add a tag whose \attr{name} attribute matches the \attr{name}
of the struct type being defined:

\begin{verbatim}
<require comment="API types not used by commands">
    <type name="VkInstanceCreateInfo"/>
    ...
\end{verbatim}

Then go to the \tag{types} tag and add a new \tag{type} tag defining
the struct names and members, somewhere below
the corresponding comment, like this:

\begin{verbatim}
<types>
    ...
    <!-- Struct types -->
        <type category="struct" name="VkInstanceCreateInfo">
            <member><type>VkStructureType</type>
                    <name>sType</name></member>
            <member>const void*
                    <name>pNext</name></member>
            <member>const <type>VkApplicationInfo</type>*
                    <name>pAppInfo</name></member>
            <member>const <type>VkAllocCallbacks</type>*
                    <name>pAllocCb</name></member>
            <member><type>uint32_t</type>
                    <name>extensionCount</name></member>
            <member>const <type>char</type>*const*
                    <name>ppEnabledExtensionNames</name></member>
        </type>
        ...
\end{verbatim}

If any of the member types are types also defined in the header, make sure
to enclose those type names in nested \tag{type} tags, as shown above. Basic
C types should not be tagged.

If the type is a C \code{union}, rather than a \code{struct}, then set the
value of the \attr{category} attribute to \code{"union"} instead of
\code{"struct"}.


\subsection{How To Add An Enumerated Type}

For this example, assume we want to define a type corresponding to a C
\code{enum} defined as follows:

\begin{verbatim}
typedef enum {
    VK_DEVICE_CREATE_VALIDATION_BIT = 0x00000001,
    VK_DEVICE_CREATE_MULTI_DEVICE_IQ_MATCH_BIT = 0x00000002;
} VkDeviceCreateFlagBits.
\end{verbatim}

If \code{VkDeviceCreateFlagBits} is the type of a parameter to a command in
the API, or of a member in a structure or union, make sure that command
parameter or struct member's definition puts \code{VkDeviceCreateFlagBits}
in nested \tag{type} tags where it's used.

Otherwise, if the enumerated type is not used directly by a command in the
API, nor required by a chain of type dependencies for commands and structs,
an explicit \tag{type} dependency should be added to the \tag{feature} tag
in exactly the same fashion as described above for \code{struct} types.

Next, go to the line labelled

\begin{verbatim}
<!-- SECTION: Vulkan enumerant (token) definitions. -->
\end{verbatim}

Below this line, add an \tag{enums} tag whose \attr{name} attribute matches
the \tag{type} name \code{VkDeviceCreateFlagBits}, and whose contents
correspond to the individual fields of the enumerated type:

\begin{verbatim}
<enums name="VkDeviceCreateFlagBits" type="bitmask">
    <enum bitpos="0" name="VK_DEVICE_CREATE_VALIDATION_BIT"/>
    <enum bitpos="1" name="VK_DEVICE_CREATE_MULTI_DEVICE_IQ_MATCH_BIT"/>
</enums>
\end{verbatim}

Several other attributes of the \tag{enums} tag can be set. In this case,
the \attr{type} attribute is set to \code{"bitmask"}, indicating that the
individual enumerants represent elements of a bitmask.

The individual \tag{enum} tags define the enumerants, just like the
definition for compile-time constants described above. In this case, because
the enumerants are bitfields, their values are specified using the
\attr{bitpos} attribute. The value of this attribute must be an integer in
the range $[0,31]$ specifying a single bit number, and the resulting value
is printed as a hexadecimal constant corresponding to that bit.

It is also possible to specify enumerant values using the \attr{value}
attribute, in which case the specified numeric value is passed through to
the C header unchanged.


\subsection{How to Add A Command}

For this example, assume we want to define the command:

\begin{verbatim}
VKAPI_ATTR VkResult VKAPI_CALL vkCreateInstance(
    const VkInstanceCreateInfo*                 pCreateInfo,
    VkInstance*                                 pInstance);
\end{verbatim}

Commands must always be explicitly required in the \tag{feature} tag. In
that tag, you can use an existing \tag{require} block including API features
which the new command should be grouped with, or define a new block. For
this example, add a new block, and require the command by using the
\tag{command} tag inside that block:

\begin{verbatim}
<!-- SECTION: Vulkan API interface definitions -->
<feature api="vulkan" name="VK_VERSION_1_0" number="1.0">
    ...
    <require comment="Device initialization">
        <command name="vkCreateInstance"/>
    </require>
    ...
</feature>
\end{verbatim}

The \tag{require} block may include a \attr{comment} attribute whose value
is a descriptive comment of the contents required within that block. The
comment is not currently used in header generation, but might be in the
future, so use comments which are polite and meaningful to users of
\code{vulkan.h}.

Then go to the \tag{commands} tag and add a new \tag{command} tag defining
the command, preferably sorted into alphabetic order with other commands for
ease of reading, as follows:

\begin{verbatim}
<!-- SECTION: Vulkan command definitions -->
<commands>
    ...
    <command>
        <proto><type>VkResult</type>
               <name>vkCreateInstance</name></proto>
        <param>const <type>VkInstanceCreateInfo</type>*
               <name>pCreateInfo</name></param>
        <param><type>VkInstance</type>*
               <name>pInstance</name></param>
    </command>
    ...
</commands>
\end{verbatim}

The \tag{proto} tag defines the return type and function name of the
command. The \tag{param} tags define the command's parameters in the order
in which they're passed, including the parameter type and name. The contents
are laid out in the same way as the structure \tag{member} tags described
previously.


\subsection{More Complicated API Representations}

The registry schema can represent a good deal of additional information, for
example by creating multiple \tag{feature} tags defining different API
versions and extensions. This capability is not yet relevant to Vulkan.
Those capabilities will be documented as they are needed.


\subsection{More Complicated Output Formats And Other Languages}

The registry schema is oriented towards C-language APIs. Types and commands
are defined using syntax which is a subset of C, especially for structure
members and command parameters. It would be possible to use a
language-independent syntax for representing such information, but since we
are writing a C API, any such representation would have to be converted into
C anyway at some stage.

The \code{vulkan.h} header is written using an {\bf output generator} object
in the Python scripts. This output generator is specialized for C, but the
design of the scripts is intended to support writing output generators for
other languages as well as purposes such as documentation (e.g. generating
Asciidoc fragments corresponding to types and commands for use in the API
specification and reference pages). When targeting other languages, the
amount of parsing required to convert type declarations into other languages
is small. However, it will probably be necessary to modify some of the
boilerplate C text, or specialize the tags by language, to support such
generators.


\subsection{Additional Semantic Tagging}

The schema is being extended to support semantic tags describing
various properties of API features, such as:

\begin{itemize}
\item constraints on allowed scalar values to function parameters (non-NULL,
      normalized floating-point, etc.)
\item length of arrays corresponding to function pointer parameters
\item miscellaneous properties of commands such as whether the application
      or system is responsible for threadsafe use; which queues they may be
      issued on; whether they are aliases or otherwise related to other
      commands; etc.
\end{itemize}

These tags will be used by other tools for purposes such as helping create
validation layers, generating serialization code, and so on. We'd like to
eventually represent everything about the API that is amenable to automatic
processing within the registry schema. Please make suggestions on the Gitlab
issue tracker.


\subsection{Stability of the XML Database and Schema}
\label{general:stability}

The Vulkan XML schema is evolving in response to corresponding
changes in the Vulkan API and ecosystem. Most such change will probably be
confined to adding attributes to existing tags and properly expressing the
relationships to them, and making API changes corresponding to accepted
feature requests. Changes to the schema should be described in the change
log of this document (see section~\ref{changelog}). Changes to the
\code{.xml} files and Python scripts are logged in Gitlab history.


\section{Change Log}
\label{changelog}

\begin{itemize}
\item 2016/02/07 - Add \attr{vendorids} tags for Khronos vendor IDs.
\item 2015/12/10 - Add \attr{author} and \attr{contact} attributes for
      \tag{extension} tags.
\item 2015/12/07 - Move \code{vulkan/vulkan.h} to a subdirectory.
\item 2015/12/01 - Add \tag{tags} tags for author tags.
\item 2015/11/18 - Bring documentation and schema up to date for extension
      enumerants.
\item 2015/11/02 - Bring documentation and schema up to date with several
      recent merges, including \tag{validity} tags. Still out of date WRT
      extension enumerants, but that will change soon.
\item 2015/09/08 - Rename {\tt threadsafe} attribute to \attr{externsync},
      and {\tt implicitunsafeparams} tag to \attr{implicitexternsync}.
\item 2015/09/07 - Update \tag{command} tag description to remove the
      \attr{threadsafe} attribute and replace it with a combination of
      \attr{threadunsafe} attributes on individual parameters, and
      \tag{implicitunsafeparams} tags describing additional unsafe objects
      for the command.
\item 2015/08/04 - Add \code{basetype} and \code{funcpointer}
      \attr{category} values for type tags, and explain the intended use and
      order in which types in each category are emitted.
\item 2015/07/02 - Update description of Makefile targets. Add
      descriptions of \attr{threadsafe}, \attr{queues}, and
      \attr{renderpass} attributes of \tag{command} tags in
      section~\ref{tag:command}, and of \attr{modified} attributes of
      \tag{param} tags in section~\ref{tag:command:param}
\item 2015/06/17 - add descriptions of allowed \attr{category} attribute
      values of \tag{type} tags, used to group and sort related categories
      of declarations together in the generated header.
\item 2015/06/04 - Add examples of making changes and additions to the
      registry in appendix~\ref{examples}.
\item 2015/06/03 - Move location to new ``vulkan'' Git repository. Add
      definition of \tag{type} tags for C struct/unions. Start adding
      examples of making changes in appendix~\ref{examples}.
\item 2015/06/02 - Branch from OpenGL specfile documentation and bring up to
      date with current Vulkan schema.
\end{itemize}

% % Copyright (c) 2013-2016 The Khronos Group Inc.
%
% Permission is hereby granted, free of charge, to any person obtaining a
% copy of this software and/or associated documentation files (the
% "Materials"), to deal in the Materials without restriction, including
% without limitation the rights to use, copy, modify, merge, publish,
% distribute, sublicense, and/or sell copies of the Materials, and to
% permit persons to whom the Materials are furnished to do so, subject to
% the following conditions:
%
% The above copyright notice and this permission notice shall be included
% in all copies or substantial portions of the Materials.
%
% THE MATERIALS ARE PROVIDED "AS IS", WITHOUT WARRANTY OF ANY KIND,
% EXPRESS OR IMPLIED, INCLUDING BUT NOT LIMITED TO THE WARRANTIES OF
% MERCHANTABILITY, FITNESS FOR A PARTICULAR PURPOSE AND NONINFRINGEMENT.
% IN NO EVENT SHALL THE AUTHORS OR COPYRIGHT HOLDERS BE LIABLE FOR ANY
% CLAIM, DAMAGES OR OTHER LIABILITY, WHETHER IN AN ACTION OF CONTRACT,
% TORT OR OTHERWISE, ARISING FROM, OUT OF OR IN CONNECTION WITH THE
% MATERIALS OR THE USE OR OTHER DEALINGS IN THE MATERIALS.

\documentclass{article}

% Various packages of possible use. {index} creates problems for some reason.
% \usepackage{index}
\def\specpdftitle{The Khronos Vulkan API Registry}
\usepackage[pdftex,bookmarksnumbered=true,linktocpage,plainpages=false,pdftitle={\specpdftitle},colorlinks=true]{hyperref}
\usepackage{amsmath,enumerate,epsfig,framed,mdframed,multicol,longtable,times,url}
\usepackage[normalem]{ulem}
% underscore allows use of _ instead of \_ in text. [strings] allows _ in file names
\usepackage[strings]{underscore}
\usepackage[breakwords,fit]{truncate}

% Index everything down through \paragraph
\setcounter{tocdepth}{5}

\makeatletter
\makeatother
\makeindex

% Some commonly used abbreviations

\def\code#1{{\tt #1}}
\def\tag#1{{\tt <#1>\index{<#1>}}}
% Because I keep mistyping \tag
\def\arg#1\tag{#1}
\def\attr#1{{\tt #1\index{#1}}}

\begin{document}

\title{The Khronos Vulkan API Registry for Vulkan}
\author{Jon Leech}
\date{Last updated 2016/02/07}
\maketitle

\begin{abstract}

This document describes the Khronos Vulkan API Registry schema, and provides
some additional information about using the registry and scripts to generate
a variety of outputs including C header files and several types of asciidoc
include files used in the Vulkan API specification and reference pages. The
underlying XML files and scripts are located on the Khronos Gitlab server,
currently in the {\em master} branch at URL

\begin{center}
{\bf \href{https://gitlab.khronos.org/vulkan/vulkan/tree/master/src/spec/}
          {https://gitlab.khronos.org/vulkan/vulkan/tree/master/src/spec/}}
\end{center}

\end{abstract}

\tableofcontents

\section{Introduction}

The registry uses an XML representation of the Vulkan API and a set of
Python scripts to manipulate the XML, based on the lxml Python bindings. It
includes an XML schema and validator.

The registry schema is based on, but not identical to the schema used for
the previously published OpenGL, OpenGL ES and EGL API registries. The
schema was extended to represent additional types and concepts not needed
for those APIS, such as structure and enumerant types.

The processed C header file corresponding to the registry is checked in
under src/vulkan/vulkan.h .


\section{Getting Started}
\label{starting}

You will need some tools installed.
For Linux systems, you will need the following packages installed (Debian
package names are noted, Ubuntu package names are probably the same).

\begin{itemize}
\item Python 3.x (\code{python3})
\item The lxml Python package (\code{python3-lxml})
\item Libxml (\code{libxml2})
\item A make tool, such as GNU make
\end{itemize}

The toolchain has also been run successfully on Microsoft Windows in the
Cygwin environment, and the individual Python-based components of the
toolchain can be run directly from the Windows command line. More details on
this are under src/spec/README in Git.

Once you have the right tools installed, perform the following steps:

\begin{itemize}
\item Check out the ``vulkan'' repository linked above from Khronos Gitlab
      (there are instructions at the link)
\item Go to \code{vulkan/src/spec} in your checked-out repo
\item Invoke \code{make clobber ; make install}
\end{itemize}

This should regenerate \code{vulkan.h} and install it in
\code{../vulkan/vulkan.h}. The result should be identical to the
version you just pulled from Gitlab (use \code{git diff
../vulkan/vulkan.h} to check).

Other Makefile targets include:

\begin{itemize}
\item \code{vulkan-docs} - regenerate the API specification and reference
      page asciidoc include files in \code{../doc/specs/vulkan/} under the
      subdirectories \code{enums}, \code{flags}, \code{protos}, and
      \code{structs}. These files are pulled in by the ref pages and API
      spec to put function prototypes and struct and enum declarations in
      documentation.

      You should probably {\bf not} update these files if you are proposing
      an API change - just the \code{vulkan.h} header. When a git merge is
      accepted including your change, the docs will be updated at that time.
      Including the dozens of include files in your merge makes it hard to
      read and adds no information beyond the header updates. Sometimes it
      is useful to include just one or two of the include files for context
      of a global change, however.

      However, you are strongly encouraged to updated the {\bf
      non-}autogenerated docs under \code{../doc/specs/vulkan/man/*.txt} to
      reflect API changes.
\item \code{full_install} - equivalent to \code{install} followed by
      \code{vulkan-docs}.
\item \code{validate} - validate \code{vk.xml} against the XML schema.
      Recommended if you're making nontrivial changes.
%% \item \code{vk.json} - generate JSON version of \code{vk.xml}. Currently the
%%       converter is generating syntactically but not semantically legal JSON,
%%       and it's difficult to map the XML schemaanso don't use this target yet.
\item \code{readme.pdf} - regenerate this document from the LaTeX source.
      Most people will never need to do this. If you do, you must have
      pdflatex installed, preferably from the TeTeX distribution.
\end{itemize}

If you just want to modify the API, changing \code{vk.xml} and running
``make'' should be all that's needed. See appendix~\ref{examples} for some
examples of modifying the XML.

If you want to repurpose the registry for reasons other than header file
and ref page include generation, or to generate headers for languages
other than C, start with the Makefile rules and inspect the files
\code{vk.xml}, \code{genvk.py}, \code{reg.py}, and \code{generator.py}.

If you're using other platforms, merge requests with additional
documentation on using the tools on those platforms would be very helpful.

\subsection{Header Generation Script - \code{genvk.py}}

When generating header files using the \code{genvk.py} script, an API name
and profile name are required, as shown in the Makefile examples.
Additionally, specific API versions and extensions can be required or
excluded. Based on this information, the generator script extracts the
relevant interfaces and creates a C-language header file for them.
\code{genvk.py} contains predefined generator options for the current
development version of Vulkan 1.0.

The generator script is intended to be generalizable to other languages by
writing new generator classes. Such generators would have to rewrite the C
types and definitions in the XML to something appropriate to their language.

\subsection{Registry Processing Script - \code{reg.py}}

XML processing is done in \code{reg.py}, which contains several objects and
methods for loading registries and extracting interfaces and extensions for
use in header generation. There is some internal documentation in the form
of comments, although nothing more extensive exists yet.

\subsection{Output Generator Script - \code{generator.py}}

Once the registry is loaded, the \code{COutputGenerator} class defined in
\code{generator.py} is used to create a header file. The
\code{DocOutputGenerator} class is used to create the asciidoc include
files. Output generators for other purposes can be added as needed.


\section{Vulkan Registry Schema}
\label{schema}

The format of the Vulkan registry is a top level \tag{registry} tag
containing \tag{types}, \tag{enums}, \tag{commands}, \tag{feature}, and
\tag{extension} tags describing the different elements of an API, as
explained below. This description corresponds to a formal Relax NG schema
file, \code{registry.rnc}, against which the XML registry files can be
validated.

At present the only registry in this schema is the core Vulkan API registry,
\code{vk.xml}.


\subsection{Profiles}
\label{schema:profile}

Types and enumerants can have different definitions depending on the API
profile requested. This capability is not used in the current Vulkan API but
may be in the future. Features and extensions can include some elements
conditionally depending on the API profile requested.


\subsection{API Names}
\label{schema:apiname}

The schema supports, but does not currently use an \attr{api} attribute on
several tags. This is an arbitrary string, specified at header generation
time, for labelling properties of a specific API or API profile. The string
can be, but is not necessarily, an actual API name. Names starting with ``vk''
are suggested if and when we start defining profiles of Vulkan.


\section{Registry Root (\tag{registry} tag)}
\label{schema:root}

A \tag{registry} contains the entire definition of one or more related
APIs.

\subsection{Attributes of \tag{registry} tags}

None.

\subsection{Contents of \tag{registry} tags}

Zero or more of each of the following tags, normally in this order
(although order shouldn't be important):

\begin{itemize}
\item \tag{comment} - Contains arbitrary text, such as a copyright
      statement.
\item \tag{vendorids} (see section~\ref{tag:vendorids}) - defines Khronos
      vendor IDs, described in detail in the ``Layers and Extensions''
      appendix of the Vulkan Specification.
\item \tag{tags} (see section~\ref{tag:tags}) - defines author prefixes used
      for extensions and layers. Prefixes are described in detail in the
      ``Layers and Extensions'' appendix of the Vulkan Specification.
\item \tag{types} (see section~\ref{tag:types}) - defines API types.
      Usually only one tag is used.
\item \tag{enums} (see section~\ref{tag:enums}) - defines API token names
      and values. Usually multiple tags are used. Related groups may be
      tagged as an enumerated type corresponding to a \tag{type} tag, and
      resulting in a C \code{enum} declaration. This ability is heavily used
      in the Vulkan API.
\item \tag{commands} (see section~\ref{tag:commands}) - defines API
      commands (functions). Usually only one tag is used.
\item \tag{feature} (see section~\ref{tag:feature}) - defines API
      feature interfaces (API versions, more or less). One tag per
      feature set.
\item \tag{extensions} (see section~\ref{tag:extensions}) - defines API
      extension interfaces. Usually only one tag is used, wrapping many
      extensions.
\end{itemize}


\section{Vendor IDs (\tag{vendorids} tag)}
\label{tag:vendorids}

The \tag{vendorids} tag contains individual \tag{vendorid} tags defining
vendor IDs for physical devices which do not have PCI vendor IDs.

Each \tag{vendorid} tag contains information defining a single vendor ID.

\subsection{Attributes of \tag{vendorid} tags}

\begin{itemize}
\item \attr{name} - required. The author prefix, as registered with Khronos.
      This must match an author prefix in the \attr{name} field of a
      \tag{tag} tag.
\item \attr{id} - required. The reserved vendor ID, as a hexadecimal number.
\item \attr{comment} - optional. Arbitrary string (unused).
\end{itemize}


\section{Author Prefixes (\tag{tags} tag)}
\label{tag:tags}

The \tag{tags} tag contains individual \tag{tag} tags defining each of the
reserved author prefixes used by extension and layer authors.

Each \tag{tag} tag contains information defining a single author prefix.

\subsection{Attributes of \tag{tag} tags}

\begin{itemize}
\item \attr{name} - required. The author prefix, as registered with Khronos.
      A short, upper-case string, usually an abbreviation of an author,
      project or company name.
\item \attr{author} - required. The author name, such as a full company
      or project name.
\item \attr{contact} - required. The contact who registered or is currently
      responsible for extensions and layers using the prefix, including
      sufficient contact information to reach the contact such as individual
      name together with email address, Github username, or other contact
      information.
\end{itemize}


\section{API types (\tag{types} tag)}
\label{tag:types}

The \tag{types} tag contains individual \tag{type} tags describing each
of the derived types used in the API.

Each \tag{type} tag contains information which can be used to generate C
code corresponding to the type. In many cases, this is simply legal C code
with attributes or embedded tags denoting the type name and other types used
in defining this type. In some cases, additional attribute and embedded type
information is used to generate more complicated C types.


\subsection{Attributes of \tag{type} tags}

\begin{itemize}
\item \attr{requires} - optional. Another type name this type requires to
      complete its definition.
\item \attr{name} - optional. Name of this type (if not defined in the tag
      body).
\item \attr{api} - optional. An API name (see \tag{feature} below) which
      specializes this definition of the named type, so that the same API
      types may have different definitions for e.g. GL ES and GL. This is
      unlikely to be used in Vulkan, where a single API supports desktop and
      mobile devices, but the functionality is retained.
\item \attr{category} - optional. A string which indicates that this type
      contains a more complex structured definition. At present the only
      accepted categories are \code{basetype} \code{bitmask}, \code{define},
      \code{enum}, \code{funcpointer}, \code{group}, \code{handle},
      \code{include}, \code{struct}, and \code{union}, as described below.
\item \attr{comment} - optional. Arbitrary string (unused).
\item \attr{parent} only applicable if category is \code{handle}. Notes another type with
      the \code{handle} category that acts as a parent object for this type.
\item \attr{returnedonly} only applicable if category is \code{struct} or \code{union}.
      Notes that this struct/union is going to be filled in by the API, rather than an
      application filling it out and passing it to the API.
\end{itemize}

\subsection{Contents of \tag{type} tags}

The valid contents depend on the \attr{category} attribute.

\subsubsection{Enumerated types - \attr{category} "enum"}

If the \attr{category} tag has the value \code{enum}, the type is a C
enumeration. The body of the tag is ignored in this case. The value of the
\attr{name} attribute must be provided and must match the \attr{name}
attribute of a \tag{enums} tag (see section~\ref{tag:enums}). The enumerant
values defined within the \tag{enums} tag are used to generate a C
\code{enum} type declaration.

\subsubsection{Structure types - \attr{category} "struct" or "union"}

If the \attr{category} tag has the values \code{struct} or \code{union}, the
type is a C structure or union, respectively. In this case, the \attr{name}
attribute must be provided, and the contents of the \tag{type} tag are a
series of \tag{member} tags defining the members of the aggregate type, in
order, followed by an optional \tag{validity} tag including asciidoc
validation language for the structure contents.


\paragraph{Structure member (\tag{member}) tags}
\par\vskip 11pt

The \tag{member} tag defines the type and name of a structure or union
member.

\paragraph{Attributes of \tag{member} tags}

\begin{itemize}
\item \attr{len} - if the member is an array, len may be one or more of the following
           things, separated by commas (one for each array indirection): another
           member of that struct, ``null-terminated'' for a string, ``1'' to indicate it's
           just a pointer (used for nested pointers), or a latex equation (prefixed with
           ``latex:'')
\item \attr{externsync} - denotes that the member should be externally synchronized
           when accessed by Vulkan
\item \attr{optional} - whether this value can be omitted by providing \code{NULL} (for
          pointers), \code{VK_NULL_HANDLE} (for handles) or 0 (for bitmasks/values)
\item \attr{noautovalidity} - prevents automatic validity language being generated
          for the tagged item. Only suppresses item-specific validity - parenting issues
          etc. are still captured.
\end{itemize}

\paragraph{Contents of \tag{member} tags}
\par\vskip 11pt

The text elements of a \tag{member} tag, with all other tags removed,
is a legal C declaration of a struct or union member. In addition it may
contain two semantic tags:

\begin{itemize}
\item The \tag{type} tag is optional. It contains text which is a valid type
      name found in another \tag{type} tag, and indicates that this type
      must be previously defined for the definition of the command to
      succeed. Builtin C types should not be wrapper in \tag{type} tags.
\item The \tag{name} tag is required, and contains the struct/union member
      name being described.
\end{itemize}

\paragraph{Validation (\tag{validity}) tags}
\par\vskip 11pt

The \tag{validity} tag, if present defines valid use cases and values for
structure members.

\paragraph{Contents of \tag{validity} tags}
\par\vskip 11pt

Each \tag{validity} tag contains zero or more \tag{usage} tags. Each
\tag{usage} tag is intended to represent a specific validation requirement
for the structure and include arbitrary asciidoc text describing that
requirement.


\subsubsection{All other types}

If the \attr{category} attribute is one of \code{basetype}, \code{bitfield},
\code{define}, \code{funcpointer}, \code{group}, \code{handle} or
\code{include}, or is not specified, \tag{type} contains text which is legal
C code for a type declaration. It may also contain embedded tags:

\begin{itemize}
\item \tag{type} - nested type tags contain other type names which are
      required by the definition of this type.
\item \tag{apientry/} - insert a platform calling convention macro here
      during header generation, used mostly for function pointer types.
\item \tag{name} - contains the name of this type (if not defined in the
      tag attributes).
\end{itemize}

There is no restriction on which sorts of definitions may be made in a given
category, although the contents of tags with \attr{category} \code{enum},
\code{struct} or \code{union} are interpreted specially as described above.

However, when generating the header, types within each category are grouped
together, and categories are generated in the order given by the following
list. Therefore, types in a category should correspond to the intended
purpose given for that category. If this recommendation is not followed, it
is possible that the resulting header file will not compile due to
out-of-order type dependencies.

\begin{itemize}
  \item \code{include} ({\tt \#include} directives)
  \item \code{define} (macro {\tt \#define} directives)
  \item \code{basetype} (scalar typedefs, such as the definition of
        {\tt VkFlags})
  \item \code{handle} (invocations of macros defining scalar types such as
        {\tt VkInstance})
  \item \code{enum} (enumeration types and {\tt \#define} for constant values)
  \item \code{group} (currently unused)
  \item \code{bitmask} (enumeration types whose members are bitmasks)
  \item \code{funcpointer} (function pointer typedefs)
  \item \code{struct} and \code{union} together (struct and union types)
\end{itemize}


\subsection{Example of a \tag{types} tag}
\label{tag:types:example}

\begin{verbatim}
<types>
    <type name="stddef">#include &lt;stddef.h&gt;</type>
    <type requires="stddef">typedef ptrdiff_t <name>VKlongint</name>;</type>
    <type name="VkEnum" category="enum"/>
    <type category="struct" name="VkStruct">
        <member><type>VkEnum</type> <name>srcEnum</name></member>
        <member><type>VkEnum</type> <name>dstEnum</name></member>
    </type>
</types>

<enums name="VkEnum" type="enum">
    <enum value="0" name="VK_ENUM_ZERO"/>
    <enum value="42" name="VK_ENUM_FORTY_TWO"/>
</enums>
\end{verbatim}

The {\tt VkStruct} type is defined to require the types {\tt VkEnum} and
{\tt VKlongint} as well. If {\tt VkStruct} is in turn required by a command
or another type during header generation, it will result in the following
declarations:

\begin{verbatim}
#include <stddef.h>
typedef ptrdiff_t VKlongint.

typedef enum {
    VK_ENUM_ZERO = 0,
    VK_ENUM_FORTY_TWO = 42
} VkEnum;

typedef struct {
    VkEnum    dstEnum;
    VkLongint dstVal;
} VkStruct;
\end{verbatim}

Note that the angle brackets around \code{stddef.h} are represented as XML
entities in the registry. This could also be done using a CDATA block but
unless there are many characters requiring special representation in XML,
using entities is preferred.


\section{Enumerant Blocks (\tag{enums} tag)}
\label{tag:enums}

The \tag{enums} tags contain individual \tag{enum} tags describing each of
the token names used in the API. In some cases these correspond to a C
\code{enum}, and in some cases they are simply compile-time constants (e.g.
\code{\#define}).

\subsection{Attributes of \tag{enums} tags}

\begin{itemize}
\item \attr{name} - optional. String naming the C \code{enum} type whose
      members are defined by this enum group. If present, this attribute
      should match the \attr{name} attribute of a corresponding \tag{type}
      tag.
\item \attr{type} - optional. String describing the data type of the values
      of this group of enums. At present the only accepted categories are
      \code{enum} and \code{bitmask}, as described below.
\item \attr{start}, \attr{end} - optional. Integers defining the start and
      end of a reserved range of enumerants for a particular vendor or
      purpose. \attr{start} must be $\leq$ \attr{end}. These fields define
      formal enumerant allocations, and are made by the Khronos Registrar on
      request from implementers following the enum allocation policy.
\item \attr{vendor} - optional. String describing the vendor or purpose to
      whom a reserved range of enumerants is allocated.
\item \attr{comment} - optional. Arbitrary string (unused).
\end{itemize}

\subsection{Contents of \tag{enums} tags}

Each \tag{enums} block contains zero or more \tag{enum} and \tag{unused}
tags, in arbitrary order (although they are typically ordered by sorting
on enumerant values, to improve human readability).

\subsection{Example of \tag{enums} tags}

An example showing a tag with attribute \attr{type}="\code{enum}" is given
above in section~\ref{tag:types:example}. The following example is for
non-enumerated tokens.

\begin{verbatim}
<enums>
    <enum value="256" name="VK_MAX_EXTENSION_NAME"/>
    <enum value="MAX_FLOAT"  name="VK_LOD_CLAMP_NONE"/>
</enums>
\end{verbatim}

When processed into a C header, and assuming all these tokens were
required, this results in

\begin{verbatim}
#define VK_MAX_EXTENSION_NAME   256
#define VK_LOD_CLAMP_NONE       MAX_FLOAT
\end{verbatim}


\section{Enumerants (\tag{enum} tag)}
\label{tag:enum}

Each \tag{enum} tag defines a single Vulkan (or other API) token.

\subsection{Attributes of \tag{enum} tags}

\begin{itemize}
\item \attr{value} or \attr{bitpos} - exactly one of these is allowed and
      required. \attr{value} is an enumerant value in the form of a legal C
      constant (usually a literal decimal or hexadecimal integer, though
      arbitrary strings are allowed). \attr{bitpos} is a literal integer bit
      position in a bitmask.
\item \attr{name} - required. Enumerant name, a legal C preprocessor token
      name.
\item \attr{api} - optional. An API name which specializes this definition of
      the named enum, so that different APIs may have different values for
      the same token. May be used to address a subtle incompatibilities.
\item \attr{type} - optional. Used only when \attr{value} is specified. C
      suffix for the value to force it to a specific type. Currently only
      \code{u} and \code{ull} are used, for \code{unsigned} 32- and 64-bit
      integer values, respectively. Separated from \attr{value} since this
      eases parsing and sorting of values, and rarely used.
\item \attr{alias} - optional. Name of another enumerant this is an alias
      of, used where token names have been changed as a result of profile
      changes or for consistency purposes. An enumerant alias is simply a
      different \attr{name} for the exact same \attr{value} or
      \attr{bitpos}.
\end{itemize}

\subsection{Contents of \tag{enum} tags}

\tag{enum} tags have no allowed contents. All information is contained
in the attributes.


\section{Unused Enumerants (\tag{unused} tag)}
\label{tag:unused}

Each \tag{unused} tag defines a range of enumerants which is allocated, but
not yet assigned to specific enums. This just tracks the unused values for
the Registrar's use, and is not used for header generation.

\subsection{Attributes of \tag{unused} tags}

\begin{itemize}
\item \attr{start} - required, \attr{end} - optional. Integers defining the
      start and end of an unused range of enumerants. \attr{start} must be
      $\leq$ \attr{end}. If \attr{end} is not present, then \attr{start}
      defines a single unused enumerant. This range should not exceed the
      range reserved by the surrounding \tag{enums} tag.
\item \attr{vendor} - optional. String describing the vendor or purposes to
      whom a reserved range of enumerants is allocated. Usually identical to
      the \attr{vendor} attribute of the surrounding \attr{enums} block.
\item \attr{comment} - optional. Arbitrary string (unused).
\end{itemize}

\subsection{Contents of \tag{unused} tags}

None.


\section{Command Blocks (\tag{commands} tag)}
\label{tag:commands}

The \tag{commands} tag contains definitions of each of the functions
(commands) used in the API.

\subsection{Attributes of \tag{commands} tags}

None.

\subsection{Contents of \tag{commands} tags}

Each \tag{commands} block contains zero or more \tag{command} tags, in
arbitrary order (although they are typically ordered by sorting on the
command name, to improve human readability).


\section{Commands (\tag{command} tag)}
\label{tag:command}

The \tag{command} tag contains a structured definition of a single API
command (function).

\subsection{Attributes of \tag{command} tags}

\begin{itemize}
\item \attr{queues} - optional. A string identifying the command queues
      this command can be placed on. The format of the string is one or
      more of the terms \code{"compute"}, \code{"dma"}, and
      \code{"graphics"}, with multiple terms separated by commas
      (\code{","}).
\item \attr{renderpass} - optional. A string identifying whether the
      command can be issued only inside a render pass (\code{"inside"}),
      only outside a render pass (\code{"outside"}), or both
      (\code{"both"}).
\item \attr{cmdbufferlevel} - optional. A string identifying the command buffer
      levels that this command can be called by. The format of the string is
      one or more of the terms \code{"primary"} and \code{"secondary"}, with
      multiple terms separated by commas (\code{","}).
\item \attr{comment} - optional. Arbitrary string (unused).
\end{itemize}

\subsection{Contents of \tag{command} tags}

\begin{itemize}
\item \tag{proto} is required and must be the first element. It is a tag
      defining the C function prototype of a command as described below, up
      to the function name and return type but not including function
      parameters.
\item \tag{param} elements for each command parameter follow, defining
      its name and type, as described below. If a command takes no
      arguments, it has no \tag{param} tags.
\item An optional \tag{validity} tag including asciidoc validation language
      for the command parameters.
\end{itemize}

Following these elements, the remaining elements in a \tag{command}
tag are optional and may be in any order:

\begin{itemize}
\item \tag{alias} - optional. Has no attributes and contains a string which
      is the name of another command this command is an alias of, used when
      promoting a function from vendor to Khronos extension or Khronos
      extension to core API status. A command alias describes the case where
      there are two function names which resolve to the {\bf same} entry
      point in the underlying layer stack.
\item \tag{description} - optional. Unused text.
\item \tag{implicitexternsyncparams} - optional. Contains a list of
      \tag{param} tags, each containing Asciidoc source text describing an
      object which is not a parameter of the command, but is related to one,
      and which also requires external synchronization as described in
      section~\ref{tag:command:param:attr}. The text is intended to be
      incorporated into the API specification.
\end{itemize}


\subsection{Command prototype (\tag{proto} tags)}
\label{tag:command:proto}

The \tag{proto} tag defines the return type and name of a command.

\subsubsection{Attributes of \tag{proto} tags}

None.

% \begin{itemize}
% \item \attr{group} - group name, an arbitrary string.
% \end{itemize}
%
% If the group name is defined, it may be interpreted as described in
% section~\ref{tag:group:meaning}.

\subsubsection{Contents of \tag{proto} tags}

The text elements of a \tag{proto} tag, with all other tags removed, is
legal C code describing the return type and name of a command. In addition
to text, it may contain two semantic tags:

\begin{itemize}
\item The \tag{type} tag is optional, and contains text which is a valid
      type name found in a \tag{type} tag. It indicates that this type must
      be previously defined for the definition of the command to succeed.
      Builtin C types, and any derived types which are expected to be found
      in other header files, should not be wrapped in \tag{type} tags.
\item The \tag{name} tag is required, and contains the command name
      being described.
\end{itemize}

\subsection{Command parameter (\tag{param} tags)}
\label{tag:command:param}

The \tag{param} tag defines the type and name of a parameter. Its contents
are very similar to the \tag{member} tag used to define struct and union
members.

\subsubsection{Attributes of \tag{param} tags}
\label{tag:command:param:attr}

\begin{itemize}
\item \attr{len} - if the param is an array, len may be one or more of the following
           things, separated by commas (one for each array indirection): another
           param of that command, ``null-terminated'' for a string, ``1'' to indicate it's
           just a pointer (used for nested pointers), or a latex equation (prefixed with
           ``latex:'')
\item \attr{optional} - whether this value can be omitted by providing \code{NULL} (for
          pointers), \code{VK_NULL_HANDLE} (for handles) or 0 (for bitmasks/values)
\item \attr{noautovalidity} -  prevents automatic validity language being generated
          for the tagged item. Only suppresses item-specific validity - parenting issues
          etc. are still captured.
\item \attr{externsync} - optional. A boolean string, which must have the
      value \code{"true"} if present, indicating that this parameter (e.g.
      the object a handle refers to, or the contents of an array a pointer
      refers to) is modified by the command, and is not protected against
      modification in multiple app threads. Parameters which do not have
      this attribute are assumed to not require external synchronization.
\end{itemize}

\subsubsection{Contents of \tag{param} tags}

The text elements of a \tag{param} tag, with all other tags removed, is
legal C code describing the type and name of a function parameter. In
addition it may contain two semantic tags:

\begin{itemize}
\item The \tag{type} tag is optional, and contains text which is a
      valid type name found in \tag{type} tag, and indicates that this
      type must be previously defined for the definition of the command
      to succeed. Builtin C types, and any derived types which are
      expected to be found in other header files, should not be wrapped
      in \tag{type} tags.
\item The \tag{name} tag is required, and contains the parameter name being
      described.
\end{itemize}

\subsection{Example of a \tag{commands} tag}

\begin{verbatim}
<commands>
    <command>
        <proto><type>VkResult</type> <name>vkCreateInstance</name></proto>
        <param>const <type>VkInstanceCreateInfo</type>* <name>pCreateInfo</name></param>
        <param><type>VkInstance</type>* <name>pInstance</name></param>
    </command>
</commands>
\end{verbatim}

When processed into a C header, this results in

\begin{verbatim}
VkResult vkCreateInstance(
    const VkInstanceCreateInfo* pCreateInfo,
    VkInstance* pInstance);
\end{verbatim}

\subsection{Parameter validation (\tag{validity}) tags}
\label{tag:command:validity}

The \tag{validity} tag, if present defines valid use cases and values for
command parameters

\paragraph{Contents of \tag{validity} tags}
\par\vskip 11pt

Each \tag{validity} tag contains zero or more \tag{usage} tags. Each
\tag{usage} tag is intended to represent a specific validation requirement
for the command, and contains arbitrary asciidoc text describing that
requirement.


\section{API Features / Versions (\tag{feature} tag)}
\label{tag:feature}

API features are described in individual \tag{feature} tags. A feature is
the set of interfaces (enumerants and commands) defined by a particular API
and version, such as Vulkan 1.0, and includes all profiles of that API and
version.

\subsection{Attributes of \tag{feature} tags}

\begin{itemize}
\item \attr{api} - required. API name this feature is for (see
      section~\ref{schema:apiname}), such as \code{vk}.
\item \attr{name} - required. Version name, used as the C preprocessor token
      under which the version's interfaces are protected against multiple
      inclusion. Example: \code{VK\_VERSION\_1\_0}.
\item \attr{number} - required. Feature version number, usually a string
      interpreted as $majorNumber.minorNumber$. Example: \code{4.2}.
\item \attr{protect} - optional. An additional preprocessor token used to
      protect a feature definition. Usually another feature or extension
      \attr{name}. Rarely used, for odd circumstances where the definition
      of a feature or extension requires another to be defined first.
\item \attr{comment} - optional. Arbitrary string (unused).
\end{itemize}

\subsection{Contents of \tag{feature} tags}

Zero or more \tag{require} and \tag{remove} tags (see
section~\ref{tag:required}), in arbitrary order. Each tag describes a
set of interfaces that is respectively required for, or removed from,
this feature, as described below.

\subsection{Example of a \tag{feature} tag}

\begin{verbatim}
<feature api="vulkan" name="VK_VERSION_1_0" number="1.0">
    <require comment="Header boilerplate">
        <type name="vk_platform"/>
    </require>
    <require comment="API constants">
        <enum name="VK_MAX_PHYSICAL_DEVICE_NAME"/>
        <enum name="VK_LOD_CLAMP_NONE"/>
    </require>
    <require comment="Device initialization">
        <command name="vkCreateInstance"/>
    </require>
</feature>
\end{verbatim}

When processed into a C header for Vulkan, this results in:

\begin{verbatim}
#ifndef VK_VERSION_1_0
#define VK_VERSION_1_0 1
#define VK_MAX_EXTENSION_NAME   256
#define VK_LOD_CLAMP_NONE       MAX_FLOAT
typedef VkResult (VKAPI_PTR *PFN_vkCreateInstance)(const VkInstanceCreateInfo* pCreateInfo, VkInstance* pInstance);
#ifdef VK_PROTOTYPES
VKAPI_ATTR VkResult VKAPI_CALL vkCreateInstance(
    const VkInstanceCreateInfo*                 pCreateInfo,
    VkInstance*                                 pInstance);
#endif
#endif /* VK_VERSION_1_0 */
\end{verbatim}


\section{Extension Blocks (\tag{extensions} tag)}
\label{tag:extensions}

The \tag{extensions} tag contains definitions of each of the extenions
which are defined for the API.

\subsection{Attributes of \tag{extensions} tags}

None.

\subsection{Contents of \tag{extensions} tags}

Each \tag{extensions} block contains zero or more \tag{extension} tags,
each describing an API extension, in arbitrary order (although they are
typically ordered by sorting on the extension name, to improve human
readability).


\section{API Extensions (\tag{extension} tag)}
\label{tag:extension}

API extensions are described in individual \tag{extension} tags. An
extension is the set of interfaces defined by a particular API extension
specification, such as \code{ARB\_multitexture}. \tag{extension} is
similar to \tag{feature}, but instead of having \attr{version} and
\attr{profile} attributes, instead has a \attr{supported} attribute,
which describes the set of API names which the extension can potentially
be implemented against.

\subsection{Attributes of \tag{extension} tags}

\begin{itemize}
\item \attr{name} - required. Extension name, following the conventions
      in the Vulkan Specification.
      Example: \code{name="VK\_VERSION\_1\_0"}.
\item \attr{number} - required. A decimal number which is the registered,
      unique extension number for \attr{name}.
\item \attr{supported} - required. A regular expression, with an implicit
      \code{\^{}} and \code{\$} bracketing it, which should match the
      \attr{api} tag of a set of \tag{feature} tags.
\item \attr{protect} - optional. An additional preprocessor token used to
      protect an extension definition. Usually another feature or extension
      \attr{name}. Rarely used, for odd circumstances where the definition
      of an extension requires another extension or a header file to be
      defined first.
\item \attr{author} - optional. The author name, such as a full company
      name. If not present, this can be taken from the corresponding
      \tag{tag} attribute. However, \code{EXT} and other multi-vendor
      extensions may not have a well-defined author or contact in the tag.
\item \attr{contact} - optional. The contact who registered or is currently
      responsible for extensions and layers using the tag, including
      sufficient contact information to reach the contact such as individual
      name together with email address, Github username, or other contact
      information. If not present, this can be taken from the corresponding
      \tag{tag} attribute just like \attr{author}.
\item \attr{comment} - optional. Arbitrary string (unused).
\end{itemize}

\subsection{Contents of \tag{extension} tags}

Zero or more \tag{require} and \tag{remove} tags (see
section~\ref{tag:required}), in arbitrary order. Each tag describes a
set of interfaces that is respectively required for, or removed from,
this extension, as described below.

\subsection{Example of an \tag{extensions} tag}

\begin{verbatim}
<extension name="VK_KHR_display_swapchain" number="4" supported="vulkan">
    <require>
        <enum value="9" name="VK_KHR_DISPLAY_SWAPCHAIN_SPEC_VERSION"/>
        <enum value="4" name="VK_KHR_DISPLAY_SWAPCHAIN_EXTENSION_NUMBER"/>
        <enum value="&quot;VK_KHR_display_swapchain&quot;"
              name="VK_KHR_DISPLAY_SWAPCHAIN_EXTENSION_NAME"/>
        <type name="VkDisplayPresentInfoKHR"/>
        <command name="vkCreateSharedSwapchainsKHR"/>
    </require>
</extension>
\end{verbatim}

The \attr{supported} attribute says that the extension is defined for the
default profile (\code{vulkan}). When processed into a C header for the
\code{vulkan} profile, this results in header contents something like
(assuming corresponding definitions of the specified \tag{type} and
\tag{command} elsewhere in the XML):

\begin{verbatim}
#define VK_KHR_display_swapchain 1
#define VK_KHR_DISPLAY_SWAPCHAIN_SPEC_VERSION 9
#define VK_KHR_DISPLAY_SWAPCHAIN_EXTENSION_NUMBER 4
#define VK_KHR_DISPLAY_SWAPCHAIN_EXTENSION_NAME "VK_KHR_display_swapchain"

typedef struct VkDisplayPresentInfoKHR {
    VkStructureType                             sType;
    const void*                                 pNext;
    VkRect2D                                    srcRect;
    VkRect2D                                    dstRect;
    VkBool32                                    persistent;
} VkDisplayPresentInfoKHR;

typedef VkResult (VKAPI_PTR *PFN_vkCreateSharedSwapchainsKHR)(
    VkDevice device, uint32_t swapchainCount,
    const VkSwapchainCreateInfoKHR* pCreateInfos,
    const VkAllocationCallbacks* pAllocator,
    VkSwapchainKHR* pSwapchains);

#ifdef VK_PROTOTYPES
VKAPI_ATTR VkResult VKAPI_CALL vkCreateSharedSwapchainsKHR(
    VkDevice                                    device,
    uint32_t                                    swapchainCount,
    const VkSwapchainCreateInfoKHR*             pCreateInfos,
    const VkAllocationCallbacks*                pAllocator,
    VkSwapchainKHR*                             pSwapchains);
#endif
\end{verbatim}


\section{Required and Removed Interfaces (\tag{require} and \tag{remove} tags)}
\label{tag:required}

A \tag{require} block defines a set of interfaces (types, enumerants and
commands) {\em required} by a \tag{feature} or \tag{extension}. A
\tag{remove} block defines a set of interfaces {\em removed} by a
\tag{feature} (this is primarily useful for future profiles of an API which
may choose to deprecated and/or remove some interfaces - extensions should
never remove interfaces, although this usage is allowed by the schema).
Except for the tag name and behavior, the contents of \tag{require} and
\tag{remove} tags are identical.

\subsection{Attributes of \tag{require} and \tag{remove} tags}

\begin{itemize}
\item \attr{profile} - optional. String name of an API profile. Interfaces
      in the tag are only required (or removed) if the specified profile is
      being generated. If not specified, interfaces are required (or
      removed) for all API profiles.
\item \attr{comment} - optional. Arbitrary string (unused).
\item \attr{api} - optional. An API name (see section~\ref{schema:apiname}).
      Interfaces in the tag are only required (or removed) if the specified
      API is being generated. If not specified, interfaces are required (or
      removed) for all APIs.

      {\bf The \attr{api} attribute is only supported inside
      \tag{extension} tags,} since \tag{feature} tags already define a
      specific API.
\end{itemize}

\subsection{Contents of \tag{require} and \tag{remove} tags}

Zero or more of the following tags, in any order:

\begin{itemize}
\item \tag{command} specifies an required (or removed) command defined
      in a \tag{commands} block. The tag has no content, but contains
      attributes:

      \begin{itemize}
      \item \attr{name} - required. Name of the command.
      \item \attr{comment} - optional. Arbitrary string (unused).
      \end{itemize}
\item \tag{enum} specifies an required (or removed) enumerant defined in a
      \tag{enums} block. All forms of this tag support the following
      attributes:

      \begin{itemize}
      \item \attr{name} - required. Name of the enumerant.
      \item \attr{comment} - optional. Arbitrary string (unused).
      \end{itemize}

      There are two forms of \tag{enum} tags:

      {\bf Reference enums} simply pull in the definition of an enumerant
      given in a separate \tag{enums} block. Reference enums are the most
      common usage, and no attributes other than \attr{name} and
      \attr{comment} are supported for them.

      {\bf Extension enums} define the value of an enumerant inline in an
      \tag{extensions} block. There are several variants, depending on which
      additional tags are defined:

      \begin{itemize}
      \item Attributes \attr{value} and (optionally) \attr{type} define a
            constant value in the same fashion as an \tag{enum} tag in an
            \tag{enums} block (see section~\ref{tag:enum}).
      \item Attribute \attr{bitpos} defines a constant bitmask value in the
            same fashion as an \tag{enum} tag in an \tag{enums} block (see
            section~\ref{tag:enum}).
      \item Attribute \attr{extends} allows an extension enumerant to be
            added to a separately defined enumerated type whose name is
            specified by the contents of the \attr{extends} attribute (e.g.
            a \tag{type} tag with \attr{category} "enum", pulling in an
            \tag{enums} block).

            There are two ways to extend an enumerated type:

            \begin{itemize}
            \item If \attr{bitpos} is also specified, the tag defines a
                  bitmask value, but adds its definition to the enumeranted
                  type specified by \attr{extends} instead of as a
                  compile-time constant.
            \item If \attr{offset} is also specified, the tag adds a new
                  enumerant value to the enumerated type specified by
                  \attr{extends}. The actual value defined depends on the
                  extension number (the \attr{number} attribute of the
                  \tag{extensions} tag) and the offset, as defined in the
                  ``Layers and Extensions'' appendix of the Vulkan
                  Specification. The \attr{dir} attribute may also be
                  specified as {\tt dir="-"} if the calculated value should
                  be negative instead of positive. Negative enumerant values
                  are normally used only for Vulkan error codes.
            \end{itemize}
      \end{itemize}

      Examples of various types of extension enumerants are given below in
      section~\ref{tag:required:examples}.
\item \tag{type} specifies a required (or removed) type defined in a
      \tag{types} block. Most types are picked up implicitly by using the
      \tag{type} tags of commands, but in a few cases, additional types need
      to be specified explicitly. It is unlikely that a type would ever be
      removed, although this usage is allowed by the schema. The tag has no
      content, but contains elements:

      \begin{itemize}
      \item \attr{name} - required. Name of the type.
      \item \attr{comment} - optional. Arbitrary string (unused).
      \end{itemize}
\end{itemize}


\subsection{Examples of Extension Enumerants}
\label{tag:required:examples}

Examples of each of the supported extension enumerant \tag{enum} tags are
given below. Note that extension enumerants are supported only inside
\tag{extension} blocks - not in \tag{feature} blocks\footnote{
    However, we will have to define additional XML tags and/or syntax for
    future core versions of Vulkan, to properly tag and group core
    enumerants for Vulkan 1.0, 1.1, etc.}.

\begin{verbatim}
<extensions>
    <extension name="VK_KHR_test_extension" number="1" supported="vulkan">
        <require>
            <enum value="42" name="VK_KHR_theanswer"/>
            <enum bitpos="29" name="VK_KHR_bitmask"/>
            <enum offset="0" dir="-" extends="VkResult"
                  name="VK_ERROR_SURFACE_LOST_KHR"/>
            <enum offset="1" extends="VkResult"
                  name="VK_SUBOPTIMAL_KHR"/>
            <enum bitpos="31" extends="VkResult"
                  name="VK_KHR_EXTENSION_BITFIELD"/>
        </require>
    </extension>
</extensions>
\end{verbatim}

The corresponding header file will include definitions like this:

\begin{verbatim}
typedef enum VkResult {
    <previously defined VkResult enumerant values},
    VK_ERROR_SURFACE_LOST_KHR = -1000000000,
    VK_SUBOPTIMAL_KHR = 1000000001,
    VK_KHR_EXTENSION_BITFIELD = 0x80000000,
};

#define VK_KHR_test_extension 1
#define VK_KHR_theanswer 42
#define VK_KHR_bitmask 0x20000000
\end{verbatim}

\clearpage
\appendix

\section{Examples / FAQ / How Do I?}
\label{examples}

For people new to the Registry, it won't be immediately obvious how to make
changes. This section includes some tips and examples that will help you
make changes to the Vulkan headers by changing the Registry XML description.

First, follow the steps in section~\ref{starting} to get the Vulkan Gitlab
repository containing the registry and assemble the tools necessary to work
with the XML registry. Once you're able to regenerate \code{vulkan.h} from
\code{vk.xml}, you can start making changes.


\subsection{General Strategy}

If you are {\bf adding} to the API, perform the following steps to {\bf
create} the description of that API element:

\begin{itemize}
\item For each type, enum group, compile-time constant, and command being
      added, create appropriate new \tag{type}, \tag{enums}, \tag{enum}, or
      \tag{command} tags defining the interface in question.
\item Make sure that all added types and commands appropriately tag their
      dependencies on other types by adding nested \tag{type} tags.
\item Make sure that each new tag defines the name of the corresponding
      type, enum group, constant, or command, and that structure/union types
      and commands tag the types and names of all their members and
      parameters. This is essential for the automatic dependency process to
      work.
\end{itemize}

If you are {\bf modifying} existing APIs, just make appropriate changes
in the existing tags.

Once the definition is added, proceed to the next section to create
dependencies on the changed feature.


\subsection{API Feature Dependencies}

When you add new API elements, they will not result in corresponding changes
in the generated header unless they are {\bf required} by the interface
being generated. This makes it possible to include different API versions
and extensions in a single registry and pull them out as needed. So you must
introduce a dependency on new features in the corresponding \tag{feature}
tag.

Initially, the only API feature is Vulkan 1.0, so there is only one
\tag{feature} tag in \code{vk.xml}. You can find it by searching
for the following block of \code{vk.xml}:

\begin{verbatim}
<!-- SECTION: Vulkan API interface definitions -->
<feature api="vulkan" name="VK_VERSION_1_0" number="1.0">
\end{verbatim}

Inside the \tag{feature} tag are nested multiple \tag{require} tags. These
are just being used as a logical grouping mechanism for related parts of
Vulkan 1.0 at present, though they may have more meaningful roles in the
future if different API profiles are defined.


\subsubsection{API Feature Walkthrough}

This section walks through the first few required API features in the
\code{vk.xml} \tag{feature} tag, showing how each requirement pulls in type,
token, and command definitions and turns those into definitions in the C
header file \code{vulkan.h} .

Consider the first few lines of the \tag{feature}:

\begin{verbatim}
<require comment="Header boilerplate">
    <type name="vk_platform"/>
</require>
<require comment="API constants">
    <enum name="VK_MAX_PHYSICAL_DEVICE_NAME"/>
    <enum name="VK_MAX_EXTENSION_NAME"/>
    ...
</require>
<require comment="Device initialization">
    <command name="vkCreateInstance"/>
    ...
\end{verbatim}

The first \tag{require} block says to require a type named
\code{vk_platform}. If you look at the beginning of the \tag{types} section,
there's a corresponding definition:


\begin{verbatim}
<type name="vk_platform">#include "vk_platform.h"
#define VK_MAKE_VERSION(major, minor, patch) \
    ((major &lt;&lt; 22) | (minor &lt;&lt; 12) | patch)
    ...
\end{verbatim}

\noindent section which is invoked by the requirement and emits a bunch of
boilerplate C code. The explicit dependency isn't strictly required since
\code{vk_platform} will be required by many other types, but placing it
first causes this to appear first in the output file.

Note that \code{vk_platform} does not correspond to an actual C type, but
instead to a collection of freeform preprocessor includes and macros and
comments. Most other \tag{type} tags do define a specific type and are much
simpler, but this approach can be used to inject arbitrary C into
\code{vulkan.h} {\bf when there's no other way}. In general inserting
arbitrary C is strongly discouraged outside of specific special cases like
this.

The next \tag{require} block pulls in some compile-time constants. These
correspond to the definitions found in the first \tag{enums} section of
\code{vk.xml}:

\begin{verbatim}
<!-- SECTION: Vulkan enumerant (token) definitions. -->

<enums comment="Misc. hardcoded constants - not an enumerated type">
        <!-- This is part of the header boilerplate -->
    <enum value="256"        name="VK_MAX_PHYSICAL_DEVICE_NAME"/>
    <enum value="256"        name="VK_MAX_EXTENSION_NAME"/>
    ...
\end{verbatim}

The third \tag{require} block starts pulling in some Vulkan commands. The
first command corresponds to the following definition found in the
\tag{commands} section of \code{vk.xml}:

\begin{verbatim}
<commands>
    <command>
        <proto><type>VkResult</type> <name>vkCreateInstance</name></proto>
        <param>const <type>VkInstanceCreateInfo</type>* <name>pCreateInfo</name></param>
        <param><type>VkInstance</type>* <name>pInstance</name></param>
    </command>
    ...
\end{verbatim}

In turn, the \tag{command} tag requires the \tag{type}s \code{VkResult},
\code{VkInstanceCreateInfo}, and \code{VkInstance} as part of its
definition. The definitions of these types are determined as follows:

For \code{VkResult}, the corresponding required \tag{type} is:

\begin{verbatim}
<type name="VkResult" category="enum"/>
\end{verbatim}

\noindent Since this is an enumeration type, it simply links to an
\tag{enums} tag with the same name:

\begin{verbatim}
<enums name="VkResult" type="enum" expand="VK_RESULT" comment="Error and return codes">
    <!-- Return codes for successful operation execution (positive values) -->
    <enum value="0"     name="VK_SUCCESS"/>
    <enum value="1"     name="VK_UNSUPPORTED"/>
    <enum value="2"     name="VK_NOT_READY"/>
    ...
\end{verbatim}

For \tag{VkInstanceCreateInfo}, the required \tag{type} is:

\begin{verbatim}
<type category="struct" name="VkInstanceCreateInfo">
    <member><type>VkStructureType</type>        <name>sType</name></member>                          <!-- Should be VK_STRUCTURE_TYPE_INSTANCE_CREATE_INFO -->
    <member>const void*                         <name>pNext</name></member>                          <!-- Pointer to next structure -->
    <member>const <type>VkApplicationInfo</type>* <name>pAppInfo</name></member>
    <member>const <type>VkAllocCallbacks</type>* <name>pAllocCb</name></member>
    <member><type>uint32_t</type>               <name>extensionCount</name></member>
    <member>const <type>char</type>*const*      <name>ppEnabledExtensionNames</name></member>        <!-- layer or extension name to be enabled -->
</type>
\end{verbatim}

\noindent This is a structure type, defining a C \code{struct} with all the
members defined in each \tag{member} tag in order. In addition, it requires
some other types, whose definitions are located by name in exactly the same
fashion.

For the final direct dependency of the command, \code{VkInstance}, the
required \tag{type} is:

\begin{verbatim}
<!-- Types which can be void pointers or class pointers, selected at compile time -->
<type>VK_DEFINE_BASE_HANDLE(<name>VkObject</name>)</type>
<type>VK_DEFINE_DISP_SUBCLASS_HANDLE(<name>VkInstance</name>, <type>VkObject</type>)</type>
\end{verbatim}

\noindent In this case, the type \code{VkInstance} is defined by a special
compile-time macro which defines it as a derived class of \code{VkObject}
(for C++) or a less typesafe definition for (for C). This macro isn't part
of the type dependency analysis, just the boilerplate used in the header.

If these are the only \tag{feature} dependencies in \code{vk.xml}, the
resulting \code{vulkan.h} header will look like this:

\begin{verbatim}

#ifndef __vulkan_h_
#define __vulkan_h_ 1

#ifdef __cplusplus
extern "C" {
#endif

/*
** Copyright (c) 2015-2016 The Khronos Group Inc.
    ...
*/

/*
** This header is generated from the Khronos Vulkan XML API Registry.
**
** Generated on date 20160208
*/


#define VK_VERSION_1_0 1
#include "vk_platform.h"
#define VK_MAKE_VERSION(major, minor, patch) \
    ((major << 22) | (minor << 12) | patch)

// Vulkan API version supported by this file
#define VK_API_VERSION VK_MAKE_VERSION(0, 104, 0)

#if defined (__cplusplus) && (VK_UINTPTRLEAST64_MAX == UINTPTR_MAX)
    #define VK_TYPE_SAFE_COMPATIBLE_HANDLES 1
#endif

#if defined(VK_TYPE_SAFE_COMPATIBLE_HANDLES) && !defined(VK_DISABLE_TYPE_SAFE_HANDLES)
    #define VK_DEFINE_PTR_HANDLE(_obj) struct _obj##_T { char _dummy; }; typedef _obj##_T* _obj;
    #define VK_DEFINE_PTR_SUBCLASS_HANDLE(_obj, _base) struct _obj##_T : public _base##_T {}; typedef _obj##_T* _obj;

    #define VK_DEFINE_BASE_HANDLE(_obj) VK_DEFINE_PTR_HANDLE(_obj)
    #define VK_DEFINE_DISP_SUBCLASS_HANDLE(_obj, _base) VK_DEFINE_PTR_SUBCLASS_HANDLE(_obj, _base)
    #define VK_DEFINE_NONDISP_SUBCLASS_HANDLE(_obj, _base) VK_DEFINE_PTR_SUBCLASS_HANDLE(_obj, _base)
#else
    #define VK_DEFINE_BASE_HANDLE(_obj) typedef VkUintPtrLeast64 _obj;
    #define VK_DEFINE_DISP_SUBCLASS_HANDLE(_obj, _base) typedef uintptr_t _obj;
    #define VK_DEFINE_NONDISP_SUBCLASS_HANDLE(_obj, _base) typedef VkUintPtrLeast64 _obj;
#endif

typedef enum {
    VK_SUCCESS = 0,
    VK_UNSUPPORTED = 1,
    VK_NOT_READY = 2,
    ...
} VkResult;
typedef enum {
    VK_STRUCTURE_TYPE_APPLICATION_INFO = 0,
    ...
} VKStructureType;
typedef struct {
    VkStructureType                             sType;
    const void*                                 pNext;
    const char*                                 pAppName;
    uint32_t                                    appVersion;
    const char*                                 pEngineName;
    uint32_t                                    engineVersion;
    uint32_t                                    apiVersion;
} VkApplicationInfo;
typedef enum {
    VK_SYSTEM_ALLOC_TYPE_API_OBJECT = 0,
    ...
} VkSystemAllocType;
typedef void* (VKAPI_PTR *PFN_vkAllocFunction)(
    void*                           pUserData,
    size_t                          size,
    size_t                          alignment,
    VkSystemAllocType               allocType);
typedef void (VKAPI_PTR *PFN_vkFreeFunction)(
    void*                           pUserData,
    void*                           pMem);
typedef struct {
    void*                                       pUserData;
    PFN_vkAllocFunction                         pfnAlloc;
    PFN_vkFreeFunction                          pfnFree;
} VkAllocCallbacks;
typedef struct {
    VkStructureType                             sType;
    const void*                                 pNext;
    const VkApplicationInfo*                    pAppInfo;
    const VkAllocCallbacks*                     pAllocCb;
    uint32_t                                    extensionCount;
    const char*const*                           ppEnabledExtensionNames;
} VkInstanceCreateInfo;
VK_DEFINE_BASE_HANDLE(VkObject)
VK_DEFINE_DISP_SUBCLASS_HANDLE(VkInstance, VkObject)
#define VK_MAX_PHYSICAL_DEVICE_NAME       256
#define VK_MAX_EXTENSION_NAME             256
typedef VkResult (VKAPI_PTR *PFN_vkCreateInstance)(const VkInstanceCreateInfo* pCreateInfo, VkInstance* pInstance);
#ifdef VK_PROTOTYPES
VKAPI_ATTR VkResult VKAPI_CALL vkCreateInstance(
    const VkInstanceCreateInfo*                 pCreateInfo,
    VkInstance*                                 pInstance);
#endif

#ifdef __cplusplus
}
#endif

#endif
\end{verbatim}

Note that several additional types are pulled in by the type dependency
analysis, but only those types, commands, and tokens required by the
specified features are generated.


\subsection{How To Add A Compile-Time Constant}

Go to the \tag{feature} tag and search for the nested block labelled

\begin{verbatim}
<require comment="API constants">
\end{verbatim}

In this block, add an (appropriately indented) tag like

\begin{verbatim}
    <enum name="VK_THE_ANSWER"/>
\end{verbatim}

Then go to the \tag{enums} block labelled

\begin{verbatim}
<enums comment="Misc. hardcoded constants - not an enumerated type">
\end{verbatim}

In this block, add a tag whose \attr{name} attribute matches the \attr{name}
you defined above and whose \attr{value} attribute is the value to give the
constant:

\begin{verbatim}
    <enum value="42"    name="VK_THE_ANSWER"/>
\end{verbatim}


\subsection{How To Add A Struct or Union Type}

For this example, assume we want to define a type
corresponding to a C \code{struct} defined as follows:

\begin{verbatim}
typedef struct {
    VkStructureType          sType;
    const void*              pNext;
    const VkApplicationInfo* pAppInfo;
    const VkAllocCallbacks*  pAllocCb;
    uint32_t                 extensionCount;
    const char*const*        ppEnabledExtensionNames;
} VkInstanceCreateInfo;
\end{verbatim}

If \code{VkInstanceCreateInfo} is the type of a parameter of a command in
the API, make sure that command's definition (see below for how to add a
command) puts \code{VkInstanceCreateInfo} in nested \tag{type} tags where
it's used.

Otherwise, if the struct type is not used directly by a command in the API,
nor required by a chain of type dependencies for other commands, an explicit
\tag{type} dependency should be added to the \tag{feature} tag. Go to the
\tag{types} tag and search for the nested block labelled

\begin{verbatim}
<require comment="API types not used by commands">
    ...
\end{verbatim}

In this block, add a tag whose \attr{name} attribute matches the \attr{name}
of the struct type being defined:

\begin{verbatim}
<require comment="API types not used by commands">
    <type name="VkInstanceCreateInfo"/>
    ...
\end{verbatim}

Then go to the \tag{types} tag and add a new \tag{type} tag defining
the struct names and members, somewhere below
the corresponding comment, like this:

\begin{verbatim}
<types>
    ...
    <!-- Struct types -->
        <type category="struct" name="VkInstanceCreateInfo">
            <member><type>VkStructureType</type>
                    <name>sType</name></member>
            <member>const void*
                    <name>pNext</name></member>
            <member>const <type>VkApplicationInfo</type>*
                    <name>pAppInfo</name></member>
            <member>const <type>VkAllocCallbacks</type>*
                    <name>pAllocCb</name></member>
            <member><type>uint32_t</type>
                    <name>extensionCount</name></member>
            <member>const <type>char</type>*const*
                    <name>ppEnabledExtensionNames</name></member>
        </type>
        ...
\end{verbatim}

If any of the member types are types also defined in the header, make sure
to enclose those type names in nested \tag{type} tags, as shown above. Basic
C types should not be tagged.

If the type is a C \code{union}, rather than a \code{struct}, then set the
value of the \attr{category} attribute to \code{"union"} instead of
\code{"struct"}.


\subsection{How To Add An Enumerated Type}

For this example, assume we want to define a type corresponding to a C
\code{enum} defined as follows:

\begin{verbatim}
typedef enum {
    VK_DEVICE_CREATE_VALIDATION_BIT = 0x00000001,
    VK_DEVICE_CREATE_MULTI_DEVICE_IQ_MATCH_BIT = 0x00000002;
} VkDeviceCreateFlagBits.
\end{verbatim}

If \code{VkDeviceCreateFlagBits} is the type of a parameter to a command in
the API, or of a member in a structure or union, make sure that command
parameter or struct member's definition puts \code{VkDeviceCreateFlagBits}
in nested \tag{type} tags where it's used.

Otherwise, if the enumerated type is not used directly by a command in the
API, nor required by a chain of type dependencies for commands and structs,
an explicit \tag{type} dependency should be added to the \tag{feature} tag
in exactly the same fashion as described above for \code{struct} types.

Next, go to the line labelled

\begin{verbatim}
<!-- SECTION: Vulkan enumerant (token) definitions. -->
\end{verbatim}

Below this line, add an \tag{enums} tag whose \attr{name} attribute matches
the \tag{type} name \code{VkDeviceCreateFlagBits}, and whose contents
correspond to the individual fields of the enumerated type:

\begin{verbatim}
<enums name="VkDeviceCreateFlagBits" type="bitmask">
    <enum bitpos="0" name="VK_DEVICE_CREATE_VALIDATION_BIT"/>
    <enum bitpos="1" name="VK_DEVICE_CREATE_MULTI_DEVICE_IQ_MATCH_BIT"/>
</enums>
\end{verbatim}

Several other attributes of the \tag{enums} tag can be set. In this case,
the \attr{type} attribute is set to \code{"bitmask"}, indicating that the
individual enumerants represent elements of a bitmask.

The individual \tag{enum} tags define the enumerants, just like the
definition for compile-time constants described above. In this case, because
the enumerants are bitfields, their values are specified using the
\attr{bitpos} attribute. The value of this attribute must be an integer in
the range $[0,31]$ specifying a single bit number, and the resulting value
is printed as a hexadecimal constant corresponding to that bit.

It is also possible to specify enumerant values using the \attr{value}
attribute, in which case the specified numeric value is passed through to
the C header unchanged.


\subsection{How to Add A Command}

For this example, assume we want to define the command:

\begin{verbatim}
VKAPI_ATTR VkResult VKAPI_CALL vkCreateInstance(
    const VkInstanceCreateInfo*                 pCreateInfo,
    VkInstance*                                 pInstance);
\end{verbatim}

Commands must always be explicitly required in the \tag{feature} tag. In
that tag, you can use an existing \tag{require} block including API features
which the new command should be grouped with, or define a new block. For
this example, add a new block, and require the command by using the
\tag{command} tag inside that block:

\begin{verbatim}
<!-- SECTION: Vulkan API interface definitions -->
<feature api="vulkan" name="VK_VERSION_1_0" number="1.0">
    ...
    <require comment="Device initialization">
        <command name="vkCreateInstance"/>
    </require>
    ...
</feature>
\end{verbatim}

The \tag{require} block may include a \attr{comment} attribute whose value
is a descriptive comment of the contents required within that block. The
comment is not currently used in header generation, but might be in the
future, so use comments which are polite and meaningful to users of
\code{vulkan.h}.

Then go to the \tag{commands} tag and add a new \tag{command} tag defining
the command, preferably sorted into alphabetic order with other commands for
ease of reading, as follows:

\begin{verbatim}
<!-- SECTION: Vulkan command definitions -->
<commands>
    ...
    <command>
        <proto><type>VkResult</type>
               <name>vkCreateInstance</name></proto>
        <param>const <type>VkInstanceCreateInfo</type>*
               <name>pCreateInfo</name></param>
        <param><type>VkInstance</type>*
               <name>pInstance</name></param>
    </command>
    ...
</commands>
\end{verbatim}

The \tag{proto} tag defines the return type and function name of the
command. The \tag{param} tags define the command's parameters in the order
in which they're passed, including the parameter type and name. The contents
are laid out in the same way as the structure \tag{member} tags described
previously.


\subsection{More Complicated API Representations}

The registry schema can represent a good deal of additional information, for
example by creating multiple \tag{feature} tags defining different API
versions and extensions. This capability is not yet relevant to Vulkan.
Those capabilities will be documented as they are needed.


\subsection{More Complicated Output Formats And Other Languages}

The registry schema is oriented towards C-language APIs. Types and commands
are defined using syntax which is a subset of C, especially for structure
members and command parameters. It would be possible to use a
language-independent syntax for representing such information, but since we
are writing a C API, any such representation would have to be converted into
C anyway at some stage.

The \code{vulkan.h} header is written using an {\bf output generator} object
in the Python scripts. This output generator is specialized for C, but the
design of the scripts is intended to support writing output generators for
other languages as well as purposes such as documentation (e.g. generating
Asciidoc fragments corresponding to types and commands for use in the API
specification and reference pages). When targeting other languages, the
amount of parsing required to convert type declarations into other languages
is small. However, it will probably be necessary to modify some of the
boilerplate C text, or specialize the tags by language, to support such
generators.


\subsection{Additional Semantic Tagging}

The schema is being extended to support semantic tags describing
various properties of API features, such as:

\begin{itemize}
\item constraints on allowed scalar values to function parameters (non-NULL,
      normalized floating-point, etc.)
\item length of arrays corresponding to function pointer parameters
\item miscellaneous properties of commands such as whether the application
      or system is responsible for threadsafe use; which queues they may be
      issued on; whether they are aliases or otherwise related to other
      commands; etc.
\end{itemize}

These tags will be used by other tools for purposes such as helping create
validation layers, generating serialization code, and so on. We'd like to
eventually represent everything about the API that is amenable to automatic
processing within the registry schema. Please make suggestions on the Gitlab
issue tracker.


\subsection{Stability of the XML Database and Schema}
\label{general:stability}

The Vulkan XML schema is evolving in response to corresponding
changes in the Vulkan API and ecosystem. Most such change will probably be
confined to adding attributes to existing tags and properly expressing the
relationships to them, and making API changes corresponding to accepted
feature requests. Changes to the schema should be described in the change
log of this document (see section~\ref{changelog}). Changes to the
\code{.xml} files and Python scripts are logged in Gitlab history.


\section{Change Log}
\label{changelog}

\begin{itemize}
\item 2016/02/07 - Add \attr{vendorids} tags for Khronos vendor IDs.
\item 2015/12/10 - Add \attr{author} and \attr{contact} attributes for
      \tag{extension} tags.
\item 2015/12/07 - Move \code{vulkan/vulkan.h} to a subdirectory.
\item 2015/12/01 - Add \tag{tags} tags for author tags.
\item 2015/11/18 - Bring documentation and schema up to date for extension
      enumerants.
\item 2015/11/02 - Bring documentation and schema up to date with several
      recent merges, including \tag{validity} tags. Still out of date WRT
      extension enumerants, but that will change soon.
\item 2015/09/08 - Rename {\tt threadsafe} attribute to \attr{externsync},
      and {\tt implicitunsafeparams} tag to \attr{implicitexternsync}.
\item 2015/09/07 - Update \tag{command} tag description to remove the
      \attr{threadsafe} attribute and replace it with a combination of
      \attr{threadunsafe} attributes on individual parameters, and
      \tag{implicitunsafeparams} tags describing additional unsafe objects
      for the command.
\item 2015/08/04 - Add \code{basetype} and \code{funcpointer}
      \attr{category} values for type tags, and explain the intended use and
      order in which types in each category are emitted.
\item 2015/07/02 - Update description of Makefile targets. Add
      descriptions of \attr{threadsafe}, \attr{queues}, and
      \attr{renderpass} attributes of \tag{command} tags in
      section~\ref{tag:command}, and of \attr{modified} attributes of
      \tag{param} tags in section~\ref{tag:command:param}
\item 2015/06/17 - add descriptions of allowed \attr{category} attribute
      values of \tag{type} tags, used to group and sort related categories
      of declarations together in the generated header.
\item 2015/06/04 - Add examples of making changes and additions to the
      registry in appendix~\ref{examples}.
\item 2015/06/03 - Move location to new ``vulkan'' Git repository. Add
      definition of \tag{type} tags for C struct/unions. Start adding
      examples of making changes in appendix~\ref{examples}.
\item 2015/06/02 - Branch from OpenGL specfile documentation and bring up to
      date with current Vulkan schema.
\end{itemize}

% % Copyright (c) 2013-2016 The Khronos Group Inc.
%
% Permission is hereby granted, free of charge, to any person obtaining a
% copy of this software and/or associated documentation files (the
% "Materials"), to deal in the Materials without restriction, including
% without limitation the rights to use, copy, modify, merge, publish,
% distribute, sublicense, and/or sell copies of the Materials, and to
% permit persons to whom the Materials are furnished to do so, subject to
% the following conditions:
%
% The above copyright notice and this permission notice shall be included
% in all copies or substantial portions of the Materials.
%
% THE MATERIALS ARE PROVIDED "AS IS", WITHOUT WARRANTY OF ANY KIND,
% EXPRESS OR IMPLIED, INCLUDING BUT NOT LIMITED TO THE WARRANTIES OF
% MERCHANTABILITY, FITNESS FOR A PARTICULAR PURPOSE AND NONINFRINGEMENT.
% IN NO EVENT SHALL THE AUTHORS OR COPYRIGHT HOLDERS BE LIABLE FOR ANY
% CLAIM, DAMAGES OR OTHER LIABILITY, WHETHER IN AN ACTION OF CONTRACT,
% TORT OR OTHERWISE, ARISING FROM, OUT OF OR IN CONNECTION WITH THE
% MATERIALS OR THE USE OR OTHER DEALINGS IN THE MATERIALS.

\documentclass{article}

% Various packages of possible use. {index} creates problems for some reason.
% \usepackage{index}
\def\specpdftitle{The Khronos Vulkan API Registry}
\usepackage[pdftex,bookmarksnumbered=true,linktocpage,plainpages=false,pdftitle={\specpdftitle},colorlinks=true]{hyperref}
\usepackage{amsmath,enumerate,epsfig,framed,mdframed,multicol,longtable,times,url}
\usepackage[normalem]{ulem}
% underscore allows use of _ instead of \_ in text. [strings] allows _ in file names
\usepackage[strings]{underscore}
\usepackage[breakwords,fit]{truncate}

% Index everything down through \paragraph
\setcounter{tocdepth}{5}

\makeatletter
\makeatother
\makeindex

% Some commonly used abbreviations

\def\code#1{{\tt #1}}
\def\tag#1{{\tt <#1>\index{<#1>}}}
% Because I keep mistyping \tag
\def\arg#1\tag{#1}
\def\attr#1{{\tt #1\index{#1}}}

\begin{document}

\title{The Khronos Vulkan API Registry for Vulkan}
\author{Jon Leech}
\date{Last updated 2016/02/07}
\maketitle

\begin{abstract}

This document describes the Khronos Vulkan API Registry schema, and provides
some additional information about using the registry and scripts to generate
a variety of outputs including C header files and several types of asciidoc
include files used in the Vulkan API specification and reference pages. The
underlying XML files and scripts are located on the Khronos Gitlab server,
currently in the {\em master} branch at URL

\begin{center}
{\bf \href{https://gitlab.khronos.org/vulkan/vulkan/tree/master/src/spec/}
          {https://gitlab.khronos.org/vulkan/vulkan/tree/master/src/spec/}}
\end{center}

\end{abstract}

\tableofcontents

\section{Introduction}

The registry uses an XML representation of the Vulkan API and a set of
Python scripts to manipulate the XML, based on the lxml Python bindings. It
includes an XML schema and validator.

The registry schema is based on, but not identical to the schema used for
the previously published OpenGL, OpenGL ES and EGL API registries. The
schema was extended to represent additional types and concepts not needed
for those APIS, such as structure and enumerant types.

The processed C header file corresponding to the registry is checked in
under src/vulkan/vulkan.h .


\section{Getting Started}
\label{starting}

You will need some tools installed.
For Linux systems, you will need the following packages installed (Debian
package names are noted, Ubuntu package names are probably the same).

\begin{itemize}
\item Python 3.x (\code{python3})
\item The lxml Python package (\code{python3-lxml})
\item Libxml (\code{libxml2})
\item A make tool, such as GNU make
\end{itemize}

The toolchain has also been run successfully on Microsoft Windows in the
Cygwin environment, and the individual Python-based components of the
toolchain can be run directly from the Windows command line. More details on
this are under src/spec/README in Git.

Once you have the right tools installed, perform the following steps:

\begin{itemize}
\item Check out the ``vulkan'' repository linked above from Khronos Gitlab
      (there are instructions at the link)
\item Go to \code{vulkan/src/spec} in your checked-out repo
\item Invoke \code{make clobber ; make install}
\end{itemize}

This should regenerate \code{vulkan.h} and install it in
\code{../vulkan/vulkan.h}. The result should be identical to the
version you just pulled from Gitlab (use \code{git diff
../vulkan/vulkan.h} to check).

Other Makefile targets include:

\begin{itemize}
\item \code{vulkan-docs} - regenerate the API specification and reference
      page asciidoc include files in \code{../doc/specs/vulkan/} under the
      subdirectories \code{enums}, \code{flags}, \code{protos}, and
      \code{structs}. These files are pulled in by the ref pages and API
      spec to put function prototypes and struct and enum declarations in
      documentation.

      You should probably {\bf not} update these files if you are proposing
      an API change - just the \code{vulkan.h} header. When a git merge is
      accepted including your change, the docs will be updated at that time.
      Including the dozens of include files in your merge makes it hard to
      read and adds no information beyond the header updates. Sometimes it
      is useful to include just one or two of the include files for context
      of a global change, however.

      However, you are strongly encouraged to updated the {\bf
      non-}autogenerated docs under \code{../doc/specs/vulkan/man/*.txt} to
      reflect API changes.
\item \code{full_install} - equivalent to \code{install} followed by
      \code{vulkan-docs}.
\item \code{validate} - validate \code{vk.xml} against the XML schema.
      Recommended if you're making nontrivial changes.
%% \item \code{vk.json} - generate JSON version of \code{vk.xml}. Currently the
%%       converter is generating syntactically but not semantically legal JSON,
%%       and it's difficult to map the XML schemaanso don't use this target yet.
\item \code{readme.pdf} - regenerate this document from the LaTeX source.
      Most people will never need to do this. If you do, you must have
      pdflatex installed, preferably from the TeTeX distribution.
\end{itemize}

If you just want to modify the API, changing \code{vk.xml} and running
``make'' should be all that's needed. See appendix~\ref{examples} for some
examples of modifying the XML.

If you want to repurpose the registry for reasons other than header file
and ref page include generation, or to generate headers for languages
other than C, start with the Makefile rules and inspect the files
\code{vk.xml}, \code{genvk.py}, \code{reg.py}, and \code{generator.py}.

If you're using other platforms, merge requests with additional
documentation on using the tools on those platforms would be very helpful.

\subsection{Header Generation Script - \code{genvk.py}}

When generating header files using the \code{genvk.py} script, an API name
and profile name are required, as shown in the Makefile examples.
Additionally, specific API versions and extensions can be required or
excluded. Based on this information, the generator script extracts the
relevant interfaces and creates a C-language header file for them.
\code{genvk.py} contains predefined generator options for the current
development version of Vulkan 1.0.

The generator script is intended to be generalizable to other languages by
writing new generator classes. Such generators would have to rewrite the C
types and definitions in the XML to something appropriate to their language.

\subsection{Registry Processing Script - \code{reg.py}}

XML processing is done in \code{reg.py}, which contains several objects and
methods for loading registries and extracting interfaces and extensions for
use in header generation. There is some internal documentation in the form
of comments, although nothing more extensive exists yet.

\subsection{Output Generator Script - \code{generator.py}}

Once the registry is loaded, the \code{COutputGenerator} class defined in
\code{generator.py} is used to create a header file. The
\code{DocOutputGenerator} class is used to create the asciidoc include
files. Output generators for other purposes can be added as needed.


\section{Vulkan Registry Schema}
\label{schema}

The format of the Vulkan registry is a top level \tag{registry} tag
containing \tag{types}, \tag{enums}, \tag{commands}, \tag{feature}, and
\tag{extension} tags describing the different elements of an API, as
explained below. This description corresponds to a formal Relax NG schema
file, \code{registry.rnc}, against which the XML registry files can be
validated.

At present the only registry in this schema is the core Vulkan API registry,
\code{vk.xml}.


\subsection{Profiles}
\label{schema:profile}

Types and enumerants can have different definitions depending on the API
profile requested. This capability is not used in the current Vulkan API but
may be in the future. Features and extensions can include some elements
conditionally depending on the API profile requested.


\subsection{API Names}
\label{schema:apiname}

The schema supports, but does not currently use an \attr{api} attribute on
several tags. This is an arbitrary string, specified at header generation
time, for labelling properties of a specific API or API profile. The string
can be, but is not necessarily, an actual API name. Names starting with ``vk''
are suggested if and when we start defining profiles of Vulkan.


\section{Registry Root (\tag{registry} tag)}
\label{schema:root}

A \tag{registry} contains the entire definition of one or more related
APIs.

\subsection{Attributes of \tag{registry} tags}

None.

\subsection{Contents of \tag{registry} tags}

Zero or more of each of the following tags, normally in this order
(although order shouldn't be important):

\begin{itemize}
\item \tag{comment} - Contains arbitrary text, such as a copyright
      statement.
\item \tag{vendorids} (see section~\ref{tag:vendorids}) - defines Khronos
      vendor IDs, described in detail in the ``Layers and Extensions''
      appendix of the Vulkan Specification.
\item \tag{tags} (see section~\ref{tag:tags}) - defines author prefixes used
      for extensions and layers. Prefixes are described in detail in the
      ``Layers and Extensions'' appendix of the Vulkan Specification.
\item \tag{types} (see section~\ref{tag:types}) - defines API types.
      Usually only one tag is used.
\item \tag{enums} (see section~\ref{tag:enums}) - defines API token names
      and values. Usually multiple tags are used. Related groups may be
      tagged as an enumerated type corresponding to a \tag{type} tag, and
      resulting in a C \code{enum} declaration. This ability is heavily used
      in the Vulkan API.
\item \tag{commands} (see section~\ref{tag:commands}) - defines API
      commands (functions). Usually only one tag is used.
\item \tag{feature} (see section~\ref{tag:feature}) - defines API
      feature interfaces (API versions, more or less). One tag per
      feature set.
\item \tag{extensions} (see section~\ref{tag:extensions}) - defines API
      extension interfaces. Usually only one tag is used, wrapping many
      extensions.
\end{itemize}


\section{Vendor IDs (\tag{vendorids} tag)}
\label{tag:vendorids}

The \tag{vendorids} tag contains individual \tag{vendorid} tags defining
vendor IDs for physical devices which do not have PCI vendor IDs.

Each \tag{vendorid} tag contains information defining a single vendor ID.

\subsection{Attributes of \tag{vendorid} tags}

\begin{itemize}
\item \attr{name} - required. The author prefix, as registered with Khronos.
      This must match an author prefix in the \attr{name} field of a
      \tag{tag} tag.
\item \attr{id} - required. The reserved vendor ID, as a hexadecimal number.
\item \attr{comment} - optional. Arbitrary string (unused).
\end{itemize}


\section{Author Prefixes (\tag{tags} tag)}
\label{tag:tags}

The \tag{tags} tag contains individual \tag{tag} tags defining each of the
reserved author prefixes used by extension and layer authors.

Each \tag{tag} tag contains information defining a single author prefix.

\subsection{Attributes of \tag{tag} tags}

\begin{itemize}
\item \attr{name} - required. The author prefix, as registered with Khronos.
      A short, upper-case string, usually an abbreviation of an author,
      project or company name.
\item \attr{author} - required. The author name, such as a full company
      or project name.
\item \attr{contact} - required. The contact who registered or is currently
      responsible for extensions and layers using the prefix, including
      sufficient contact information to reach the contact such as individual
      name together with email address, Github username, or other contact
      information.
\end{itemize}


\section{API types (\tag{types} tag)}
\label{tag:types}

The \tag{types} tag contains individual \tag{type} tags describing each
of the derived types used in the API.

Each \tag{type} tag contains information which can be used to generate C
code corresponding to the type. In many cases, this is simply legal C code
with attributes or embedded tags denoting the type name and other types used
in defining this type. In some cases, additional attribute and embedded type
information is used to generate more complicated C types.


\subsection{Attributes of \tag{type} tags}

\begin{itemize}
\item \attr{requires} - optional. Another type name this type requires to
      complete its definition.
\item \attr{name} - optional. Name of this type (if not defined in the tag
      body).
\item \attr{api} - optional. An API name (see \tag{feature} below) which
      specializes this definition of the named type, so that the same API
      types may have different definitions for e.g. GL ES and GL. This is
      unlikely to be used in Vulkan, where a single API supports desktop and
      mobile devices, but the functionality is retained.
\item \attr{category} - optional. A string which indicates that this type
      contains a more complex structured definition. At present the only
      accepted categories are \code{basetype} \code{bitmask}, \code{define},
      \code{enum}, \code{funcpointer}, \code{group}, \code{handle},
      \code{include}, \code{struct}, and \code{union}, as described below.
\item \attr{comment} - optional. Arbitrary string (unused).
\item \attr{parent} only applicable if category is \code{handle}. Notes another type with
      the \code{handle} category that acts as a parent object for this type.
\item \attr{returnedonly} only applicable if category is \code{struct} or \code{union}.
      Notes that this struct/union is going to be filled in by the API, rather than an
      application filling it out and passing it to the API.
\end{itemize}

\subsection{Contents of \tag{type} tags}

The valid contents depend on the \attr{category} attribute.

\subsubsection{Enumerated types - \attr{category} "enum"}

If the \attr{category} tag has the value \code{enum}, the type is a C
enumeration. The body of the tag is ignored in this case. The value of the
\attr{name} attribute must be provided and must match the \attr{name}
attribute of a \tag{enums} tag (see section~\ref{tag:enums}). The enumerant
values defined within the \tag{enums} tag are used to generate a C
\code{enum} type declaration.

\subsubsection{Structure types - \attr{category} "struct" or "union"}

If the \attr{category} tag has the values \code{struct} or \code{union}, the
type is a C structure or union, respectively. In this case, the \attr{name}
attribute must be provided, and the contents of the \tag{type} tag are a
series of \tag{member} tags defining the members of the aggregate type, in
order, followed by an optional \tag{validity} tag including asciidoc
validation language for the structure contents.


\paragraph{Structure member (\tag{member}) tags}
\par\vskip 11pt

The \tag{member} tag defines the type and name of a structure or union
member.

\paragraph{Attributes of \tag{member} tags}

\begin{itemize}
\item \attr{len} - if the member is an array, len may be one or more of the following
           things, separated by commas (one for each array indirection): another
           member of that struct, ``null-terminated'' for a string, ``1'' to indicate it's
           just a pointer (used for nested pointers), or a latex equation (prefixed with
           ``latex:'')
\item \attr{externsync} - denotes that the member should be externally synchronized
           when accessed by Vulkan
\item \attr{optional} - whether this value can be omitted by providing \code{NULL} (for
          pointers), \code{VK_NULL_HANDLE} (for handles) or 0 (for bitmasks/values)
\item \attr{noautovalidity} - prevents automatic validity language being generated
          for the tagged item. Only suppresses item-specific validity - parenting issues
          etc. are still captured.
\end{itemize}

\paragraph{Contents of \tag{member} tags}
\par\vskip 11pt

The text elements of a \tag{member} tag, with all other tags removed,
is a legal C declaration of a struct or union member. In addition it may
contain two semantic tags:

\begin{itemize}
\item The \tag{type} tag is optional. It contains text which is a valid type
      name found in another \tag{type} tag, and indicates that this type
      must be previously defined for the definition of the command to
      succeed. Builtin C types should not be wrapper in \tag{type} tags.
\item The \tag{name} tag is required, and contains the struct/union member
      name being described.
\end{itemize}

\paragraph{Validation (\tag{validity}) tags}
\par\vskip 11pt

The \tag{validity} tag, if present defines valid use cases and values for
structure members.

\paragraph{Contents of \tag{validity} tags}
\par\vskip 11pt

Each \tag{validity} tag contains zero or more \tag{usage} tags. Each
\tag{usage} tag is intended to represent a specific validation requirement
for the structure and include arbitrary asciidoc text describing that
requirement.


\subsubsection{All other types}

If the \attr{category} attribute is one of \code{basetype}, \code{bitfield},
\code{define}, \code{funcpointer}, \code{group}, \code{handle} or
\code{include}, or is not specified, \tag{type} contains text which is legal
C code for a type declaration. It may also contain embedded tags:

\begin{itemize}
\item \tag{type} - nested type tags contain other type names which are
      required by the definition of this type.
\item \tag{apientry/} - insert a platform calling convention macro here
      during header generation, used mostly for function pointer types.
\item \tag{name} - contains the name of this type (if not defined in the
      tag attributes).
\end{itemize}

There is no restriction on which sorts of definitions may be made in a given
category, although the contents of tags with \attr{category} \code{enum},
\code{struct} or \code{union} are interpreted specially as described above.

However, when generating the header, types within each category are grouped
together, and categories are generated in the order given by the following
list. Therefore, types in a category should correspond to the intended
purpose given for that category. If this recommendation is not followed, it
is possible that the resulting header file will not compile due to
out-of-order type dependencies.

\begin{itemize}
  \item \code{include} ({\tt \#include} directives)
  \item \code{define} (macro {\tt \#define} directives)
  \item \code{basetype} (scalar typedefs, such as the definition of
        {\tt VkFlags})
  \item \code{handle} (invocations of macros defining scalar types such as
        {\tt VkInstance})
  \item \code{enum} (enumeration types and {\tt \#define} for constant values)
  \item \code{group} (currently unused)
  \item \code{bitmask} (enumeration types whose members are bitmasks)
  \item \code{funcpointer} (function pointer typedefs)
  \item \code{struct} and \code{union} together (struct and union types)
\end{itemize}


\subsection{Example of a \tag{types} tag}
\label{tag:types:example}

\begin{verbatim}
<types>
    <type name="stddef">#include &lt;stddef.h&gt;</type>
    <type requires="stddef">typedef ptrdiff_t <name>VKlongint</name>;</type>
    <type name="VkEnum" category="enum"/>
    <type category="struct" name="VkStruct">
        <member><type>VkEnum</type> <name>srcEnum</name></member>
        <member><type>VkEnum</type> <name>dstEnum</name></member>
    </type>
</types>

<enums name="VkEnum" type="enum">
    <enum value="0" name="VK_ENUM_ZERO"/>
    <enum value="42" name="VK_ENUM_FORTY_TWO"/>
</enums>
\end{verbatim}

The {\tt VkStruct} type is defined to require the types {\tt VkEnum} and
{\tt VKlongint} as well. If {\tt VkStruct} is in turn required by a command
or another type during header generation, it will result in the following
declarations:

\begin{verbatim}
#include <stddef.h>
typedef ptrdiff_t VKlongint.

typedef enum {
    VK_ENUM_ZERO = 0,
    VK_ENUM_FORTY_TWO = 42
} VkEnum;

typedef struct {
    VkEnum    dstEnum;
    VkLongint dstVal;
} VkStruct;
\end{verbatim}

Note that the angle brackets around \code{stddef.h} are represented as XML
entities in the registry. This could also be done using a CDATA block but
unless there are many characters requiring special representation in XML,
using entities is preferred.


\section{Enumerant Blocks (\tag{enums} tag)}
\label{tag:enums}

The \tag{enums} tags contain individual \tag{enum} tags describing each of
the token names used in the API. In some cases these correspond to a C
\code{enum}, and in some cases they are simply compile-time constants (e.g.
\code{\#define}).

\subsection{Attributes of \tag{enums} tags}

\begin{itemize}
\item \attr{name} - optional. String naming the C \code{enum} type whose
      members are defined by this enum group. If present, this attribute
      should match the \attr{name} attribute of a corresponding \tag{type}
      tag.
\item \attr{type} - optional. String describing the data type of the values
      of this group of enums. At present the only accepted categories are
      \code{enum} and \code{bitmask}, as described below.
\item \attr{start}, \attr{end} - optional. Integers defining the start and
      end of a reserved range of enumerants for a particular vendor or
      purpose. \attr{start} must be $\leq$ \attr{end}. These fields define
      formal enumerant allocations, and are made by the Khronos Registrar on
      request from implementers following the enum allocation policy.
\item \attr{vendor} - optional. String describing the vendor or purpose to
      whom a reserved range of enumerants is allocated.
\item \attr{comment} - optional. Arbitrary string (unused).
\end{itemize}

\subsection{Contents of \tag{enums} tags}

Each \tag{enums} block contains zero or more \tag{enum} and \tag{unused}
tags, in arbitrary order (although they are typically ordered by sorting
on enumerant values, to improve human readability).

\subsection{Example of \tag{enums} tags}

An example showing a tag with attribute \attr{type}="\code{enum}" is given
above in section~\ref{tag:types:example}. The following example is for
non-enumerated tokens.

\begin{verbatim}
<enums>
    <enum value="256" name="VK_MAX_EXTENSION_NAME"/>
    <enum value="MAX_FLOAT"  name="VK_LOD_CLAMP_NONE"/>
</enums>
\end{verbatim}

When processed into a C header, and assuming all these tokens were
required, this results in

\begin{verbatim}
#define VK_MAX_EXTENSION_NAME   256
#define VK_LOD_CLAMP_NONE       MAX_FLOAT
\end{verbatim}


\section{Enumerants (\tag{enum} tag)}
\label{tag:enum}

Each \tag{enum} tag defines a single Vulkan (or other API) token.

\subsection{Attributes of \tag{enum} tags}

\begin{itemize}
\item \attr{value} or \attr{bitpos} - exactly one of these is allowed and
      required. \attr{value} is an enumerant value in the form of a legal C
      constant (usually a literal decimal or hexadecimal integer, though
      arbitrary strings are allowed). \attr{bitpos} is a literal integer bit
      position in a bitmask.
\item \attr{name} - required. Enumerant name, a legal C preprocessor token
      name.
\item \attr{api} - optional. An API name which specializes this definition of
      the named enum, so that different APIs may have different values for
      the same token. May be used to address a subtle incompatibilities.
\item \attr{type} - optional. Used only when \attr{value} is specified. C
      suffix for the value to force it to a specific type. Currently only
      \code{u} and \code{ull} are used, for \code{unsigned} 32- and 64-bit
      integer values, respectively. Separated from \attr{value} since this
      eases parsing and sorting of values, and rarely used.
\item \attr{alias} - optional. Name of another enumerant this is an alias
      of, used where token names have been changed as a result of profile
      changes or for consistency purposes. An enumerant alias is simply a
      different \attr{name} for the exact same \attr{value} or
      \attr{bitpos}.
\end{itemize}

\subsection{Contents of \tag{enum} tags}

\tag{enum} tags have no allowed contents. All information is contained
in the attributes.


\section{Unused Enumerants (\tag{unused} tag)}
\label{tag:unused}

Each \tag{unused} tag defines a range of enumerants which is allocated, but
not yet assigned to specific enums. This just tracks the unused values for
the Registrar's use, and is not used for header generation.

\subsection{Attributes of \tag{unused} tags}

\begin{itemize}
\item \attr{start} - required, \attr{end} - optional. Integers defining the
      start and end of an unused range of enumerants. \attr{start} must be
      $\leq$ \attr{end}. If \attr{end} is not present, then \attr{start}
      defines a single unused enumerant. This range should not exceed the
      range reserved by the surrounding \tag{enums} tag.
\item \attr{vendor} - optional. String describing the vendor or purposes to
      whom a reserved range of enumerants is allocated. Usually identical to
      the \attr{vendor} attribute of the surrounding \attr{enums} block.
\item \attr{comment} - optional. Arbitrary string (unused).
\end{itemize}

\subsection{Contents of \tag{unused} tags}

None.


\section{Command Blocks (\tag{commands} tag)}
\label{tag:commands}

The \tag{commands} tag contains definitions of each of the functions
(commands) used in the API.

\subsection{Attributes of \tag{commands} tags}

None.

\subsection{Contents of \tag{commands} tags}

Each \tag{commands} block contains zero or more \tag{command} tags, in
arbitrary order (although they are typically ordered by sorting on the
command name, to improve human readability).


\section{Commands (\tag{command} tag)}
\label{tag:command}

The \tag{command} tag contains a structured definition of a single API
command (function).

\subsection{Attributes of \tag{command} tags}

\begin{itemize}
\item \attr{queues} - optional. A string identifying the command queues
      this command can be placed on. The format of the string is one or
      more of the terms \code{"compute"}, \code{"dma"}, and
      \code{"graphics"}, with multiple terms separated by commas
      (\code{","}).
\item \attr{renderpass} - optional. A string identifying whether the
      command can be issued only inside a render pass (\code{"inside"}),
      only outside a render pass (\code{"outside"}), or both
      (\code{"both"}).
\item \attr{cmdbufferlevel} - optional. A string identifying the command buffer
      levels that this command can be called by. The format of the string is
      one or more of the terms \code{"primary"} and \code{"secondary"}, with
      multiple terms separated by commas (\code{","}).
\item \attr{comment} - optional. Arbitrary string (unused).
\end{itemize}

\subsection{Contents of \tag{command} tags}

\begin{itemize}
\item \tag{proto} is required and must be the first element. It is a tag
      defining the C function prototype of a command as described below, up
      to the function name and return type but not including function
      parameters.
\item \tag{param} elements for each command parameter follow, defining
      its name and type, as described below. If a command takes no
      arguments, it has no \tag{param} tags.
\item An optional \tag{validity} tag including asciidoc validation language
      for the command parameters.
\end{itemize}

Following these elements, the remaining elements in a \tag{command}
tag are optional and may be in any order:

\begin{itemize}
\item \tag{alias} - optional. Has no attributes and contains a string which
      is the name of another command this command is an alias of, used when
      promoting a function from vendor to Khronos extension or Khronos
      extension to core API status. A command alias describes the case where
      there are two function names which resolve to the {\bf same} entry
      point in the underlying layer stack.
\item \tag{description} - optional. Unused text.
\item \tag{implicitexternsyncparams} - optional. Contains a list of
      \tag{param} tags, each containing Asciidoc source text describing an
      object which is not a parameter of the command, but is related to one,
      and which also requires external synchronization as described in
      section~\ref{tag:command:param:attr}. The text is intended to be
      incorporated into the API specification.
\end{itemize}


\subsection{Command prototype (\tag{proto} tags)}
\label{tag:command:proto}

The \tag{proto} tag defines the return type and name of a command.

\subsubsection{Attributes of \tag{proto} tags}

None.

% \begin{itemize}
% \item \attr{group} - group name, an arbitrary string.
% \end{itemize}
%
% If the group name is defined, it may be interpreted as described in
% section~\ref{tag:group:meaning}.

\subsubsection{Contents of \tag{proto} tags}

The text elements of a \tag{proto} tag, with all other tags removed, is
legal C code describing the return type and name of a command. In addition
to text, it may contain two semantic tags:

\begin{itemize}
\item The \tag{type} tag is optional, and contains text which is a valid
      type name found in a \tag{type} tag. It indicates that this type must
      be previously defined for the definition of the command to succeed.
      Builtin C types, and any derived types which are expected to be found
      in other header files, should not be wrapped in \tag{type} tags.
\item The \tag{name} tag is required, and contains the command name
      being described.
\end{itemize}

\subsection{Command parameter (\tag{param} tags)}
\label{tag:command:param}

The \tag{param} tag defines the type and name of a parameter. Its contents
are very similar to the \tag{member} tag used to define struct and union
members.

\subsubsection{Attributes of \tag{param} tags}
\label{tag:command:param:attr}

\begin{itemize}
\item \attr{len} - if the param is an array, len may be one or more of the following
           things, separated by commas (one for each array indirection): another
           param of that command, ``null-terminated'' for a string, ``1'' to indicate it's
           just a pointer (used for nested pointers), or a latex equation (prefixed with
           ``latex:'')
\item \attr{optional} - whether this value can be omitted by providing \code{NULL} (for
          pointers), \code{VK_NULL_HANDLE} (for handles) or 0 (for bitmasks/values)
\item \attr{noautovalidity} -  prevents automatic validity language being generated
          for the tagged item. Only suppresses item-specific validity - parenting issues
          etc. are still captured.
\item \attr{externsync} - optional. A boolean string, which must have the
      value \code{"true"} if present, indicating that this parameter (e.g.
      the object a handle refers to, or the contents of an array a pointer
      refers to) is modified by the command, and is not protected against
      modification in multiple app threads. Parameters which do not have
      this attribute are assumed to not require external synchronization.
\end{itemize}

\subsubsection{Contents of \tag{param} tags}

The text elements of a \tag{param} tag, with all other tags removed, is
legal C code describing the type and name of a function parameter. In
addition it may contain two semantic tags:

\begin{itemize}
\item The \tag{type} tag is optional, and contains text which is a
      valid type name found in \tag{type} tag, and indicates that this
      type must be previously defined for the definition of the command
      to succeed. Builtin C types, and any derived types which are
      expected to be found in other header files, should not be wrapped
      in \tag{type} tags.
\item The \tag{name} tag is required, and contains the parameter name being
      described.
\end{itemize}

\subsection{Example of a \tag{commands} tag}

\begin{verbatim}
<commands>
    <command>
        <proto><type>VkResult</type> <name>vkCreateInstance</name></proto>
        <param>const <type>VkInstanceCreateInfo</type>* <name>pCreateInfo</name></param>
        <param><type>VkInstance</type>* <name>pInstance</name></param>
    </command>
</commands>
\end{verbatim}

When processed into a C header, this results in

\begin{verbatim}
VkResult vkCreateInstance(
    const VkInstanceCreateInfo* pCreateInfo,
    VkInstance* pInstance);
\end{verbatim}

\subsection{Parameter validation (\tag{validity}) tags}
\label{tag:command:validity}

The \tag{validity} tag, if present defines valid use cases and values for
command parameters

\paragraph{Contents of \tag{validity} tags}
\par\vskip 11pt

Each \tag{validity} tag contains zero or more \tag{usage} tags. Each
\tag{usage} tag is intended to represent a specific validation requirement
for the command, and contains arbitrary asciidoc text describing that
requirement.


\section{API Features / Versions (\tag{feature} tag)}
\label{tag:feature}

API features are described in individual \tag{feature} tags. A feature is
the set of interfaces (enumerants and commands) defined by a particular API
and version, such as Vulkan 1.0, and includes all profiles of that API and
version.

\subsection{Attributes of \tag{feature} tags}

\begin{itemize}
\item \attr{api} - required. API name this feature is for (see
      section~\ref{schema:apiname}), such as \code{vk}.
\item \attr{name} - required. Version name, used as the C preprocessor token
      under which the version's interfaces are protected against multiple
      inclusion. Example: \code{VK\_VERSION\_1\_0}.
\item \attr{number} - required. Feature version number, usually a string
      interpreted as $majorNumber.minorNumber$. Example: \code{4.2}.
\item \attr{protect} - optional. An additional preprocessor token used to
      protect a feature definition. Usually another feature or extension
      \attr{name}. Rarely used, for odd circumstances where the definition
      of a feature or extension requires another to be defined first.
\item \attr{comment} - optional. Arbitrary string (unused).
\end{itemize}

\subsection{Contents of \tag{feature} tags}

Zero or more \tag{require} and \tag{remove} tags (see
section~\ref{tag:required}), in arbitrary order. Each tag describes a
set of interfaces that is respectively required for, or removed from,
this feature, as described below.

\subsection{Example of a \tag{feature} tag}

\begin{verbatim}
<feature api="vulkan" name="VK_VERSION_1_0" number="1.0">
    <require comment="Header boilerplate">
        <type name="vk_platform"/>
    </require>
    <require comment="API constants">
        <enum name="VK_MAX_PHYSICAL_DEVICE_NAME"/>
        <enum name="VK_LOD_CLAMP_NONE"/>
    </require>
    <require comment="Device initialization">
        <command name="vkCreateInstance"/>
    </require>
</feature>
\end{verbatim}

When processed into a C header for Vulkan, this results in:

\begin{verbatim}
#ifndef VK_VERSION_1_0
#define VK_VERSION_1_0 1
#define VK_MAX_EXTENSION_NAME   256
#define VK_LOD_CLAMP_NONE       MAX_FLOAT
typedef VkResult (VKAPI_PTR *PFN_vkCreateInstance)(const VkInstanceCreateInfo* pCreateInfo, VkInstance* pInstance);
#ifdef VK_PROTOTYPES
VKAPI_ATTR VkResult VKAPI_CALL vkCreateInstance(
    const VkInstanceCreateInfo*                 pCreateInfo,
    VkInstance*                                 pInstance);
#endif
#endif /* VK_VERSION_1_0 */
\end{verbatim}


\section{Extension Blocks (\tag{extensions} tag)}
\label{tag:extensions}

The \tag{extensions} tag contains definitions of each of the extenions
which are defined for the API.

\subsection{Attributes of \tag{extensions} tags}

None.

\subsection{Contents of \tag{extensions} tags}

Each \tag{extensions} block contains zero or more \tag{extension} tags,
each describing an API extension, in arbitrary order (although they are
typically ordered by sorting on the extension name, to improve human
readability).


\section{API Extensions (\tag{extension} tag)}
\label{tag:extension}

API extensions are described in individual \tag{extension} tags. An
extension is the set of interfaces defined by a particular API extension
specification, such as \code{ARB\_multitexture}. \tag{extension} is
similar to \tag{feature}, but instead of having \attr{version} and
\attr{profile} attributes, instead has a \attr{supported} attribute,
which describes the set of API names which the extension can potentially
be implemented against.

\subsection{Attributes of \tag{extension} tags}

\begin{itemize}
\item \attr{name} - required. Extension name, following the conventions
      in the Vulkan Specification.
      Example: \code{name="VK\_VERSION\_1\_0"}.
\item \attr{number} - required. A decimal number which is the registered,
      unique extension number for \attr{name}.
\item \attr{supported} - required. A regular expression, with an implicit
      \code{\^{}} and \code{\$} bracketing it, which should match the
      \attr{api} tag of a set of \tag{feature} tags.
\item \attr{protect} - optional. An additional preprocessor token used to
      protect an extension definition. Usually another feature or extension
      \attr{name}. Rarely used, for odd circumstances where the definition
      of an extension requires another extension or a header file to be
      defined first.
\item \attr{author} - optional. The author name, such as a full company
      name. If not present, this can be taken from the corresponding
      \tag{tag} attribute. However, \code{EXT} and other multi-vendor
      extensions may not have a well-defined author or contact in the tag.
\item \attr{contact} - optional. The contact who registered or is currently
      responsible for extensions and layers using the tag, including
      sufficient contact information to reach the contact such as individual
      name together with email address, Github username, or other contact
      information. If not present, this can be taken from the corresponding
      \tag{tag} attribute just like \attr{author}.
\item \attr{comment} - optional. Arbitrary string (unused).
\end{itemize}

\subsection{Contents of \tag{extension} tags}

Zero or more \tag{require} and \tag{remove} tags (see
section~\ref{tag:required}), in arbitrary order. Each tag describes a
set of interfaces that is respectively required for, or removed from,
this extension, as described below.

\subsection{Example of an \tag{extensions} tag}

\begin{verbatim}
<extension name="VK_KHR_display_swapchain" number="4" supported="vulkan">
    <require>
        <enum value="9" name="VK_KHR_DISPLAY_SWAPCHAIN_SPEC_VERSION"/>
        <enum value="4" name="VK_KHR_DISPLAY_SWAPCHAIN_EXTENSION_NUMBER"/>
        <enum value="&quot;VK_KHR_display_swapchain&quot;"
              name="VK_KHR_DISPLAY_SWAPCHAIN_EXTENSION_NAME"/>
        <type name="VkDisplayPresentInfoKHR"/>
        <command name="vkCreateSharedSwapchainsKHR"/>
    </require>
</extension>
\end{verbatim}

The \attr{supported} attribute says that the extension is defined for the
default profile (\code{vulkan}). When processed into a C header for the
\code{vulkan} profile, this results in header contents something like
(assuming corresponding definitions of the specified \tag{type} and
\tag{command} elsewhere in the XML):

\begin{verbatim}
#define VK_KHR_display_swapchain 1
#define VK_KHR_DISPLAY_SWAPCHAIN_SPEC_VERSION 9
#define VK_KHR_DISPLAY_SWAPCHAIN_EXTENSION_NUMBER 4
#define VK_KHR_DISPLAY_SWAPCHAIN_EXTENSION_NAME "VK_KHR_display_swapchain"

typedef struct VkDisplayPresentInfoKHR {
    VkStructureType                             sType;
    const void*                                 pNext;
    VkRect2D                                    srcRect;
    VkRect2D                                    dstRect;
    VkBool32                                    persistent;
} VkDisplayPresentInfoKHR;

typedef VkResult (VKAPI_PTR *PFN_vkCreateSharedSwapchainsKHR)(
    VkDevice device, uint32_t swapchainCount,
    const VkSwapchainCreateInfoKHR* pCreateInfos,
    const VkAllocationCallbacks* pAllocator,
    VkSwapchainKHR* pSwapchains);

#ifdef VK_PROTOTYPES
VKAPI_ATTR VkResult VKAPI_CALL vkCreateSharedSwapchainsKHR(
    VkDevice                                    device,
    uint32_t                                    swapchainCount,
    const VkSwapchainCreateInfoKHR*             pCreateInfos,
    const VkAllocationCallbacks*                pAllocator,
    VkSwapchainKHR*                             pSwapchains);
#endif
\end{verbatim}


\section{Required and Removed Interfaces (\tag{require} and \tag{remove} tags)}
\label{tag:required}

A \tag{require} block defines a set of interfaces (types, enumerants and
commands) {\em required} by a \tag{feature} or \tag{extension}. A
\tag{remove} block defines a set of interfaces {\em removed} by a
\tag{feature} (this is primarily useful for future profiles of an API which
may choose to deprecated and/or remove some interfaces - extensions should
never remove interfaces, although this usage is allowed by the schema).
Except for the tag name and behavior, the contents of \tag{require} and
\tag{remove} tags are identical.

\subsection{Attributes of \tag{require} and \tag{remove} tags}

\begin{itemize}
\item \attr{profile} - optional. String name of an API profile. Interfaces
      in the tag are only required (or removed) if the specified profile is
      being generated. If not specified, interfaces are required (or
      removed) for all API profiles.
\item \attr{comment} - optional. Arbitrary string (unused).
\item \attr{api} - optional. An API name (see section~\ref{schema:apiname}).
      Interfaces in the tag are only required (or removed) if the specified
      API is being generated. If not specified, interfaces are required (or
      removed) for all APIs.

      {\bf The \attr{api} attribute is only supported inside
      \tag{extension} tags,} since \tag{feature} tags already define a
      specific API.
\end{itemize}

\subsection{Contents of \tag{require} and \tag{remove} tags}

Zero or more of the following tags, in any order:

\begin{itemize}
\item \tag{command} specifies an required (or removed) command defined
      in a \tag{commands} block. The tag has no content, but contains
      attributes:

      \begin{itemize}
      \item \attr{name} - required. Name of the command.
      \item \attr{comment} - optional. Arbitrary string (unused).
      \end{itemize}
\item \tag{enum} specifies an required (or removed) enumerant defined in a
      \tag{enums} block. All forms of this tag support the following
      attributes:

      \begin{itemize}
      \item \attr{name} - required. Name of the enumerant.
      \item \attr{comment} - optional. Arbitrary string (unused).
      \end{itemize}

      There are two forms of \tag{enum} tags:

      {\bf Reference enums} simply pull in the definition of an enumerant
      given in a separate \tag{enums} block. Reference enums are the most
      common usage, and no attributes other than \attr{name} and
      \attr{comment} are supported for them.

      {\bf Extension enums} define the value of an enumerant inline in an
      \tag{extensions} block. There are several variants, depending on which
      additional tags are defined:

      \begin{itemize}
      \item Attributes \attr{value} and (optionally) \attr{type} define a
            constant value in the same fashion as an \tag{enum} tag in an
            \tag{enums} block (see section~\ref{tag:enum}).
      \item Attribute \attr{bitpos} defines a constant bitmask value in the
            same fashion as an \tag{enum} tag in an \tag{enums} block (see
            section~\ref{tag:enum}).
      \item Attribute \attr{extends} allows an extension enumerant to be
            added to a separately defined enumerated type whose name is
            specified by the contents of the \attr{extends} attribute (e.g.
            a \tag{type} tag with \attr{category} "enum", pulling in an
            \tag{enums} block).

            There are two ways to extend an enumerated type:

            \begin{itemize}
            \item If \attr{bitpos} is also specified, the tag defines a
                  bitmask value, but adds its definition to the enumeranted
                  type specified by \attr{extends} instead of as a
                  compile-time constant.
            \item If \attr{offset} is also specified, the tag adds a new
                  enumerant value to the enumerated type specified by
                  \attr{extends}. The actual value defined depends on the
                  extension number (the \attr{number} attribute of the
                  \tag{extensions} tag) and the offset, as defined in the
                  ``Layers and Extensions'' appendix of the Vulkan
                  Specification. The \attr{dir} attribute may also be
                  specified as {\tt dir="-"} if the calculated value should
                  be negative instead of positive. Negative enumerant values
                  are normally used only for Vulkan error codes.
            \end{itemize}
      \end{itemize}

      Examples of various types of extension enumerants are given below in
      section~\ref{tag:required:examples}.
\item \tag{type} specifies a required (or removed) type defined in a
      \tag{types} block. Most types are picked up implicitly by using the
      \tag{type} tags of commands, but in a few cases, additional types need
      to be specified explicitly. It is unlikely that a type would ever be
      removed, although this usage is allowed by the schema. The tag has no
      content, but contains elements:

      \begin{itemize}
      \item \attr{name} - required. Name of the type.
      \item \attr{comment} - optional. Arbitrary string (unused).
      \end{itemize}
\end{itemize}


\subsection{Examples of Extension Enumerants}
\label{tag:required:examples}

Examples of each of the supported extension enumerant \tag{enum} tags are
given below. Note that extension enumerants are supported only inside
\tag{extension} blocks - not in \tag{feature} blocks\footnote{
    However, we will have to define additional XML tags and/or syntax for
    future core versions of Vulkan, to properly tag and group core
    enumerants for Vulkan 1.0, 1.1, etc.}.

\begin{verbatim}
<extensions>
    <extension name="VK_KHR_test_extension" number="1" supported="vulkan">
        <require>
            <enum value="42" name="VK_KHR_theanswer"/>
            <enum bitpos="29" name="VK_KHR_bitmask"/>
            <enum offset="0" dir="-" extends="VkResult"
                  name="VK_ERROR_SURFACE_LOST_KHR"/>
            <enum offset="1" extends="VkResult"
                  name="VK_SUBOPTIMAL_KHR"/>
            <enum bitpos="31" extends="VkResult"
                  name="VK_KHR_EXTENSION_BITFIELD"/>
        </require>
    </extension>
</extensions>
\end{verbatim}

The corresponding header file will include definitions like this:

\begin{verbatim}
typedef enum VkResult {
    <previously defined VkResult enumerant values},
    VK_ERROR_SURFACE_LOST_KHR = -1000000000,
    VK_SUBOPTIMAL_KHR = 1000000001,
    VK_KHR_EXTENSION_BITFIELD = 0x80000000,
};

#define VK_KHR_test_extension 1
#define VK_KHR_theanswer 42
#define VK_KHR_bitmask 0x20000000
\end{verbatim}

\clearpage
\appendix

\section{Examples / FAQ / How Do I?}
\label{examples}

For people new to the Registry, it won't be immediately obvious how to make
changes. This section includes some tips and examples that will help you
make changes to the Vulkan headers by changing the Registry XML description.

First, follow the steps in section~\ref{starting} to get the Vulkan Gitlab
repository containing the registry and assemble the tools necessary to work
with the XML registry. Once you're able to regenerate \code{vulkan.h} from
\code{vk.xml}, you can start making changes.


\subsection{General Strategy}

If you are {\bf adding} to the API, perform the following steps to {\bf
create} the description of that API element:

\begin{itemize}
\item For each type, enum group, compile-time constant, and command being
      added, create appropriate new \tag{type}, \tag{enums}, \tag{enum}, or
      \tag{command} tags defining the interface in question.
\item Make sure that all added types and commands appropriately tag their
      dependencies on other types by adding nested \tag{type} tags.
\item Make sure that each new tag defines the name of the corresponding
      type, enum group, constant, or command, and that structure/union types
      and commands tag the types and names of all their members and
      parameters. This is essential for the automatic dependency process to
      work.
\end{itemize}

If you are {\bf modifying} existing APIs, just make appropriate changes
in the existing tags.

Once the definition is added, proceed to the next section to create
dependencies on the changed feature.


\subsection{API Feature Dependencies}

When you add new API elements, they will not result in corresponding changes
in the generated header unless they are {\bf required} by the interface
being generated. This makes it possible to include different API versions
and extensions in a single registry and pull them out as needed. So you must
introduce a dependency on new features in the corresponding \tag{feature}
tag.

Initially, the only API feature is Vulkan 1.0, so there is only one
\tag{feature} tag in \code{vk.xml}. You can find it by searching
for the following block of \code{vk.xml}:

\begin{verbatim}
<!-- SECTION: Vulkan API interface definitions -->
<feature api="vulkan" name="VK_VERSION_1_0" number="1.0">
\end{verbatim}

Inside the \tag{feature} tag are nested multiple \tag{require} tags. These
are just being used as a logical grouping mechanism for related parts of
Vulkan 1.0 at present, though they may have more meaningful roles in the
future if different API profiles are defined.


\subsubsection{API Feature Walkthrough}

This section walks through the first few required API features in the
\code{vk.xml} \tag{feature} tag, showing how each requirement pulls in type,
token, and command definitions and turns those into definitions in the C
header file \code{vulkan.h} .

Consider the first few lines of the \tag{feature}:

\begin{verbatim}
<require comment="Header boilerplate">
    <type name="vk_platform"/>
</require>
<require comment="API constants">
    <enum name="VK_MAX_PHYSICAL_DEVICE_NAME"/>
    <enum name="VK_MAX_EXTENSION_NAME"/>
    ...
</require>
<require comment="Device initialization">
    <command name="vkCreateInstance"/>
    ...
\end{verbatim}

The first \tag{require} block says to require a type named
\code{vk_platform}. If you look at the beginning of the \tag{types} section,
there's a corresponding definition:


\begin{verbatim}
<type name="vk_platform">#include "vk_platform.h"
#define VK_MAKE_VERSION(major, minor, patch) \
    ((major &lt;&lt; 22) | (minor &lt;&lt; 12) | patch)
    ...
\end{verbatim}

\noindent section which is invoked by the requirement and emits a bunch of
boilerplate C code. The explicit dependency isn't strictly required since
\code{vk_platform} will be required by many other types, but placing it
first causes this to appear first in the output file.

Note that \code{vk_platform} does not correspond to an actual C type, but
instead to a collection of freeform preprocessor includes and macros and
comments. Most other \tag{type} tags do define a specific type and are much
simpler, but this approach can be used to inject arbitrary C into
\code{vulkan.h} {\bf when there's no other way}. In general inserting
arbitrary C is strongly discouraged outside of specific special cases like
this.

The next \tag{require} block pulls in some compile-time constants. These
correspond to the definitions found in the first \tag{enums} section of
\code{vk.xml}:

\begin{verbatim}
<!-- SECTION: Vulkan enumerant (token) definitions. -->

<enums comment="Misc. hardcoded constants - not an enumerated type">
        <!-- This is part of the header boilerplate -->
    <enum value="256"        name="VK_MAX_PHYSICAL_DEVICE_NAME"/>
    <enum value="256"        name="VK_MAX_EXTENSION_NAME"/>
    ...
\end{verbatim}

The third \tag{require} block starts pulling in some Vulkan commands. The
first command corresponds to the following definition found in the
\tag{commands} section of \code{vk.xml}:

\begin{verbatim}
<commands>
    <command>
        <proto><type>VkResult</type> <name>vkCreateInstance</name></proto>
        <param>const <type>VkInstanceCreateInfo</type>* <name>pCreateInfo</name></param>
        <param><type>VkInstance</type>* <name>pInstance</name></param>
    </command>
    ...
\end{verbatim}

In turn, the \tag{command} tag requires the \tag{type}s \code{VkResult},
\code{VkInstanceCreateInfo}, and \code{VkInstance} as part of its
definition. The definitions of these types are determined as follows:

For \code{VkResult}, the corresponding required \tag{type} is:

\begin{verbatim}
<type name="VkResult" category="enum"/>
\end{verbatim}

\noindent Since this is an enumeration type, it simply links to an
\tag{enums} tag with the same name:

\begin{verbatim}
<enums name="VkResult" type="enum" expand="VK_RESULT" comment="Error and return codes">
    <!-- Return codes for successful operation execution (positive values) -->
    <enum value="0"     name="VK_SUCCESS"/>
    <enum value="1"     name="VK_UNSUPPORTED"/>
    <enum value="2"     name="VK_NOT_READY"/>
    ...
\end{verbatim}

For \tag{VkInstanceCreateInfo}, the required \tag{type} is:

\begin{verbatim}
<type category="struct" name="VkInstanceCreateInfo">
    <member><type>VkStructureType</type>        <name>sType</name></member>                          <!-- Should be VK_STRUCTURE_TYPE_INSTANCE_CREATE_INFO -->
    <member>const void*                         <name>pNext</name></member>                          <!-- Pointer to next structure -->
    <member>const <type>VkApplicationInfo</type>* <name>pAppInfo</name></member>
    <member>const <type>VkAllocCallbacks</type>* <name>pAllocCb</name></member>
    <member><type>uint32_t</type>               <name>extensionCount</name></member>
    <member>const <type>char</type>*const*      <name>ppEnabledExtensionNames</name></member>        <!-- layer or extension name to be enabled -->
</type>
\end{verbatim}

\noindent This is a structure type, defining a C \code{struct} with all the
members defined in each \tag{member} tag in order. In addition, it requires
some other types, whose definitions are located by name in exactly the same
fashion.

For the final direct dependency of the command, \code{VkInstance}, the
required \tag{type} is:

\begin{verbatim}
<!-- Types which can be void pointers or class pointers, selected at compile time -->
<type>VK_DEFINE_BASE_HANDLE(<name>VkObject</name>)</type>
<type>VK_DEFINE_DISP_SUBCLASS_HANDLE(<name>VkInstance</name>, <type>VkObject</type>)</type>
\end{verbatim}

\noindent In this case, the type \code{VkInstance} is defined by a special
compile-time macro which defines it as a derived class of \code{VkObject}
(for C++) or a less typesafe definition for (for C). This macro isn't part
of the type dependency analysis, just the boilerplate used in the header.

If these are the only \tag{feature} dependencies in \code{vk.xml}, the
resulting \code{vulkan.h} header will look like this:

\begin{verbatim}

#ifndef __vulkan_h_
#define __vulkan_h_ 1

#ifdef __cplusplus
extern "C" {
#endif

/*
** Copyright (c) 2015-2016 The Khronos Group Inc.
    ...
*/

/*
** This header is generated from the Khronos Vulkan XML API Registry.
**
** Generated on date 20160208
*/


#define VK_VERSION_1_0 1
#include "vk_platform.h"
#define VK_MAKE_VERSION(major, minor, patch) \
    ((major << 22) | (minor << 12) | patch)

// Vulkan API version supported by this file
#define VK_API_VERSION VK_MAKE_VERSION(0, 104, 0)

#if defined (__cplusplus) && (VK_UINTPTRLEAST64_MAX == UINTPTR_MAX)
    #define VK_TYPE_SAFE_COMPATIBLE_HANDLES 1
#endif

#if defined(VK_TYPE_SAFE_COMPATIBLE_HANDLES) && !defined(VK_DISABLE_TYPE_SAFE_HANDLES)
    #define VK_DEFINE_PTR_HANDLE(_obj) struct _obj##_T { char _dummy; }; typedef _obj##_T* _obj;
    #define VK_DEFINE_PTR_SUBCLASS_HANDLE(_obj, _base) struct _obj##_T : public _base##_T {}; typedef _obj##_T* _obj;

    #define VK_DEFINE_BASE_HANDLE(_obj) VK_DEFINE_PTR_HANDLE(_obj)
    #define VK_DEFINE_DISP_SUBCLASS_HANDLE(_obj, _base) VK_DEFINE_PTR_SUBCLASS_HANDLE(_obj, _base)
    #define VK_DEFINE_NONDISP_SUBCLASS_HANDLE(_obj, _base) VK_DEFINE_PTR_SUBCLASS_HANDLE(_obj, _base)
#else
    #define VK_DEFINE_BASE_HANDLE(_obj) typedef VkUintPtrLeast64 _obj;
    #define VK_DEFINE_DISP_SUBCLASS_HANDLE(_obj, _base) typedef uintptr_t _obj;
    #define VK_DEFINE_NONDISP_SUBCLASS_HANDLE(_obj, _base) typedef VkUintPtrLeast64 _obj;
#endif

typedef enum {
    VK_SUCCESS = 0,
    VK_UNSUPPORTED = 1,
    VK_NOT_READY = 2,
    ...
} VkResult;
typedef enum {
    VK_STRUCTURE_TYPE_APPLICATION_INFO = 0,
    ...
} VKStructureType;
typedef struct {
    VkStructureType                             sType;
    const void*                                 pNext;
    const char*                                 pAppName;
    uint32_t                                    appVersion;
    const char*                                 pEngineName;
    uint32_t                                    engineVersion;
    uint32_t                                    apiVersion;
} VkApplicationInfo;
typedef enum {
    VK_SYSTEM_ALLOC_TYPE_API_OBJECT = 0,
    ...
} VkSystemAllocType;
typedef void* (VKAPI_PTR *PFN_vkAllocFunction)(
    void*                           pUserData,
    size_t                          size,
    size_t                          alignment,
    VkSystemAllocType               allocType);
typedef void (VKAPI_PTR *PFN_vkFreeFunction)(
    void*                           pUserData,
    void*                           pMem);
typedef struct {
    void*                                       pUserData;
    PFN_vkAllocFunction                         pfnAlloc;
    PFN_vkFreeFunction                          pfnFree;
} VkAllocCallbacks;
typedef struct {
    VkStructureType                             sType;
    const void*                                 pNext;
    const VkApplicationInfo*                    pAppInfo;
    const VkAllocCallbacks*                     pAllocCb;
    uint32_t                                    extensionCount;
    const char*const*                           ppEnabledExtensionNames;
} VkInstanceCreateInfo;
VK_DEFINE_BASE_HANDLE(VkObject)
VK_DEFINE_DISP_SUBCLASS_HANDLE(VkInstance, VkObject)
#define VK_MAX_PHYSICAL_DEVICE_NAME       256
#define VK_MAX_EXTENSION_NAME             256
typedef VkResult (VKAPI_PTR *PFN_vkCreateInstance)(const VkInstanceCreateInfo* pCreateInfo, VkInstance* pInstance);
#ifdef VK_PROTOTYPES
VKAPI_ATTR VkResult VKAPI_CALL vkCreateInstance(
    const VkInstanceCreateInfo*                 pCreateInfo,
    VkInstance*                                 pInstance);
#endif

#ifdef __cplusplus
}
#endif

#endif
\end{verbatim}

Note that several additional types are pulled in by the type dependency
analysis, but only those types, commands, and tokens required by the
specified features are generated.


\subsection{How To Add A Compile-Time Constant}

Go to the \tag{feature} tag and search for the nested block labelled

\begin{verbatim}
<require comment="API constants">
\end{verbatim}

In this block, add an (appropriately indented) tag like

\begin{verbatim}
    <enum name="VK_THE_ANSWER"/>
\end{verbatim}

Then go to the \tag{enums} block labelled

\begin{verbatim}
<enums comment="Misc. hardcoded constants - not an enumerated type">
\end{verbatim}

In this block, add a tag whose \attr{name} attribute matches the \attr{name}
you defined above and whose \attr{value} attribute is the value to give the
constant:

\begin{verbatim}
    <enum value="42"    name="VK_THE_ANSWER"/>
\end{verbatim}


\subsection{How To Add A Struct or Union Type}

For this example, assume we want to define a type
corresponding to a C \code{struct} defined as follows:

\begin{verbatim}
typedef struct {
    VkStructureType          sType;
    const void*              pNext;
    const VkApplicationInfo* pAppInfo;
    const VkAllocCallbacks*  pAllocCb;
    uint32_t                 extensionCount;
    const char*const*        ppEnabledExtensionNames;
} VkInstanceCreateInfo;
\end{verbatim}

If \code{VkInstanceCreateInfo} is the type of a parameter of a command in
the API, make sure that command's definition (see below for how to add a
command) puts \code{VkInstanceCreateInfo} in nested \tag{type} tags where
it's used.

Otherwise, if the struct type is not used directly by a command in the API,
nor required by a chain of type dependencies for other commands, an explicit
\tag{type} dependency should be added to the \tag{feature} tag. Go to the
\tag{types} tag and search for the nested block labelled

\begin{verbatim}
<require comment="API types not used by commands">
    ...
\end{verbatim}

In this block, add a tag whose \attr{name} attribute matches the \attr{name}
of the struct type being defined:

\begin{verbatim}
<require comment="API types not used by commands">
    <type name="VkInstanceCreateInfo"/>
    ...
\end{verbatim}

Then go to the \tag{types} tag and add a new \tag{type} tag defining
the struct names and members, somewhere below
the corresponding comment, like this:

\begin{verbatim}
<types>
    ...
    <!-- Struct types -->
        <type category="struct" name="VkInstanceCreateInfo">
            <member><type>VkStructureType</type>
                    <name>sType</name></member>
            <member>const void*
                    <name>pNext</name></member>
            <member>const <type>VkApplicationInfo</type>*
                    <name>pAppInfo</name></member>
            <member>const <type>VkAllocCallbacks</type>*
                    <name>pAllocCb</name></member>
            <member><type>uint32_t</type>
                    <name>extensionCount</name></member>
            <member>const <type>char</type>*const*
                    <name>ppEnabledExtensionNames</name></member>
        </type>
        ...
\end{verbatim}

If any of the member types are types also defined in the header, make sure
to enclose those type names in nested \tag{type} tags, as shown above. Basic
C types should not be tagged.

If the type is a C \code{union}, rather than a \code{struct}, then set the
value of the \attr{category} attribute to \code{"union"} instead of
\code{"struct"}.


\subsection{How To Add An Enumerated Type}

For this example, assume we want to define a type corresponding to a C
\code{enum} defined as follows:

\begin{verbatim}
typedef enum {
    VK_DEVICE_CREATE_VALIDATION_BIT = 0x00000001,
    VK_DEVICE_CREATE_MULTI_DEVICE_IQ_MATCH_BIT = 0x00000002;
} VkDeviceCreateFlagBits.
\end{verbatim}

If \code{VkDeviceCreateFlagBits} is the type of a parameter to a command in
the API, or of a member in a structure or union, make sure that command
parameter or struct member's definition puts \code{VkDeviceCreateFlagBits}
in nested \tag{type} tags where it's used.

Otherwise, if the enumerated type is not used directly by a command in the
API, nor required by a chain of type dependencies for commands and structs,
an explicit \tag{type} dependency should be added to the \tag{feature} tag
in exactly the same fashion as described above for \code{struct} types.

Next, go to the line labelled

\begin{verbatim}
<!-- SECTION: Vulkan enumerant (token) definitions. -->
\end{verbatim}

Below this line, add an \tag{enums} tag whose \attr{name} attribute matches
the \tag{type} name \code{VkDeviceCreateFlagBits}, and whose contents
correspond to the individual fields of the enumerated type:

\begin{verbatim}
<enums name="VkDeviceCreateFlagBits" type="bitmask">
    <enum bitpos="0" name="VK_DEVICE_CREATE_VALIDATION_BIT"/>
    <enum bitpos="1" name="VK_DEVICE_CREATE_MULTI_DEVICE_IQ_MATCH_BIT"/>
</enums>
\end{verbatim}

Several other attributes of the \tag{enums} tag can be set. In this case,
the \attr{type} attribute is set to \code{"bitmask"}, indicating that the
individual enumerants represent elements of a bitmask.

The individual \tag{enum} tags define the enumerants, just like the
definition for compile-time constants described above. In this case, because
the enumerants are bitfields, their values are specified using the
\attr{bitpos} attribute. The value of this attribute must be an integer in
the range $[0,31]$ specifying a single bit number, and the resulting value
is printed as a hexadecimal constant corresponding to that bit.

It is also possible to specify enumerant values using the \attr{value}
attribute, in which case the specified numeric value is passed through to
the C header unchanged.


\subsection{How to Add A Command}

For this example, assume we want to define the command:

\begin{verbatim}
VKAPI_ATTR VkResult VKAPI_CALL vkCreateInstance(
    const VkInstanceCreateInfo*                 pCreateInfo,
    VkInstance*                                 pInstance);
\end{verbatim}

Commands must always be explicitly required in the \tag{feature} tag. In
that tag, you can use an existing \tag{require} block including API features
which the new command should be grouped with, or define a new block. For
this example, add a new block, and require the command by using the
\tag{command} tag inside that block:

\begin{verbatim}
<!-- SECTION: Vulkan API interface definitions -->
<feature api="vulkan" name="VK_VERSION_1_0" number="1.0">
    ...
    <require comment="Device initialization">
        <command name="vkCreateInstance"/>
    </require>
    ...
</feature>
\end{verbatim}

The \tag{require} block may include a \attr{comment} attribute whose value
is a descriptive comment of the contents required within that block. The
comment is not currently used in header generation, but might be in the
future, so use comments which are polite and meaningful to users of
\code{vulkan.h}.

Then go to the \tag{commands} tag and add a new \tag{command} tag defining
the command, preferably sorted into alphabetic order with other commands for
ease of reading, as follows:

\begin{verbatim}
<!-- SECTION: Vulkan command definitions -->
<commands>
    ...
    <command>
        <proto><type>VkResult</type>
               <name>vkCreateInstance</name></proto>
        <param>const <type>VkInstanceCreateInfo</type>*
               <name>pCreateInfo</name></param>
        <param><type>VkInstance</type>*
               <name>pInstance</name></param>
    </command>
    ...
</commands>
\end{verbatim}

The \tag{proto} tag defines the return type and function name of the
command. The \tag{param} tags define the command's parameters in the order
in which they're passed, including the parameter type and name. The contents
are laid out in the same way as the structure \tag{member} tags described
previously.


\subsection{More Complicated API Representations}

The registry schema can represent a good deal of additional information, for
example by creating multiple \tag{feature} tags defining different API
versions and extensions. This capability is not yet relevant to Vulkan.
Those capabilities will be documented as they are needed.


\subsection{More Complicated Output Formats And Other Languages}

The registry schema is oriented towards C-language APIs. Types and commands
are defined using syntax which is a subset of C, especially for structure
members and command parameters. It would be possible to use a
language-independent syntax for representing such information, but since we
are writing a C API, any such representation would have to be converted into
C anyway at some stage.

The \code{vulkan.h} header is written using an {\bf output generator} object
in the Python scripts. This output generator is specialized for C, but the
design of the scripts is intended to support writing output generators for
other languages as well as purposes such as documentation (e.g. generating
Asciidoc fragments corresponding to types and commands for use in the API
specification and reference pages). When targeting other languages, the
amount of parsing required to convert type declarations into other languages
is small. However, it will probably be necessary to modify some of the
boilerplate C text, or specialize the tags by language, to support such
generators.


\subsection{Additional Semantic Tagging}

The schema is being extended to support semantic tags describing
various properties of API features, such as:

\begin{itemize}
\item constraints on allowed scalar values to function parameters (non-NULL,
      normalized floating-point, etc.)
\item length of arrays corresponding to function pointer parameters
\item miscellaneous properties of commands such as whether the application
      or system is responsible for threadsafe use; which queues they may be
      issued on; whether they are aliases or otherwise related to other
      commands; etc.
\end{itemize}

These tags will be used by other tools for purposes such as helping create
validation layers, generating serialization code, and so on. We'd like to
eventually represent everything about the API that is amenable to automatic
processing within the registry schema. Please make suggestions on the Gitlab
issue tracker.


\subsection{Stability of the XML Database and Schema}
\label{general:stability}

The Vulkan XML schema is evolving in response to corresponding
changes in the Vulkan API and ecosystem. Most such change will probably be
confined to adding attributes to existing tags and properly expressing the
relationships to them, and making API changes corresponding to accepted
feature requests. Changes to the schema should be described in the change
log of this document (see section~\ref{changelog}). Changes to the
\code{.xml} files and Python scripts are logged in Gitlab history.


\section{Change Log}
\label{changelog}

\begin{itemize}
\item 2016/02/07 - Add \attr{vendorids} tags for Khronos vendor IDs.
\item 2015/12/10 - Add \attr{author} and \attr{contact} attributes for
      \tag{extension} tags.
\item 2015/12/07 - Move \code{vulkan/vulkan.h} to a subdirectory.
\item 2015/12/01 - Add \tag{tags} tags for author tags.
\item 2015/11/18 - Bring documentation and schema up to date for extension
      enumerants.
\item 2015/11/02 - Bring documentation and schema up to date with several
      recent merges, including \tag{validity} tags. Still out of date WRT
      extension enumerants, but that will change soon.
\item 2015/09/08 - Rename {\tt threadsafe} attribute to \attr{externsync},
      and {\tt implicitunsafeparams} tag to \attr{implicitexternsync}.
\item 2015/09/07 - Update \tag{command} tag description to remove the
      \attr{threadsafe} attribute and replace it with a combination of
      \attr{threadunsafe} attributes on individual parameters, and
      \tag{implicitunsafeparams} tags describing additional unsafe objects
      for the command.
\item 2015/08/04 - Add \code{basetype} and \code{funcpointer}
      \attr{category} values for type tags, and explain the intended use and
      order in which types in each category are emitted.
\item 2015/07/02 - Update description of Makefile targets. Add
      descriptions of \attr{threadsafe}, \attr{queues}, and
      \attr{renderpass} attributes of \tag{command} tags in
      section~\ref{tag:command}, and of \attr{modified} attributes of
      \tag{param} tags in section~\ref{tag:command:param}
\item 2015/06/17 - add descriptions of allowed \attr{category} attribute
      values of \tag{type} tags, used to group and sort related categories
      of declarations together in the generated header.
\item 2015/06/04 - Add examples of making changes and additions to the
      registry in appendix~\ref{examples}.
\item 2015/06/03 - Move location to new ``vulkan'' Git repository. Add
      definition of \tag{type} tags for C struct/unions. Start adding
      examples of making changes in appendix~\ref{examples}.
\item 2015/06/02 - Branch from OpenGL specfile documentation and bring up to
      date with current Vulkan schema.
\end{itemize}

% \input{readme.ind}

% \printindex

\end{document}


% \printindex

\end{document}


% \printindex

\end{document}


% \printindex

\end{document}
